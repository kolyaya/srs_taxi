\usepackage{cmap}					% поиск в PDF
\usepackage[T2A]{fontenc}			% кодировка
\usepackage[utf8]{inputenc}			% кодировка исходного текста
\usepackage[english,russian]{babel}	% локализация и переносы

\usepackage[titletoc]{appendix}     % для приложений
\PassOptionsToPackage{hyphens}{url} % |
\usepackage{hyperref} 				% | для длинных ссылок

%%% Страница
	%\usepackage{extsizes} % Возможность сделать 14-й шрифт
		
	% Исключает абзацы	
		\usepackage{parskip}

    % Нумерация с:
		\setcounter{secnumdepth}{5}
        \setcounter{tocdepth}{4}

    % Счетчик требований
        \newcounter{sr}
         \newcommand{\sr}[1]{%
          \addtocounter{sr}{1}
          [SR-\arabic{sr}] #1%
          }

    % Счетчик фильтров
      \newcounter{flt}
         \newcommand{\flt}[2]{%
          \addtocounter{flt}{1}
          \refstepcounter{flt}\label{#1}
          [FLT-\arabic{flt}] #2%
          }  

    % Счетчик входных/выходных данных
      \newcounter{crdt}
         \newcommand{\crdt}[2]{%
          \addtocounter{crdt}{1}
          \refstepcounter{crdt}\label{#1}
          [CRDT-\arabic{crdt}] #2%
          }    

    % Счетчик действий сервера.
        \newcounter{srvact}
         \newcommand{\srvact}[2]{%
          \addtocounter{srvact}{1}
          \refstepcounter{srvact}\label{#1}
          [SRVACT-\arabic{srvact}] #2%
          }    

    % Счетчик константных значений

    	\newcounter{stat}
         \newcommand{\stat}[2]{%
          \addtocounter{stat}{1}
          \refstepcounter{stat}\label{#1}
          [STAT-\arabic{stat}] #2%
          } 

    % Счетчик уведомлений диспетчерской

    	\newcounter{ntdsp}
         \newcommand{\ntdsp}[2]{%
          \addtocounter{ntdsp}{1}
          \refstepcounter{ntdsp}\label{#1}
          [NTDSP-\arabic{ntdsp}] #2%
          } 

    % Счетчик уведомлений диспетчерской

    	\newcounter{nttax}
         \newcommand{\nttax}[2]{%
          \addtocounter{nttax}{1}
          \refstepcounter{nttax}\label{#1}
          [NTTAX-\arabic{nttax}] #2%
          } 

    % Счетчик элементов таксометра

    	\newcounter{eltax}
         \newcommand{\eltax}[2]{%
          \addtocounter{eltax}{1}
          \refstepcounter{eltax}\label{#1}
          [ELTAX-\arabic{eltax}] #2%
          }     

    % Теоремы 

    	\usepackage{amsthm}
    	%\newcounter{alg_counter}
    	\theoremstyle{defenition}
    	\newtheorem{alg}{ALG -}[section]

	% Простой способ задавать поля
		\usepackage{geometry} 
			\geometry{top=25mm}
			\geometry{bottom=35mm}
			\geometry{left=35mm}
			\geometry{right=20mm}

	% Колонтитулы
		\usepackage{fancyhdr} 
		 	\pagestyle{fancy}
		 	\renewcommand{\footrulewidth}{1pt} 
		 	\renewcommand{\headrulewidth}{1pt} % Толщина линейки, отчеркивающей верхний колонтитул
		 	\rfoot{Такси "Ритм" 2014}
		 	\lfoot{Техническая спецификация}


	% Узнать, сколько всего страниц в документе.
		\usepackage{lastpage} 

	% Модификаторы начертания
		\usepackage{soulutf8} 

	 	\usepackage{hyperref}
		\usepackage[usenames,dvipsnames,svgnames,table,rgb]{xcolor}
		\hypersetup{				% Гиперссылки
		    unicode=true,           % русские буквы в раздела PDF
		    pdftitle={Заголовок},   % Заголовок
		    pdfauthor={Автор},      % Автор
		    pdfsubject={Тема},      % Тема
		    pdfcreator={Создатель}, % Создатель
		    pdfproducer={Производитель}, % Производитель
		    pdfkeywords={keyword1} {key2} {key3}, % Ключевые слова
		    colorlinks=true,       	% false: ссылки в рамках; true: цветные ссылки
		    linkcolor=red,          % внутренние ссылки
		    citecolor=green,        % на библиографию
		    filecolor=magenta,      % на файлы
		    urlcolor=blue           % на URL
		}

%%% Работа со списками
	\usepackage{enumitem}
	\newlist{longenum}{enumerate}{5}
	\setlist[longenum,1]{label=\roman*)}
	\setlist[longenum,2]{label=\alph*)}
	\setlist[longenum,3]{label=\arabic*)}
	\setlist[longenum,4]{label=(\roman*)}
	\setlist[longenum,5]{label=(\alph*)}

%%% Работа с таблицами
	\usepackage{array, tabularx, tabulary, booktabs} % Дополнительная работа с таблицами
	\usepackage{longtable} % Длинные таблицы
	\usepackage{multirow} % Слияние строк в таблице
	\usepackage{caption,fixltx2e}
	\usepackage[flushleft]{threeparttable}

%%% Работа с картинками
	\usepackage{graphicx}  % Для вставки рисунков
	\graphicspath{{Android_app/}{diagrams/}}  % папки с картинками
	\setlength\fboxsep{3pt} % Отступ рамки \fbox{} от рисунка
	\setlength\fboxrule{1pt} % Толщина линий рамки \fbox{}
	\usepackage{wrapfig} % Обтекание рисунков и таблиц текстом

%%% Программирование
	\usepackage{etoolbox} % логические операторы

%%% Дополнительная работа с математикой
	\usepackage{amsmath,amsfonts,amssymb,amsthm,mathtools} % AMS
	\usepackage{icomma} % "Умная" запятая: $0,2$ --- число, $0, 2$ --- перечисление

%% Номера формул
	%\mathtoolsset{showonlyrefs=true} % Показывать номера только у тех формул, на которые есть \eqref{} в тексте.
	%\usepackage{leqno} % Нумерация формул слева

%% Свои команды
	\DeclareMathOperator{\sgn}{\mathop{sgn}}

%% Перенос знаков в формулах (по Львовскому)
	\newcommand*{\hm}[1]{#1\nobreak\discretionary{}
	{\hbox{$\mathsurround=0pt #1$}}{}}

%%Enumerate
	\usepackage{enumitem}