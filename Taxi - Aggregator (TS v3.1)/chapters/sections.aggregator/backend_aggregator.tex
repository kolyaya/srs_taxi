\section{Серверная часть}

	\subsection{Интеграция с Яндекс.Такси}

		DESC: Дальнейшее описание требований агрегатора подразумевают под собой реализованную интеграцию с этим сервисом.  

		\begin{itemize}

		\item{
			TITLE: Интеграция с сервисом Яндекс.Такси\\
			\sr{Агрегатор интегрирован с сервисом Яндекс.Такси согласно протоколу предоставленному компанией Яндекс. Протокол прикреплен к спецификации.}\\
			PRIOR: CRITICAL\\
		}

		\end{itemize}

	\subsection{Интеграция Служб Такси с Агрегатором}

		DESC: Сервис предоставляет API, с помощью которого Службы Такси (партнеров) и Агрегатор обмениваются информацией о заказах, тарифах и водителях. За основу взят програмный интерфейс сервиса Яндекс.Такси.\\

		\subsubsection{Общие требования}

			\begin{itemize}

				\item {

					TITLE: Единая база агрегатора.\\
					\sr{Агрегатор используя информацию полученную от Служб Такси агрегатор-партнеров ведет единую базу водителей.}\\
					PRIOR: CRITICAL\\

				}

			\end{itemize}

		    \begin{table}[h]
	            \begin{center}
	            \caption {Переодичность запросов данных агрегатора (Константные значения).}
	            \setlength{\extrarowheight}{2mm}
	            \begin{tabular}{|p{3cm}|p{6cm}|p{4cm}|}
	               \hline     \textbf{ID} & \textbf{Описание} & \textbf{Значение}\\ [2mm]


	               \hline \stat{request_freq_driver_profiles}{} & Запросы получение данных о профилях водителей. & Раз в 24 ч.\\ [2mm]
	               
	               \hline

	            \end{tabular}
	            \end{center}
            \end{table}

		\subsubsection{Профили водителей} \label{aggregator_api_for_ts_drivers_profiles}

			DESC: Агрегатор запрашивает у сервера агрегатор-партнера профили всех имеющихся водителей. Профиль каждого водителя состоит из двух частей: личные данные водителя и характеристики ТС.

			\begin{itemize}

				\item {

					TITLE: Получение данных о профилях водителей.\\
					\sr{Сервер агрегатора запрашивает профили водителей с переодичностью в STAT-\ref{request_freq_driver_profiles}.}\\
					\sr{Описание запрашиваемых данных находиться в разделе \ref{aggregator_api_for_ts_drivers_profiles}.}\\
					PRIOR: CRITICAL\\
				}

			\end{itemize}

			\setlength{\extrarowheight}{2mm}
				\begin{longtable}{|p{4cm}|p{9cm}|}
					\caption {Параметры водителей}\\

				    \hline	\textbf{Параметр}&\textbf{Описание} \\ [2mm]
				    \endfirsthead
				    \hline	\textbf{Параметр}&\textbf{Описание} \\ [2mm]
				    \endhead

					\hline	Идентификатор &  Присваивается Службой Такси. Каждый идентификатор водителя должен быть уникален в рамках Службы Такси. Также идентификаторы должны быть не длиннее 32 символов, и должны состоять только из цифр. \\ [2mm]

				    \hline	Идентификатор тарифа & Для одного водителя может быть указано несколько тарифов. Первым в списке должен идти основной тариф, затем перечисляются тарифы, по которым Водитель готов взять заказ, если более высокий тариф не устраивает клиента. \\ [2mm]

					\hline	Информация о водителе & \begin{itemize} 
														\item ФИО водителя на русском языке.
														\item Номер телефона водителя в формате “+7ХХХХХХХХХХ”.
														\item Год рождения водителя.
														\item Элемент, содержащий характеристики машины.
													\end{itemize} \\ [2mm]

					\hline	Характеристики машины & \begin{itemize} 
														\item Марка и модель автомобиля.
														\item Год выпуска автомобиля.
														\item Цвет машины.
														\item Государственный номер и регион регистрации автомобиля. Причем все буквы номера — заглавные буквы кириллицы.
														\item Номер разрешения на перевозки.
														\item Перевозка животных.
															\begin{itemize} 
																\item «yes» — в машине можно перевозить животных;
																\item «no» — животных перевозить нельзя.
															\end{itemize} 
														\item Квитанция
															\begin{itemize} 
																\item «yes» — водитель может выписать квитанцию об оплате заказа (документ строгой отчетности);
																\item «no» — водитель не может выписать квитанцию.
															\end{itemize}
														\item Детское кресло.
															\begin{itemize} 
																\item <возраст ребенка>
																\item «no» — детского кресла нет.
															\end{itemize}
														\item Автомобильный кондиционер.
															\begin{itemize} 
																\item «yes» — в машине есть кондиционер;
																\item «no» — кондиционера нет.
															\end{itemize}
														\item Разрешено ли курение.
															\begin{itemize} 
																\item «yes» — водитель не курит, пассажирам курить запрещено;
																\item «no» — пассажирам курить разрешено.
															\end{itemize}
														\item Кузов «универсал».
															\begin{itemize} 
																\item «yes» — машина с кузовом «универсал»;
																\item «no» — машина с другим кузовом.
															\end{itemize}
													\end{itemize} \\ [2mm]
					\hline 
				\end{longtable}

	\subsection{Обработка заказов}

		\subsubsection{Обработка заказов Яндекс.Такси}

		\subsubsection{Обработка партнерских заказов}
