\section{Серверная часть}

	\subsection{Интеграция с Яндекс.Такси}

		DESC: Дальнейшее описание требований агрегатора подразумевают под собой реализованную интеграцию с этим сервисом.  

		TITLE: Интеграция с сервисом Яндекс.Такси\\
		\sr{Агрегатор интегрирован с сервисом Яндекс.Такси согласно протоколу предоставленному компанией Яндекс. Протокол прикреплен к спецификации.}\\
		PRIOR: CRITICAL\\

	\subsection{Интеграция Служб Такси с Агрегатором}

		DESC: Сервис предоставляет API, с помощью которого Службы Такси (партнеров) и Агрегатор обмениваются информацией о заказах, тарифах и водителях. За основу взят програмный интерфейс сервиса Яндекс.Такси.\\

		Общие требования:
		\begin{enumerate}
			\item Агрегатор используя информацию полученную от Служб Такси агрегатор-партнеров ведет единую базу водителей.
			\item Запросы производятся с периодичностью установленной владельцем.
		\end{enumerate}

		\subsubsection{Профили водителей}

			DESC: Агрегатор запрашивает у сервера агрегатор-партнера профили всех имеющихся водителей. Профиль каждого водителя состоит из двух частей: личные данные водителя и характеристики ТС.

			\setlength{\extrarowheight}{2mm}
				\begin{longtable}{|p{4cm}|p{9cm}|}
					\caption {Параметры водителей}\\

				    \hline	\textbf{Параметр}&\textbf{Описание} \\ [2mm]
				    \endfirsthead
				    \hline	\textbf{Параметр}&\textbf{Описание} \\ [2mm]
				    \endhead

					\hline	Идентификатор &  Присваивается Службой Такси. Каждый идентификатор водителя должен быть уникален в рамках Службы Такси. Также идентификаторы должны быть не длиннее 32 символов, и должны состоять только из цифр. \\ [2mm]
				    \hline	Идентификатор тарифа & Для одного водителя может быть указано несколько тарифов. Первым в списке должен идти основной тариф, затем перечисляются тарифы, по которым Водитель готов взять заказ, если более высокий тариф не устраивает клиента. \\ [2mm]
					\hline	Информация о водителе & \begin{itemize} 
														\item ФИО водителя на русском языке
														\item Номер телефона водителя в формате “+7ХХХХХХХХХХ”
														\item Год рождения водителя
														\item Элемент, содержащий характеристики машины
													\end{itemize} \\ [2mm]
					\hline	\textbf{Характеристики машины:}	&\\ [2mm]					
					\hline	Марка и модель автомобиля &  \\ [2mm]
					\hline	Год выпуска автомобиля &  \\ [2mm]
					\hline	Цвет машины &  \\ [2mm]
					\hline	Государственный номер и регион регистрации автомобиля. Причем все буквы номера — заглавные буквы кириллицы. &  \\ [2mm]
					\hline	Номер разрешения на перевозки &  \\ [2mm]
					\hline	Перевозка животных. & 	\begin{itemize} 
														\item «yes» — в машине можно перевозить животных;
														\item «no» — животных перевозить нельзя.
													\end{itemize} \\ [2mm]
					\hline	Квитанция & \begin{itemize} 
											\item «yes» — водитель может выписать квитанцию об оплате заказа (документ строгой отчетности);
											\item «no» — водитель не может выписать квитанцию.
										\end{itemize} \\ [2mm]
					\hline	Детское кресло. & 	\begin{itemize} 
													\item <возраст ребенка> — детское кресло есть, предназначено для детей указанного возраста. Возраст указывается в виде промежутка между минимальным и максимальным значением, например: «0.11-4». Возраст меньше года указывается в формате «0.<количество месяцев>» (11 месяцев следует записать как «0.11»). Возраст больше года указывается как целое число лет.
													\item «no» — детского кресла нет.
												\end{itemize} \\ [2mm]
					\hline	Автомобильный кондиционер. & 	\begin{itemize} 
																\item «yes» — в машине есть кондиционер;
																\item «no» — кондиционера нет.
															\end{itemize} \\ [2mm]
					\hline	Разрешено ли курение. & \begin{itemize} 
														\item «yes» — водитель не курит, пассажирам курить запрещено;
														\item «no» — пассажирам курить разрешено.
													\end{itemize} \\ [2mm]
					\hline	Кузов «универсал». &\begin{itemize} 
													\item «yes» — машина с кузовом «универсал»;
													\item «no» — машина с другим кузовом.
												\end{itemize}  \\ [2mm]
					\hline 
				\end{longtable}

	\subsection{Обработка заказов}

		\subsubsection{Обработка заказов Яндекс.Такси}

		\subsubsection{Обработка партнерских заказов}
