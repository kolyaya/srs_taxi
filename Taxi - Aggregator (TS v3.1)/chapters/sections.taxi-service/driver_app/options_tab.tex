\subsection{Настройки} \label{driver_app_settings_tab}

    DESC: Во вкладке “Настройки” пользователь может настроить мобильное приложение.

    \begin{itemize}

      \item{
      TITLE: Элементы вкладки\\
      \sr{По нажатию на вкладку должен появлятся выпадающий список, элементы которого перечислены и описаны в таблице \ref{options_tab_elements}.\\ NOTE: Некоторые элементы доступны только в определенном режиме работы мобильного приложения. В каком режиме работы доступен каждый элемент описано в столбце “Отображение в режиме”.}\\
      PRIOR: MODERATE\\}

      \item{
      TITLE: Отображение элементов списка\\
      \sr{Элемент списка состоит из двух частей: 
        \begin{itemize} 
          \item{Иконка. Расположена слева. Для каждого элемента она соответствующая, описана в столбце “Иконка” таблицы \ref{options_tab_elements}.}
          \item{Справа от иконки расположено название элемента.}
        \end{itemize}}
      PRIOR: MODERATE\\}

    \end{itemize}

    \setlength{\extrarowheight}{2mm}
        \begin{longtable}{|p{3cm}|p{3cm}|p{2cm}|p{7cm}|}
            
        \caption {Элементы выпадающего списка вкладки “Настройки”} \label{options_tab_elements} \\
          \hline  \textbf{Название} & \textbf{Отбражение в режиме} & \textbf{Иконка} & \textbf{Требования/Описание} \\ [2mm]
          \endfirsthead
          \hline  \textbf{Название} & \textbf{Отбражение в режиме} & \textbf{Иконка} & \textbf{Требования/Описание} \\ [2mm]
          \endhead

          \hline Checkbox “Занят” & Пользователь авторизован & N/A & \sr{При установке checkbox-а, мобильное приложение отправляет серверу запрос на изменение статуса. Пока checkbox установлен на водителя накладываются ограничения в соответствии с требованиями владельца.}\\ [2mm]

          \hline Вкладка “Уведомления” & Пользователь авторизован/Автономный & Колокол & \sr{Описание в разделе \ref{options_tab_notifications_driver_app}.}\\ [2mm]

          \hline Вкладка “Тарифы” & Пользователь авторизован & Кошелек и доллар & 
          \begin{itemize} 
            \item{
            \sr{При нажатии на вкладку “Тарифы” выдвигается информативное модальное окно, в котором находится актуальная информация о тарифах.}} 
            \item{
              \sr{Информация о тарифе представляет собой карточку. В каждой карточке должно быть указано: 
                \begin{itemize} 
                  \item{Название тарифа} 
                  \item{Минимальная стоимость поездки} 
                  \item{Время бесплатного ожидания и стоимость минуты простоя по истечению этого времени.} 
                  \item{
                    Внизу карточки находится кнопка “подробнее...”. По нажатию на кнопку должна появляться следующая информация о тарифе: 
                    \begin{itemize} 
                      \item{Тарифы по Москве} 
                      \item{Тарифы за пределами МКАД} 
                      \item{Расценки на поездки в аэропорты} 
                      \item{Поездки из одного аэропорта в другой} 
                    \end{itemize}
                    } 
                \end{itemize}
              }
            } 
          \end{itemize}\\ [2mm]

          \hline Вкладка “Вопрос/Ответ” & Пользователь авторизован/Автономный & Знак вопроса и знак восклицания & \sr{При нажатии на вкладку “Вопрос/Ответ” выдвигается информативное модальное окно, в котором находится список часто задаваемых вопросов и ответы на них.}\\ [2mm]

          \hline Вкладка “Контакты” & Пользователь авторизован & Анфас человека и телефонная трубка & \sr{При нажатии на вкладку “Контакты” выдвигается информативное модальное окно, в котором находится информация о контактах службы владельца.}\\ [2mm]
    
          \hline Вкладка “Тех. поддержка” & Пользователь авторизован & Молоток с имитацией удара & \sr{В этой вкладке пользователь имеет возможность сообщить об ошибке или написать пожелание по доработке разработчикам приложения. При нажатии на вкладку “Тех. поддержка” выдвигается модальное окно, элементы которого описаны в таблице \ref{options_tab_tech_help_elements}.}\\ [2mm]

          \hline Вкладка “Основные настройки” & Пользователь авторизован/Автономный & Две шестеренки & \sr{Описание в разделе \ref{options_tab_global_options}.}\\ [2mm]
     
          \hline Вкладка “Выбрать сервер” & Пользователь авторизован/Автономный & Две шестеренки & \sr{При нажатии на вкладку “Выбрать сервер” на экране появляется модальное окно с элементами, которые описаны в таблице \ref{options_tab_select_server_elements}.}\\ [2mm]

          \hline Кнопка “Войти” & Автономный & N/A & \sr{Описание в разделе \ref{options_tab_authorization}.}\\ [2mm]

          \hline Кнопка “Выход” & Пользователь авторизован/Автономный & N/A & \sr{По нажатию кнопки на экране появляется модальное окно с запросом подтверждения закрытия приложения.}\\ [2mm]
  
          \hline

        \end{longtable}

      %%%Техподдержка
      \begin{table}
        \begin{center}
        \caption{Элементы вкладки “Тех. поддержка”}
        \label{options_tab_tech_help_elements}
        \setlength{\extrarowheight}{2mm}
        \begin{tabular}{|p{3cm}|p{3cm}|p{9cm}|}
           \hline   \textbf{ID}&  \textbf{Название}&\textbf{Требования/Описание} \\ [2mm]

           \hline \eltax{tech_help_message}{} & Написать сообщение & \sr{Поле для ввода. Сюда пользователь может ввести содержание своего пожелания по работе приложения.}\\ [2mm]

           \hline \eltax{tech_help_send_message}{} & Отправить & \sr{Кнопка. При нажатии на кнопку мобильное приложение отправляет соответствующий запрос, содержащий в себе сообщение, написанное пользователем в поле для ввода ELTAX-\ref{tech_help_message}, на сервер.}\\ [2mm]

           \hline \eltax{tech_help_send_message}{} & Закрыть & \sr{Кнопка. При нажатии закрывается модальное окно.}\\ [2mm]

           \hline
        \end{tabular}
        \end{center}
      \end{table}

      %%%Выбрать сервер
      \begin{table}
        \begin{center}
        \caption{Элементы вкладки “Выбрать сервер”}
        \label{options_tab_select_server_elements}
        \setlength{\extrarowheight}{2mm}
        \begin{tabular}{|p{3cm}|p{3cm}|p{9cm}|}
           \hline   \textbf{ID}&  \textbf{Название}&\textbf{Требования/Описание} \\ [2mm]

           \hline \eltax{select_server_url}{} & Url-адрес & \sr{Поле для ввода. Сюда пользователь вводит url-адрес сервера, к которому хочет подключиться.}\\ [2mm]

           \hline \eltax{select_server_set_button}{} & Установить & \sr{Кнопка. Подтверждает ввод нового url-адреса, приложение совершает попытку подключения к новому серверу.}\\ [2mm]

           \hline
        \end{tabular}
        \end{center}
      \end{table}  

    \subsubsection{Уведомления} \label{options_tab_notifications}

      \paragraph{Функциональные требования} \mbox{}\\
        \begin{itemize}

          \item{TITLE: Новое уведомление. \\
                \sr{Новое уведомление помечается как непрочитанное, если пользователь не находится во вкладке "Уведомления" DriverApp. Инкрементируется счетчик непрочитанных уведомлений. Если на момент поступления нового уведомления пользователь находится во вкладке "Уведомления" DriverApp, новое уведомление помечается как прочитанное и счетчик непрочитанных уведомлений не инкрементируется.} \\
                PRIOR: MODERATE \\}

          \item{TITLE: Прочитанное уведомление. \\
                \sr{Уведомления считаются прочитанными, если была открыта вкладка "Уведомления" в DriverApp. Счетчик непрочитанных уведомлений при этом обнуляется.} \\
                PRIOR: MODERATE \\}

          \item{TITLE: Счетчик непрочитанных уведомлений. \\
                \sr{Означает количество непрочитанных уведомлений на данный момент. Инкрементируется, если появилось новое непрочитанное уведомление. Обнуляется при просмотре непрочитанных уведомлений через вкладку "Уведомления" в DriverApp.} \\
                PRIOR: MODERATE \\}
          
          \item{TITLE: Иконка. \\
                \sr{Выбирается в зависимости от инициатора уведомления из "Таблицы инициаторов уведомления".} \\
                PRIOR: MODERATE \\}

        \end{itemize}

      \paragraph{Уведомления в панели уведомлений Android(ПУА)} \mbox{}\\

        \subparagraph{Функциональные требования} \mbox{}\\

          \begin{itemize}

            \item{TITLE: Появление нового уведомления. \\
                  \sr{При появлении нового уведомления от какого-либо инициатора уведомления, оно выводится в панели уведомлений Android(ПУА). Если в данный момент появляется сразу несколько новых уведомлений, счетчик непрочитанных уведомлений инкрементируется на их количество.} \\
                  PRIOR: MODERATE \\}

            \item{TITLE: Открытие уведомления. \\
                  \sr{По нажатию на уведомление в ПУА, открывается DriverApp на вкладке "Уведомления". При этом, счетчик непрочитанных уведомлений обнуляется.} \\
                  PRIOR: MODERATE \\}

            \item{TITLE: Разрешенные действия с уведомлениями. \\
                  \sr{Разрешается нажать на уведомление в ПУА. В этом случае откроется DriverApp на вкладке "Уведомления", счетчик непрочитанных уведомлений обнулится.} \\ 
                  \sr{Разрешается убрать уведомление из ПУА путем скрола уведомления в сторону. В этом случае не произойдет никаких действий, счетчик непрочитанных уведомлений не обнулится.} \\
                  PRIOR: MODERATE \\}

            \item{TITLE: Интерфейс уведомления. \\
                  \sr{В панели уведомлений Android уведомления отображаются согласно \href{https://www.evernote.com/shard/s389/sh/8d4e738c-61f0-4500-98bb-5882d89c9906/9ab0dc75caa4a5144ac1e5042e808b4f}{прототипу}.} \\
                  PRIOR: MODERATE \\}

            \item{TITLE: Строка "Новых уведомлений"(верхняя строка). \\
                  \sr{Отображется: [Новых уведомлений: ] + [общее количество непрочитанных на данный момент уведомлений].} \\
                  PRIOR: MODERATE \\}

            \item{TITLE: Строка "Последних уведомлений"(нижняя строка). \\                          
                  \sr{Отображется: [последнее непрочитанное уведомление] + [,] + [предпоследнее непрочитанное уведомление] + [...] в конце, если непрочитанных сообщений больше двух.} \\
                  PRIOR: MODERATE \\}

            \item{TITLE: Дата. \\
                  \sr{Отображается время появления уведомления в формате "Часы:Минуты".} \\
                  PRIOR: MODERATE \\}

          \end{itemize}

      \paragraph{Уведомления в DriverApp во вкладке Уведомления} \mbox{}\\ \label{options_tab_notifications_driver_app}

        \subparagraph{Функциональные требования} \mbox{}\\

          \begin{itemize}

            \item{TITLE: Появление нового уведомления. \\
                  \sr{Непрочитанные уведомления появляются сверху списка во вкладке "Уведомления".} \\
                  PRIOR: MODERATE \\}

            \item{TITLE: Открытие вкладки "Уведомления". \\
                  \sr{При открытии вкладки "Уведомления", обнуляется счетчик непрочитанных уведомлений.} \\
                  PRIOR: MODERATE \\}

            \item{TITLE: Разрешенные действия с уведомлениями. \\
                  \sr{Уведомления во вкладке "Уведомления" разрешается только просмотреть.} \\
                  PRIOR: MODERATE \\}

            \item{TITLE: Интерфейс уведомления. \\
                  \sr{Во вкладке "Уведомления" уведомления отображаются согласно \href{https://www.evernote.com/shard/s389/sh/f7d9319d-bb3a-47de-b857-5a37e440419c/774bb30105ccb402d93522ae9e333b5e}{прототипу}.} \\
                  PRIOR: MODERATE \\}

            \item{TITLE: Строка "Уведомления"(верхняя строка). \\
                  \sr{Отображается содержание уведомления.} \\
                  PRIOR: MODERATE \\}

            \item{TITLE: Строка "Дата и время"(нижняя строка). \\
                  \sr{Отображается дата и время появления уведомления в формате "ДД.ММ.ГГГГ Часы:Минуты".} \\
                  PRIOR: MODERATE \\}

          \end{itemize}

      % Таблица уведомлений
      \label{taxometr_notifications_table}
      \setlength{\extrarowheight}{2mm}
          \begin{longtable}{|p{3cm}|p{4cm}|p{5cm}|p{3cm}|}
              \caption {Таблица уведомлений}\\

              \hline     \textbf{ID}&\textbf{Название}&\textbf{Содержимое уведомления} & \textbf{Инициатор}\\ [2mm]
              \endfirsthead
              \hline     \textbf{ID}&\textbf{Название}&\textbf{Содержимое уведомления} & \textbf{Инициатор}\\ [2mm]
              \endhead
              
              \hline  \nttax{notif_driver_of_remove_driver_from_the_order}{} & Снятие водителя с заказа. & Вы сняты с заказа + [№Заказа]. & SRVACT-\ref{act_remove_driver_driver_notification} \\ [2mm]

              \hline \nttax{notif_of_new_order_in_reserve}{} & Добавление нового заказа в резерв. & Заказ + [№Заказа] + поступил в резерв. & \\ [2mm]       %TODO: Дописать SRVACT-\ref{act_add_new_order_in_reserve}

              \hline
          \end{longtable}

    \subsubsection{Основные настройки} \label{options_tab_global_options}

      \paragraph{Тарифы}\mbox{}\\

        \begin{itemize}
          
          \item{
          TITLE: Отображение в режимах работы приложения\\
          \sr{Отображается как при автономном режиме работы, так и в режиме работы с авторизованным пользователем.}\\
          PRIOR: MODERATE\\}

          \item{
          TITLE: Элементы вкладки\\
          \sr{По нажатию на вкладку Тарифы на экране должно открываться новое окно, в котором будут присутствовать элементы, перечисленные и описанные в таблице \ref{options_tab_global_options_tarifs}.}\\
          PRIOR: MODERATE\\}
        \end{itemize}

        \begin{table}
        \begin{center}
        \caption{Элементы вкладки “Тарифы”}
        \label{options_tab_global_options_tarifs}
        \setlength{\extrarowheight}{2mm}
        \begin{tabular}{|p{3cm}|p{3cm}|p{9cm}|}
           \hline   \textbf{ID}&  \textbf{Название}&\textbf{Требования/Описание} \\ [2mm]

           \hline \eltax{global_options_back_button}{} & Назад & \sr{Кнопка. Располагается в левом верхнем углу. По нажатию приложение возвращается в предыдущее окно.}\\ [2mm]

           \hline \eltax{global_options_costs_per_min}{} & Стоимость минуты поездки & \sr{Поле для ввода. Пользователь вводит числовое значение в это поле. Введенное число означает стоимость минуты поездки в рублях.}\\ [2mm]

           \hline \eltax{global_options_set_button}{} & Установить & \sr{Кнопка. По нажатию на кнопку на сервер отправляется сообщение со значением поля ELTAX-\ref{global_options_costs_per_min}. Приложение возвращается в предыдущее окно.}\\ [2mm]

           \hline
        \end{tabular}
        \end{center}
      \end{table}

      \paragraph{Фото}\mbox{}\\                                                                                           %%% Не понятно, зачем?

        Пользователь имеет возможность сделать фото.

      \paragraph{Уведомления}\mbox{}\\

        DESC: Пользователь имеет возможность настроить от каких источников он хочет получать уведомления.

        \begin{itemize}

          \item{
          TITLE: Отображение в режимах работы приложения\\
          \sr{Отображается только в режиме работы с авторизованным пользователем.}\\
          PRIOR: MODERATE\\}

          \item{
            TITLE: Элементы вкладки\\
            \sr{Вкладка должна содержать:
              \begin{itemize}
                \item{Cписок из всех возможных источников уведомлений. Каждый источник уведомления состоит из следующих элементов: 
                  \begin{itemize} 
                    \item{Название} 
                    \item{Свитчер ON/OFF - включить/выключить получение уведомлений от соответствующего источника} 
                  \end{itemize}
                  Все источники уведомлений делятся на 3 категории. Они описаны в таблице \ref{options_tab_global_options_notif}.
                }
                \item{Кнопка “Назад”. По нажатию на нее происходит возврат на предыдущее окно.}
              \end{itemize}  
            }
            PRIOR: MODERATE\\
          }
        \end{itemize}

        \begin{table}
        \begin{center}
        \caption{Источники уведомлений}
        \label{options_tab_global_options_notif}
        \setlength{\extrarowheight}{2mm}
        \begin{tabular}{|p{5cm}|p{10cm}|}
           \hline   \textbf{Название}&\textbf{Требования/Описание} \\ [2mm]

           \hline Для водителя & Категория должна содержать следующие источники уведомлений: \begin{itemize} \item{Новый заказ} \item{Новое сообщение} \item{Уведомления во время работы таксометра} \end{itemize}\\ [2mm]

           \hline Таксометр & Категория должна содержать следующие источники уведомлений: \begin{itemize} \item{Поступил заказ в цепочку} \item{Заказ отменён} \item{Заказ из резерва отменён} \item{Смена тарифной зоны} \end{itemize}\\ [2mm]

           \hline Таксометр & Категория должна содержать следующие источники уведомлений: \begin{itemize} \item{Нет GPS} \item{Нет интернета} \end{itemize}\\ [2mm]

           \hline
        \end{tabular}
        \end{center}
      \end{table}