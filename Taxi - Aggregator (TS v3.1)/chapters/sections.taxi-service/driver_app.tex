\section{Водительское приложение}
	%\includegraphics[scale=0.25, keepaspectratio]{frog.jpg}
    %\includegraphics[width=10cm]{frog.jpg}
    %\includegraphics[draft]{frog.jpg}

  %Бар

  \subsection{Общие требования к мобильному приложению}

    TITLE: Отправка GPS данных на сервер\\
    \sr{Мобильное приложение оповещает сервер о местоположении водителя отсылая GPS координаты с периодичностью в STAT-\ref{periodicity_transfer_coordinates_taximeter_to_server} секунд.}\\
    PRIOR: MODERATE\\

  \subsection{Пользователи}

      \subsubsection{Роли, состояния приложения}

      		DESC: У приложения есть два состояния, они перечислены и описаны в таблице \ref{app_state}. 

             \begin{table}
             \begin{center}
             \label{app_state}
             \caption {Состояния приложений}
             \setlength{\extrarowheight}{2mm}
             \begin{tabular}{|p{5cm}|p{10cm}|}
                 \hline     \textbf{Состояние}&\textbf{Описание, уровень доступа} \\ [2mm]

                 \hline   Авторизованный пользователь & В этом состоянии пользователю доступны все элементы интерфейса и функциональные возможности приложения.\\ [2mm]
                 \hline   Автономный режим & В этом состоянии пользователю доступны следующие функции: \begin{itemize} \item Таксометр \item Навигатор \item \item\end{itemize}\\ [2mm]
                  \hline
             \end{tabular}
             \end{center}
             \end{table}

  	  \subsubsection{Авторизация}

  		DESC: Водитель имеет возможность ввести имя компании, свой позывной (логин) и пароль в специальном окне. Мобильное приложение отправляет эти данные на сервер для проверки и если они валидны то открывается соединение с сервером и все элементы интерфейса становятся доступны.

  		Требования: 
  		Если водитель не авторизован, то все элементы интерфейса кроме настроек, навигатора и таксометра становятся не доступны.
  		Вместе с позывным и паролем, при авторизации отправляется также GPS координаты - местоположение водителя.

  \subsection{Таксометр}


    3.2.3 Таксометр
    3.2.3.1 Работа с текущим заказом
    3.2.3.2 Просчет стоимости

  \subsection{Заказы}

    экран Заказа

  \subsection{Навигатор}

    DESC: В мобильное приложение встроен Яндекс.Навигатор.

    \subsubsection{Функциональные требования}

      \begin{itemize}
        \item 
        {TITLE: Запуск навигатора.\\
        \sr{При нажатии на вкладку "Навигатор", на устройстве запускается приложение "Яндекс.Навигатор". Если приложение не установлено, то пользователю предлагается загрузить приложение на устройство из Play Market}\\
        PRIOR: MODERATE\\}

        \item{
        TITLE: Построение маршрута для заказа.\\
        \sr{При выполнении текущего заказа в навигатор передаются координаты в зависимости от стадии заказа. По этим координатам строится маршрут.}\\
        \sr{Если статус заказа "Едет к клиенту", то в навигатор передаются: [текущее местоположение водителя] + [начальная точка заказа].}\\
        \sr{Если статус заказа "В пути", то в навигатор передаются: [начальная точка заказа] + [конечная точка заказа].}\\
        PRIOR: MODERATE\\}
      \end{itemize}

  \subsection{Робот}

    DESC: При переходе на вкладку “Робот” отображается список настроек и список режимов и опций для каждого из них.

    \subsubsection{Общие требования} \mbox{}\\

      \begin{itemize}

        \item{
          TITLE: Отправка настроек и опций на сервер.\\
          \sr{При изменении опций или параметров, мобильное приложение автоматически отправляет обновленные настройки на сервер.}\\
          PRIOR: MODERATE\\
          }

        \item{
          TITLE: Сохранение настроек.\\
          \sr{Для каждого пользователя настройки сохраняются в базе СТ.}\\
          NOTE: При перезапуске или при разрыве соединения сервер возвращает приложению последние записанные в базу СТ настройки для конкретного пользователя.\\
          PRIOR: MODERATE\\
          }

      \end{itemize}

    \subsubsection{Интерфейс вкладки} \mbox{}\\

      \begin{itemize}

        \item{
          TITLE: Деление экрана.\\
          \sr{Экран делится на две части, при помощи вертикальной линии, в соотношении 1 : 1.}\\          
          PRIOR: MODERATE\\}

        \item {
          TITLE: Расположение общих настроек.\\
          \sr{В левой части расположен список общих настроек робота. (Описание - \ref{driver_app_robot_tab_general_settings})}\\
          PRIOR: MODERATE\\}

        \item {
          TITLE: Расположение настроек робота.\\
          \sr{В правой части расположен список режимов работы робота. (Описание - \ref{driver_app_robot_tab_robot_settings})}\\
          PRIOR: MODERATE\\}

      \end{itemize}

    \subsubsection{Общие настройки} \label{driver_app_robot_tab_general_settings} \mbox{}\\

      \begin{itemize}

        \item{
          \sr{В списке должны присутствовать настройки, перечисленные в таблице \ref{driver_app_robot_tab_table_general_settings}.}\\
          PRIOR: MODERATE\\}

      \end{itemize}

      %%%Таблица общих настроек робота.
      \begin{table}[h]
        \begin{center}
        \label{driver_app_robot_tab_table_general_settings}
        \caption {Общие настройки робота}
        \setlength{\extrarowheight}{2mm}
        \begin{tabular}{|p{4cm}|p{3cm}|p{8cm}|}

          \hline     \textbf{Название}&\textbf{Способ изменения настройки}&\textbf{Описание} \\ [2mm]

          \hline   Заказы классом ниже & Переключатель ON/OFF & Водитель с машиной определенного класса может принимать заказы ниже по классу.\\ [2mm]

          \hline   Заказы от Яндекса & Переключатель ON/OFF & Возможность принимать заказы через канал Яндекс.Такси. \\ [2mm]
            
          \hline   Цепочка & Переключатель ON/OFF & Возможность закрепления за водителем заказов “в цепочку”. \\ [2mm]

          \hline   Дистанция поиска & Поле для ввода & Возможность задания водителем километража, в пределах которого сервер подбирает заказ. \\ [2mm]

          \hline   Время подачи & Поле для ввода & Возможность задания водителем максимального времени, за которое он хочет добраться до начальной точки заказа. \\ [2mm]

          \hline
        \end{tabular}
        \end{center}
      \end{table}

    \subsubsection{Роботы} \label{driver_app_robot_tab_robot_settings} \mbox{}\\

      \begin{itemize}

        \item{
          TITLE: Включение и выключение робота.\\
          \sr{Робот автоматически включается после установки общих настроек и выбора режимов работы. Если никакие настройки не устновлены и не выбран ни один из режимов работы - робот выключается.}\\
          PRIOR: MODERATE\\
          }

        \item{
          TITLE: Режимы работы роботов. \\
          \sr{В списке должны присутствовать режимы, перечисленные в таблице \ref{driver_app_robot_tab_modes}.}\\
          PRIOR: MODERATE\\}
      \end{itemize}

      %%%Таблица режимов работы робота.
      \begin{table}
          \begin{center}
          \label{driver_app_robot_tab_modes}
          \caption {Режимы работы робота}
          \setlength{\extrarowheight}{2mm}
          \begin{tabular}{|p{3cm}|p{3cm}|p{6cm}|p{3cm}|}

            \hline     \textbf{Название режима работы}&\textbf{Способ выбора режима работы}&\textbf{Список настроек}&\textbf{Описание настроек} \\ [2mm]

            \hline   Городской & Переключатель ON/OFF & Можно изменить следующие настройки: \begin{itemize} \item Округ. \end{itemize} & См. в таблице \ref{driver_app_robot_tab_table_town_mode} \\ [2mm]

            \hline   Портовый & Переключатель ON/OFF & Можно изменить следующие настройки: \begin{itemize} \item Порты.;  \item Встреча;  \item Проводы. \end{itemize} & См. в таблице \ref{driver_app_robot_tab_table_port_mode}  \\ [2mm]

            \hline
          \end{tabular}
          \end{center}
      \end{table}

      %%%Таблица описания настроек в городском режиме работы.
      \begin{table}
          \begin{center}
          \label{driver_app_robot_tab_table_town_mode}
          \caption {Описание настроек в городском режиме работы}
          \setlength{\extrarowheight}{2mm}
          \begin{tabular}{|p{4cm}|p{3cm}|p{8cm}|}

            \hline     \textbf{Название}&\textbf{Способ изменения}&\textbf{Описание} \\ [2mm]

            \hline   Округ & Выпадающий список & Водитель помечает чекбоксом округи в списке, таким образом исключая заказы с конечной точкой в выбранных округах. Доступен выбор из следующих округов: \begin{itemize} \item ЦАО \item САО \item ЮАО \item СЗАО \item ЮЗАО \item ВАО \item СВАО \item ЮВАО \item ЗАО \end{itemize} \\ [2mm]

            \hline
          \end{tabular}
          \end{center}
      \end{table}        

      %%%Таблица описания настроек в портовом режиме работы.
      \begin{table}
          \begin{center}
          \label{driver_app_robot_tab_table_port_mode}
          \caption {Описание настроек в портовом режиме работы}
          \setlength{\extrarowheight}{2mm}
          \begin{tabular}{|p{4cm}|p{3cm}|p{8cm}|}

            \hline     \textbf{Название}&\textbf{Способ изменения}&\textbf{Описание} \\ [2mm]

            \hline   Порты & Выпадающий список & Водитель помечает чекбоксом названия аэропортов. Доступен выбор из следующих аэропортов: \begin{itemize} \item Домодедово \item Шереметьево \item Внуково \end{itemize} \\ [2mm]

            \hline   Встреча & Переключатель ON/OFF & Опция, указывающая на то, что местом назначения для заказа этого типа будет один или несколько аэропортов, выбранных водителем. \\ [2mm]

            \hline   Проводы & Переключатель ON/OFF & Опция, указывающая на то, что местом прибытия для заказа этого типа является один или несколько аэропортов, выбранных водителем. \\ [2mm]

            \hline
          \end{tabular}
          \end{center}
      \end{table}

  \subsection{Баланс}

    DESC: Во вкладке баланс водитель может посмотреть актуальное состояние фискального и арендного баланса, а так же пополнить их. 

      TITLE: Элементы вкладки\\
      \sr{Элементы этого экрана описаны в таблице \ref{driver_app_balance_tab_elements}.}\\
      PRIOR: MODERATE\\

      TITLE: Расположение элементов\\
      \sr{Экран делится на две части, при помощи вертикальной линии, в соотношении 6,5 : 3,5.}\\
      \sr{В левой части расположен элемент ELTAX-\ref{driver_element_transaction_area}}\\
      \sr{В правой части расположены элементы ELTAX-\ref{driver_element_fiscal_balance}, ELTAX-\ref{driver_element_rent_balance}, ELTAX-\ref{driver_element_ui_update_balance} в том же порядке в котором они преречислены.}\\
      PRIOR: MODERATE\\

      \begin{table}
        \begin{center}
        \caption{Элементы вкладки "Счет"}
        \label{driver_app_balance_tab_elements}
        \setlength{\extrarowheight}{2mm}
        \begin{tabular}{|p{3cm}|p{4cm}|p{8cm}|}
           \hline   \textbf{ID}&  \textbf{Название}&\textbf{Требования/Описание} \\ [2mm]


           \hline \eltax{driver_element_fiscal_balance}{} & Фискальный баланс & \sr{Элемент состоит из поля - [Фискальный баланс в рублях]}\\ [2mm]

           \hline \eltax{driver_element_rent_balance}{} & Арендный баланс & \sr{Элемент состоит из двух полей: [Баланс  в рублях], [Баланс конвертированный в дни]}\\ [2mm]

           \hline \eltax{driver_element_ui_update_balance}{} & Интерфейс выставления счета  & --- НАХОДИТСЯ В СТАДИИ РАЗРАБОТКИ ---\\ [2mm]

           \hline \eltax{driver_element_transaction_area}{} & Область транзакций  & --- НАХОДИТСЯ В СТАДИИ РАЗРАБОТКИ ---\\ [2mm]    

           \hline
        \end{tabular}
        \end{center}
      \end{table}   

  \subsection{Настройки}

    \subsubsection{Уведомления}

      \paragraph{Общие требования} \mbox{}\\

      \subparagraph{Функциональные требования} \mbox{}\\      
        \begin{itemize}

          \item{TITLE: Новое уведомление. \\
                \sr{Новое уведомление помечается как непрочитанное, если пользователь не находится во вкладке "Уведомления" DriverApp. Инкрементируется счетчик непрочитанных уведомлений. Если на момент поступления нового уведомления пользователь находится во вкладке "Уведомления" DriverApp, новое уведомление помечается как прочитанное и счетчик непрочитанных уведомлений не инкрементируется.} \\
                PRIOR: MODERATE \\}

          \item{TITLE: Прочитанное уведомление. \\
                \sr{Уведомления считаются прочитанными, если была открыта вкладка "Уведомления" в DriverApp. Счетчик непрочитанных уведомлений при этом обнуляется.} \\
                PRIOR: MODERATE \\}

          \item{TITLE: Счетчик непрочитанных уведомлений. \\
                \sr{Означает количество непрочитанных уведомлений на данный момент. Инкрементируется, если появилось новое непрочитанное уведомление. Обнуляется при просмотре непрочитанных уведомлений через вкладку "Уведомления" в DriverApp.} \\
                PRIOR: MODERATE \\}
          
          \item{TITLE: Иконка. \\
                \sr{Выбирается в зависимости от инициатора уведомления из "Таблицы инициаторов уведомления".} \\
                PRIOR: MODERATE \\}

        \end{itemize}

      \paragraph{Уведомления в панели уведомлений Android(ПУА)} \mbox{}\\

        \subparagraph{Функциональные требования} \mbox{}\\

          \begin{itemize}

            \item{TITLE: Появление нового уведомления. \\
                  \sr{При появлении нового уведомления от какого-либо инициатора уведомления, оно выводится в панели уведомлений Android(ПУА). Если в данный момент появляется сразу несколько новых уведомлений, счетчик непрочитанных уведомлений инкрементируется на их количество.} \\
                  PRIOR: MODERATE \\}

            \item{TITLE: Открытие уведомления. \\
                  \sr{По нажатию на уведомление в ПУА, открывается DriverApp на вкладке "Уведомления". При этом, счетчик непрочитанных уведомлений обнуляется.} \\
                  PRIOR: MODERATE \\}

            \item{TITLE: Разрешенные действия с уведомлениями. \\
                  \sr{Разрешается нажать на уведомление в ПУА. В этом случае откроется DriverApp на вкладке "Уведомления", счетчик непрочитанных уведомлений обнулится.} \\ 
                  \sr{Разрешается убрать уведомление из ПУА путем скрола уведомления в сторону. В этом случае не произойдет никаких действий, счетчик непрочитанных уведомлений не обнулится.} \\
                  PRIOR: MODERATE \\}

            \item{TITLE: Интерфейс уведомления. \\
                  \sr{В панели уведомлений Android уведомления отображаются согласно \href{https://www.evernote.com/shard/s389/sh/8d4e738c-61f0-4500-98bb-5882d89c9906/9ab0dc75caa4a5144ac1e5042e808b4f}{прототипу}.} \\
                  PRIOR: MODERATE \\}

            \item{TITLE: Строка "Новых уведомлений"(верхняя строка). \\
                  \sr{Отображется: [Новых уведомлений: ] + [общее количество непрочитанных на данный момент уведомлений].} \\
                  PRIOR: MODERATE \\}

            \item{TITLE: Строка "Последних уведомлений"(нижняя строка). \\                          
                  \sr{Отображется: [последнее непрочитанное уведомление] + [,] + [предпоследнее непрочитанное уведомление] + [...] в конце, если непрочитанных сообщений больше двух.} \\
                  PRIOR: MODERATE \\}

            \item{TITLE: Дата. \\
                  \sr{Отображается время появления уведомления в формате "Часы:Минуты".} \\
                  PRIOR: MODERATE \\}

          \end{itemize}

      \paragraph{Уведомления в DriverApp во вкладке Уведомления} \mbox{}\\

        \subparagraph{Функциональные требования} \mbox{}\\

          \begin{itemize}

            \item{TITLE: Появление нового уведомления. \\
                  \sr{Непрочитанные уведомления появляются сверху списка во вкладке "Уведомления".} \\
                  PRIOR: MODERATE \\}

            \item{TITLE: Открытие вкладки "Уведомления". \\
                  \sr{При открытии вкладки "Уведомления", обнуляется счетчик непрочитанных уведомлений.} \\
                  PRIOR: MODERATE \\}

            \item{TITLE: Разрешенные действия с уведомлениями. \\
                  \sr{Уведомления во вкладке "Уведомления" разрешается только просмотреть.} \\
                  PRIOR: MODERATE \\}

            \item{TITLE: Интерфейс уведомления. \\
                  \sr{Во вкладке "Уведомления" уведомления отображаются согласно \href{https://www.evernote.com/shard/s389/sh/f7d9319d-bb3a-47de-b857-5a37e440419c/774bb30105ccb402d93522ae9e333b5e}{прототипу}.} \\
                  PRIOR: MODERATE \\}

            \item{TITLE: Строка "Уведомления"(верхняя строка). \\
                  \sr{Отображается содержание уведомления.} \\
                  PRIOR: MODERATE \\}

            \item{TITLE: Строка "Дата и время"(нижняя строка). \\
                  \sr{Отображается дата и время появления уведомления в формате "ДД.ММ.ГГГГ Часы:Минуты".} \\
                  PRIOR: MODERATE \\}

          \end{itemize}

      % Таблица уведомлений
      \label{taxometr_notifications_table}
      \setlength{\extrarowheight}{2mm}
          \begin{longtable}{|p{3cm}|p{4cm}|p{5cm}|p{3cm}|}
              \caption {Таблица уведомлений}\\

              \hline     \textbf{ID}&\textbf{Название}&\textbf{Содержимое уведомления} & \textbf{Инициатор}\\ [2mm]
              \endfirsthead
              \hline     \textbf{ID}&\textbf{Название}&\textbf{Содержимое уведомления} & \textbf{Инициатор}\\ [2mm]
              \endhead
              
              \hline  \nttax{notif_driver_of_remove_driver_from_the_order}{} & Снятие водителя с заказа. & Вы сняты с заказа + [№Заказа]. & SRVACT-\ref{act_remove_driver_driver_notification} \\ [2mm]

              \hline \nttax{notif_of_new_order_in_reserve}{} & Добавление нового заказа в резерв. & Заказ + [№Заказа] + поступил в резерв. & \\ [2mm]       %TODO: Дописать SRVACT-\ref{act_add_new_order_in_reserve}

              \hline
          \end{longtable}

    3.2.9 Настройки мобильного клиента
    3.2.8.1 Основные настройки
    3.2.8.1.1 Звуковые уведомления
    3.2.8.2 История заказов
    3.2.1.1 Кнопка выхода