\section{Водительское приложение}

  \subsection{Общие требования к мобильному приложению}

    \begin{itemize}

      \item{
        TITLE: Отправка GPS данных на сервер\\
        \sr{Мобильное приложение оповещает сервер о местоположении водителя отсылая GPS координаты с периодичностью в STAT-\ref{periodicity_transfer_coordinates_taximeter_to_server} секунд.}\\
        PRIOR: MODERATE\\}

      \item{
        TITLE: Ориентация приложения на экране устройства\\
        \sr{Приложение имеет альбомную ориентацию.}\\
        PRIOR: MODERATE\\}

    \end{itemize}

  \subsection{Интерфейс приложения}

    DESC: Описание интерфейса водительского мобильного приложения.
       
    TITLE: Элементы интерфейса\\
    \sr{В интерфейсе приложения должны присутствовать элементы, которые перечислены и описаны в таблице \ref{app_interface}.}\\
    PRIOR: CRITICAL\\  

    TITLE: Расположение элементов\\
    \sr{Вверху экрана находится элемент ELTAX-\ref{app_interface_lenta_vkladok}.}\\
    \sr{Под элементом ELTAX-\ref{app_interface_lenta_vkladok} находится элемент ELTAX-\ref{app_interface_osn_obl}.}\\
    PRIOR: CRITICAL\\
    
    \begin{table}[h]
    \begin{center}
    \caption {Интерфейс приложения}
    \label{app_interface}
    \setlength{\extrarowheight}{2mm}
      \begin{tabular}{|p{3cm}|p{3cm}|p{9cm}|}
        \hline     \textbf{ID}&\textbf{Название элемента}&\textbf{Описание элемента} \\ [2mm]

        \hline   \eltax{app_interface_lenta_vkladok}{} & Лента вкладок & \sr{Содержит следующие вкладки: \begin{itemize} \item Таксометр. (Описание - \ref{driver_app_taximeter_tab}) \item Заказы. (Описание - \ref{driver_app_orders_tab}) \item Навигатор. (Описание - \ref{driver_app_navigator_tab}) \item Счёт. (Описание - \ref{driver_app_bill_tab}) \item Робот. (Описание - \ref{driver_app_robot_tab}) \item Настройки. (Описание - \ref{driver_app_settings_tab}) \end{itemize}}\\ [2mm]

        \hline   \eltax{app_interface_osn_obl}{} & Основная область & \sr{Находится под 'Лентой вкладок' и занимает оставшуюся часть экрана. В этой области отображается интерфейс выбранной вкладки.}\\ [2mm]

        \hline
      \end{tabular}
    \end{center}
    \end{table}

  \subsection{Пользователи}

      \subsubsection{Роли, состояния приложения}

      		DESC: У приложения есть два состояния, они перечислены и описаны в таблице \ref{app_state}. 

             \begin{table}
             \begin{center}
             \caption {Состояния приложений}
             \label{app_state}
             \setlength{\extrarowheight}{2mm}
             \begin{tabular}{|p{5cm}|p{10cm}|}
                 \hline     \textbf{Состояние}&\textbf{Описание, уровень доступа} \\ [2mm]

                 \hline   Авторизованный пользователь & В этом состоянии пользователю доступны все элементы интерфейса и функциональные возможности приложения.\\ [2mm]
                 \hline   Автономный режим & В этом состоянии пользователю доступны следующие элементы интерфейса: \begin{itemize} \item Таксометр \item Навигатор \item Настройки \end{itemize}\\ [2mm]
                  \hline
             \end{tabular}
             \end{center}
             \end{table}
		    
      \subsubsection{Авторизация} \label{options_tab_authorization}

        DESC: Водитель имеет возможность авторизоваться в специальном окне, которое появляется после выбора пункта “Войти” во вкладке “Настройки”.

        \begin{itemize}
          
          \item{
            TITLE: Элементы окна для авторизации\\
            \sr{В окне должны присутствовать элементы, которые перечислены и описаны в таблице \ref{driver_app_authorization_tab_elements}.}\\
            PRIOR: MODERATE\\}

          \item{
            TITLE: Расположение элементов\\
            \sr{Окно условно делится на две части, расположенных одна под другой.}\\
            \sr{В верхней части друг под другом расположены элементы ELTAX-\ref{driver_element_auth_company}, ELTAX-\ref{driver_element_auth_login}, ELTAX-\ref{driver_element_auth_pwd}, ELTAX-\ref{driver_element_auth_remember} в том же порядке, в котором они перечислены.}\\
            \sr{В нижней части друг под другом расположены элементы ELTAX-\ref{driver_element_auth_enter}, ELTAX-\ref{driver_element_auth_autonom}, в том же порядке, в котором они перечислены.}\\
            PRIOR: MODERATE\\}

        \end{itemize}        

        \begin{table}[h]
          \begin{center}
          \caption{Элементы окна авторизации}
          \label{driver_app_authorization_tab_elements}
          \setlength{\extrarowheight}{2mm}
          \begin{tabular}{|p{3cm}|p{3cm}|p{9cm}|}
             \hline   \textbf{ID}&  \textbf{Название}&\textbf{Требования/Описание} \\ [2mm]


             \hline \eltax{driver_element_auth_company}{} & Компания & \sr{Выпадающий список. Водитель выбирает название своей компании из списка, элементы которого выдаёт сервер.}\\ [2mm]

             \hline \eltax{driver_element_auth_login}{} & Логин & \sr{Текстовое поле для ввода. Водитель вводит свой позывной.}\\ [2mm]

             \hline \eltax{driver_element_auth_pwd}{} & Пароль & \sr{Текстовое поле для ввода. Водитель вводить пароль для своего логина.}\\ [2mm]

             \hline \eltax{driver_element_auth_remember}{} & Запомнить & \sr{Chechbox. При наличии флажка в checkbox-е, введенные водителем логин и пароль сохраняются при следующем входе в приложение. При отсутствии флажка, введенные данные не сохраняются.}\\ [2mm]    

             \hline \eltax{driver_element_auth_enter}{} & Кнопка “Войти” & \sr{При нажатии на кнопку “Войти” на сервер отправляются: \begin{itemize} \item Введенные водителем в поля “Логин” и “Пароль” данные; \item Состояние флажка “Запомнить”; \item Выбранный элемент из списка “Компания”. \end{itemize} На сервере происходит проверка введенных данных, и если они валидны, происходит авторизация водителя. Если данные не проходят валидацию, приложение возвращает водителя обратно на окно авторизации, и водителю вновь необходимо ввести данные.}\\ [2mm]

             \hline \eltax{driver_element_auth_autonom}{} & Кнопка “Работать автономно” & \sr{При нажатии на кнопку “Работать автономно”, на сервер не отправляется никаких данных и авторизация не происходит. Приложение возвращает водителя на вкладку “Таксометр”.}\\ [2mm]

             \hline
          \end{tabular}
          \end{center}
        \end{table}

  \subsection{Таксометр} \label{driver_app_taximeter_tab}

    DESC: При переходе на вкладку “Таксометр” отображается экран таксометра.

    \subsubsection{Работа с заказом “От бордюра”}

      \paragraph{Вкладка при первом открытии} \label{driver_app_taximeter_tab_first_opening}
        \begin{itemize}

          \item{
            TITLE: Элементы вкладки при первом открытии\\
            \sr{Во вкладке при первом открытии должны присутствовать элементы, которые перечислены и описаны в таблице \ref{driver_app_taximeter_tab_first_opening_elements}.}\\
            PRIOR: MODERATE\\}

          \item{
            TITLE: Расположение элементов\\
            \sr{Окно условно делится на три части, расположенных одна под другой.}\\
            \sr{В верхней части в два ряда расположены элементы ELTAX-\ref{driver_element_driver_balance}, ELTAX-\ref{driver_element_driver_pozyvnoy}, ELTAX-\ref{driver_element_driver_auth_state}, ELTAX-\ref{driver_element_driver_curr_coords}, ELTAX-\ref{driver_element_this_order_time}, ELTAX-\ref{driver_element_this_order_dist}.}\\
            \sr{В средней части расположен элемент ELTAX-\ref{driver_element_driver_state}.}\\
            \sr{В нижней части расположен элемент ELTAX-\ref{driver_element_start_button}.}\\
            PRIOR: MODERATE\\}

        \end{itemize}

        \setlength{\extrarowheight}{2mm}
          \begin{longtable}{|p{3cm}|p{3cm}|p{9cm}|}
              
          \caption {Элементы вкладки при первом открытии} \label{driver_app_taximeter_tab_first_opening_elements} \\

            \hline  \textbf{ID}  & \textbf{Название} & \textbf{Требования/Описание} \\ [2mm]
            \endfirsthead
            \hline  \textbf{ID}  & \textbf{Название} & \textbf{Требования/Описание} \\ [2mm]
            \endhead

            \hline \eltax{driver_element_driver_balance}{} & Баланс водителя & \sr{Отображается в виде: \begin{itemize} \item Водитель авторизован: [Баланс: ] + арендный счёт водителя. \item Водитель не авторизован: [Баланс: ] + [-]. \end{itemize} Арендный счет водителя берется с сервера.}\\ [2mm]

            \hline \eltax{driver_element_driver_pozyvnoy}{} & Позывной водителя & \sr{Отображается в виде: \begin{itemize} \item Водитель авторизован: [Позывной: ] + позывной водителя. \item Водитель не авторизован: Не отображается. \end{itemize} Позывным водителя является логин, введенный им при авторизации.}\\ [2mm]

            \hline \eltax{driver_element_driver_auth_state}{} & Состояние авторизации водителя & \sr{Отображается в виде: \begin{itemize} \item Водитель авторизован: Индикатор зеленого цвета + [В сети]. \item Водитель не авторизован: Индикатор красного цвета + [Не в сети]. \item Если уже авторизованный водитель по каким-либо причинам пропадает из сети, то происходит попытка переподключения: Индикатор желтого цвета + [Переподключение] \end{itemize} Водитель считается авторизованным, если он ввел корректные логин и пароль при авторизации и с сервера пришел положительный ответ на запрос об авторизации с этими данными}\\ [2mm]

            \hline \eltax{driver_element_driver_curr_coords}{} & Текущее местоположение водителя & \sr{Отображается в виде: \begin{itemize} \item GPS включен: Индикатор зеленого цвета + [GPS работает]. \item GPS выключен: Индикатор красного цвета + [Нет координат]. \end{itemize}}\\ [2mm]

            \hline \eltax{driver_element_this_order_time}{} & Время, затраченное водителем на текущий заказ & \sr{Отображается счетчик времени, затраченного на текущий заказ, в формате [ЧЧ:ММ:СС].}\\ [2mm]

            \hline \eltax{driver_element_this_order_dist}{} & Километраж текущего заказа & \sr{Отображается счетчик километража текущего заказа в формате [км,м].}\\ [2mm]

            \hline \eltax{driver_element_driver_state}{} & Состояние водителя & \sr{Отображается состояние водителя. Есть два состояния: \begin{itemize} \item “Свободен” - отображается, когда у водителя нету активных заказов. \item “Занят” - отображается, когда у водителя есть активный заказ. \end{itemize}}\\ [2mm]

            \hline \eltax{driver_element_start_button}{} & Кнопка “Старт” & \sr{По нажатию кнопки “Старт” интерфейс вкладки становится таким, как описано в разделе \ref{driver_app_taximeter_tab_after_start_button}.}\\ [2mm]

            \hline

          \end{longtable}

      \paragraph{Вкладка после нажатия кнопки “Старт”}  \label{driver_app_taximeter_tab_after_start_button}
        \begin{itemize}

          \item{
            TITLE: Элементы вкладки после нажатия кнопки “Старт”\\
            \sr{Во вкладке после нажатия кнопки “Старт” в добавок либо в замен старым должны появляться новые элементы, которые перечислены и описаны в таблице \ref{driver_app_taximeter_tab_after_start_button_elements}.}\\
            PRIOR: MODERATE\\}

          \item{
            TITLE: Расположение элементов\\
            \sr{Окно условно делится на три части, расположенных одна под другой.}\\
            \sr{В верхней части в два ряда расположены элементы ELTAX-\ref{driver_element_driver_balance}, ELTAX-\ref{driver_element_driver_pozyvnoy}, ELTAX-\ref{driver_element_driver_auth_state}, ELTAX-\ref{driver_element_driver_curr_coords}, ELTAX-\ref{driver_element_this_order_time_after_start_button}, ELTAX-\ref{driver_element_this_order_dist_after_start_button}.}\\
            \sr{В средней части расположен элемент ELTAX-\ref{driver_element_order_costs}.}\\
            \sr{В нижней части расположен элемент ELTAX-\ref{driver_element_end_button}.}\\
            \sr{Слева расположена скрытая боковая панель ELTAX-\ref{driver_element_order_info}.}\\
            PRIOR: MODERATE\\}

        \end{itemize}

        \setlength{\extrarowheight}{2mm}
          \begin{longtable}{|p{3cm}|p{3cm}|p{9cm}|}
              
          \caption {Новые элементы вкладки после нажатия кнопки “Старт”} \label{driver_app_taximeter_tab_after_start_button_elements} \\

            \hline  \textbf{ID}  & \textbf{Название} & \textbf{Требования/Описание} \\ [2mm]
            \endfirsthead
            \hline  \textbf{ID}  & \textbf{Название} & \textbf{Требования/Описание} \\ [2mm]
            \endhead

            \hline \eltax{driver_element_this_order_time_after_start_button}{} & Время, затраченное водителем на текущий заказ & \sr{Отображается счетчик времени, затраченного на текущий заказ, в формате [ЧЧ:ММ:СС]. Динамически увеличивается ежесекундно после нажатия кнопки “Старт”(ELTAX-\ref{driver_element_start_button}).}\\ [2mm]

            \hline \eltax{driver_element_this_order_dist_after_start_button}{} & Километраж текущего заказа & \sr{Отображается счетчик километража текущего заказа в формате [км,м]. Динамически увеличивается на соответствующее проеханное водителем расстояние после нажатия кнопки “Старт”.}\\ [2mm]

            \hline \eltax{driver_element_order_costs}{} & Стоимость поездки & \sr{Отображается счетчик стоимости поездки в формате [Значение счетчика] + [руб.]. Значения счетчика изменяются по формуле: ([Стоимость одной минуты в рублях] * [Время, затраченное водителем на текущий заказ]). Значение счетчика нельзя изменить вручную. }\\ [2mm]

            \hline \eltax{driver_element_order_info}{} & Информация о заказе & \sr{Выдвигаемая панель. В этой панели находятся следующие элементы: \begin{itemize} \item Заголовок - [“Информация о текущем заказе”] \item “Время” - на какое время назначен заказ. \item “Откуда” - адрес, откуда начинается выполнение заказа. \item “Куда” - адрес конечного места назначения заказа. \item “Примечания” - указываются какие-либо требования к обслуживанию. \end{itemize}}\\ [2mm]

            \hline \eltax{driver_element_end_button}{} & Кнопка “Завершить” & \sr{По нажатию кнопки “Завершить”: 
            \begin{itemize} 
              \item Интерфейс вкладки становится таким, как описано в разделе \ref{driver_app_taximeter_tab_after_end_button}. 
              \item Прекращается изменение счетчика ELTAX-\ref{driver_element_this_order_time_after_start_button}, его значение сохраняется. 
              \item Прекращается изменение счетчика ELTAX-\ref{driver_element_this_order_dist_after_start_button}, его значение сохраняется. 
              \item Прекращается изменение счетчика ELTAX-\ref{driver_element_order_costs}, его значение сохраняется. 
              \item Все вкладки приложения, кроме “Таксометр”, становятся неактивными. 
            \end{itemize}}\\ [2mm]

            \hline

          \end{longtable}      
            
      \paragraph{Вкладка после нажатия кнопки “Завершение”} \label{driver_app_taximeter_tab_after_end_button} 
        \begin{itemize}

          \item{
            TITLE: Элементы вкладки после нажатия кнопки “Завершение”\\
            \sr{Во вкладке после нажатия кнопки “Завершение” в добавок либо в замен старым должны появляться новые элементы, которые перечислены и описаны в таблице \ref{driver_app_taximeter_tab_after_end_button_elements}.}\\
            PRIOR: MODERATE\\}

          \item{
            TITLE: Расположение элементов\\
            \sr{Окно условно делится на три части, расположенных одна под другой.}\\
            \sr{В верхней части в два ряда расположены элементы ELTAX-\ref{driver_element_driver_balance}, ELTAX-\ref{driver_element_driver_pozyvnoy}, ELTAX-\ref{driver_element_driver_auth_state}, ELTAX-\ref{driver_element_driver_curr_coords}, ELTAX-\ref{driver_element_this_order_time_after_end_button}, ELTAX-\ref{driver_element_this_order_dist_after_end_button}.}\\
            \sr{В средней части расположен элемент ELTAX-\ref{driver_element_order_costs_after_end_button}.}\\
            \sr{В нижней части расположен элементы ELTAX-\ref{driver_element_checkout_button}, ELTAX-\ref{driver_element_change_button}.}\\
            \sr{Слева расположена скрытая боковая панель ELTAX-\ref{driver_element_order_info}.}\\
            PRIOR: MODERATE\\}
        \end{itemize}

        \setlength{\extrarowheight}{2mm}
          \begin{longtable}{|p{3cm}|p{3cm}|p{9cm}|}
              
          \caption {Новые элементы вкладки после нажатия кнопки “Завершение”} \label{driver_app_taximeter_tab_after_end_button_elements} \\

            \hline  \textbf{ID}  & \textbf{Название} & \textbf{Требования/Описание} \\ [2mm]
            \endfirsthead
            \hline  \textbf{ID}  & \textbf{Название} & \textbf{Требования/Описание} \\ [2mm]
            \endhead

            \hline \eltax{driver_element_this_order_time_after_end_button}{} & Время, затраченное водителем на текущий заказ & \sr{Отображается сохраненное после нажатия кнопки “Завершение”(ELTAX-\ref{driver_element_end_button}) значение счетчика ELTAX-\ref{driver_element_this_order_time_after_start_button} .}\\ [2mm]

            \hline \eltax{driver_element_this_order_dist_after_end_button}{} & Километраж текущего заказа & \sr{Отображается сохраненное после нажатия кнопки “Завершение”(ELTAX-\ref{driver_element_end_button}) значение счетчика ELTAX-\ref{driver_element_this_order_dist_after_start_button} .}\\ [2mm]

            \hline \eltax{driver_element_order_costs_after_end_button}{} & Стоимость поездки & \sr{Отображается сохраненное после нажатия кнопки “Завершение”(ELTAX-\ref{driver_element_end_button} значение счетчика ELTAX-\ref{driver_element_order_costs} ).}\\ [2mm] 

            \hline \eltax{driver_element_checkout_button}{} & Кнопка “Оплата” & \sr{По нажатию кнопки “Оплата”: \begin{itemize} \item Интерфейс изменяется на такой, как описано в разделе \ref{driver_app_taximeter_tab_first_opening}. \item Мобильное приложение делает запрос на сервер о завершении заказа, в котором передает итоговую стоимость поездки, коей является сохраненное значение счетчика ELTAX-\ref{driver_element_order_costs} после нажатия кнопки “Завершение”(ELTAX-\ref{driver_element_end_button}) или после изменения значения этого счетчика с помощью кнопки “Изменить”(ELTAX-\ref{driver_element_change_button}). В ответе сервер присваивает водителю статус “Свободен”. \item Все вкладки приложения вновь становятся активными. \end{itemize}}\\ [2mm]

            \hline  \eltax{driver_element_change_button}{} & Кнопка “Изменить” & По нажатию кнопки “Изменить” становится возможным изменение счетчика ELTAX-\ref{driver_element_order_costs} вручную, но только в большую сторону. \\ [2mm]

            \hline

          \end{longtable}

    \subsubsection{Работа с заказом от диспетчера/робота} \label{driver_app_taximeter_tab_from_disp_or_robot}

      DESC: При приеме заказа от диспетчера или робота происходит переход на вкладку “Таксометр”.

      \paragraph{Вкладка “Таксометр” при приеме заказа от диспетчера} \label{driver_app_taximeter_tab_from_orders_tab}
        \begin{itemize}

          \item{
            TITLE: Элементы вкладки \\
            \sr{Во вкладке должны присутствовать элементы, которые перечислены и описаны в таблице \ref{driver_app_taximeter_tab_from_orders_tab_elements}.}\\
            PRIOR: MODERATE\\}

          \item{
            TITLE: Расположение элементов\\
            \sr{Окно условно делится на три части, расположенных одна под другой.}\\
            \sr{В верхней части в два ряда расположены элементы ELTAX-\ref{driver_element_driver_balance_disp}, ELTAX-\ref{driver_element_driver_pozyvnoy_disp}, ELTAX-\ref{driver_element_driver_auth_state_disp}, ELTAX-\ref{driver_element_driver_curr_coords_disp}, ELTAX-\ref{driver_element_this_order_time_disp}, ELTAX-\ref{driver_element_this_order_dist_disp}.}\\
            \sr{В средней части расположен элемент ELTAX-\ref{driver_element_driver_state_disp}.}\\
            \sr{В нижней части расположен элемент ELTAX-\ref{driver_element_in_place_button_disp}, ELTAX-\ref{driver_element_delay_button_disp}.}\\
            \sr{Слева расположена скрытая боковая панель ELTAX-\ref{driver_element_order_info_disp}.}\\
            PRIOR: MODERATE\\}

        \end{itemize}

        \setlength{\extrarowheight}{2mm}
          \begin{longtable}{|p{3cm}|p{3cm}|p{9cm}|}
              
          \caption {Элементы вкладки “Таксометр” при приеме заказа от диспетчера} \label{driver_app_taximeter_tab_from_orders_tab_elements} \\

            \hline  \textbf{ID}  & \textbf{Название} & \textbf{Требования/Описание} \\ [2mm]
            \endfirsthead
            \hline  \textbf{ID}  & \textbf{Название} & \textbf{Требования/Описание} \\ [2mm]
            \endhead

            \hline \eltax{driver_element_driver_balance_disp}{} & Баланс водителя & \sr{Отображается в виде: \begin{itemize} \item Водитель авторизован: [Баланс: ] + арендный счёт водителя. \item Водитель не авторизован: [Баланс: ] + [-]. \end{itemize} Арендный счет водителя берется с сервера.}\\ [2mm]

            \hline \eltax{driver_element_driver_pozyvnoy_disp}{} & Позывной водителя & \sr{Отображается в виде: \begin{itemize} \item Водитель авторизован: [Позывной: ] + позывной водителя. \item Водитель не авторизован: Не отображается. \end{itemize} Позывным водителя является логин, введенный им при авторизации.}\\ [2mm]

            \hline \eltax{driver_element_driver_auth_state_disp}{} & Состояние авторизации водителя & \sr{Отображается в виде: \begin{itemize} \item Водитель авторизован: Индикатор зеленого цвета + [В сети]. \item Водитель не авторизован: Индикатор красного цвета + [Не в сети]. \item Если уже авторизованный водитель по каким-либо причинам пропадает из сети, то происходит попытка переподключения: Индикатор желтого цвета + [Переподключение] \end{itemize} Водитель считается авторизованным, если он ввел корректные логин и пароль при авторизации и с сервера пришел положительный ответ на запрос об авторизации с этими данными}\\ [2mm]

            \hline \eltax{driver_element_driver_curr_coords_disp}{} & Текущее местоположение водителя & \sr{Отображается в виде: \begin{itemize} \item GPS включен: Индикатор зеленого цвета + [GPS работает]. \item GPS выключен: Индикатор красного цвета + [Нет координат]. \end{itemize}}\\ [2mm]

            \hline \eltax{driver_element_this_order_time_disp}{} & Время, затраченное водителем на текущий заказ & \sr{Отображается счетчик времени, затраченного на текущий заказ, в формате [ЧЧ:ММ:СС].}\\ [2mm]

            \hline \eltax{driver_element_this_order_dist_disp}{} & Километраж текущего заказа & \sr{Отображается счетчик километража текущего заказа в формате [км,м].}\\ [2mm]

            \hline \eltax{driver_element_driver_state_disp}{} & Состояние водителя & \sr{Отображается стутус [Еду к клиенту]. Отображается, пока не нажата кнопка “На месте”(ELTAX-\ref{driver_element_in_place_button_disp}).}\\ [2mm]

            \hline \eltax{driver_element_in_place_button_disp}{} & Кнопка “На месте” & \sr{Нажатие на кнопку означает то, что водитель прибыл к месту, откуда поступил заказ.}\\ [2mm]

            \hline \eltax{driver_element_delay_button_disp}{} & Кнопка “Опаздываю” & \sr{Нажатие на кнопку означает то, что водитель не успевает прибыть к месту, откуда поступил заказ к нужному времени. Достепен выбор времени, на сколько опаздывает водитель: \begin{itemize} \item 5 минут \item 10 минут \item 15 минут \end{itemize}}\\ [2mm]

            \hline \eltax{driver_element_order_info_disp}{} & Информация о заказе & \sr{Выдвигаемая панель. В этой панели находятся следующие элементы: \begin{itemize} \item Заголовок - [“Информация о текущем заказе”] \item “Время” - на какое время назначен заказ. \item “Откуда” - адрес, откуда начинается выполнение заказа. \item “Куда” - адрес конечного места назначения заказа. \item “Примечания” - указываются какие-либо требования к обслуживанию. \end{itemize}}\\ [2mm]

            \hline

          \end{longtable}

      \paragraph{Вкладка после нажатия кнопки “На месте”}  \label{driver_app_taximeter_tab_after_in_place_button}
        \begin{itemize}

          \item{
            TITLE: Элементы вкладки после нажатия кнопки “На месте”\\
            \sr{Во вкладке после нажатия кнопки “На месте” в добавок либо в замен старым должны появляться новые элементы, которые перечислены и описаны в таблице \ref{driver_app_taximeter_tab_after_start_button_elements}.}\\
            PRIOR: MODERATE\\}

          \item{
            TITLE: Расположение элементов\\
            \sr{Окно условно делится на три части, расположенных одна под другой.}\\
            \sr{В верхней части в два ряда расположены элементы ELTAX-\ref{driver_element_driver_balance_disp}, ELTAX-\ref{driver_element_driver_pozyvnoy_disp}, ELTAX-\ref{driver_element_driver_auth_state_disp}, ELTAX-\ref{driver_element_driver_curr_coords_disp}, ELTAX-\ref{driver_element_this_order_time_disp_after_in_place_button}, ELTAX-\ref{driver_element_this_order_dist_disp_after_in_place_button}.}\\
            \sr{В средней части расположен элемент ELTAX-\ref{driver_element_order_costs_disp_after_in_place_button}.}\\
            \sr{В нижней части расположен элемент ELTAX-\ref{driver_element_driving_button_disp_after_in_place_button}. После нажатия на элемент ELTAX-\ref{driver_element_driving_button_disp_after_in_place_button}, на его месте появляется новый элемент ELTAX-\ref{driver_element_end_button_disp_after_in_place_button}.}\\
            \sr{Слева расположена скрытая боковая панель ELTAX-\ref{driver_element_order_info_disp}.}\\
            PRIOR: MODERATE\\}

        \end{itemize}

        \setlength{\extrarowheight}{2mm}
          \begin{longtable}{|p{3cm}|p{3cm}|p{9cm}|}
              
          \caption {Новые элементы вкладки после нажатия кнопки “На месте”} \label{driver_app_taximeter_tab_after_start_button_elements} \\

            \hline  \textbf{ID}  & \textbf{Название} & \textbf{Требования/Описание} \\ [2mm]
            \endfirsthead
            \hline  \textbf{ID}  & \textbf{Название} & \textbf{Требования/Описание} \\ [2mm]
            \endhead

            \hline \eltax{driver_element_this_order_time_disp_after_in_place_button}{} & Время, затраченное водителем на текущий заказ & \sr{Отображается счетчик времени, затраченного на текущий заказ, в формате [ЧЧ:ММ:СС]. Динамически увеличивается ежесекундно после нажатия кнопки “На месте”(ELTAX-\ref{driver_element_in_place_button_disp}).}\\ [2mm]

            \hline \eltax{driver_element_this_order_dist_disp_after_in_place_button}{} & Километраж текущего заказа & \sr{Отображается счетчик километража текущего заказа в формате [км,м]. Динамически увеличивается на соответствующее проеханное водителем расстояние после нажатия кнопки “В пути”(ELTAX-\ref{driver_element_driving_button_disp_after_in_place_button}).}\\ [2mm]

            \hline \eltax{driver_element_order_costs_disp_after_in_place_button}{} & Стоимость поездки & \sr{Отображается счетчик стоимости поездки в формате [Значение счетчика] + [руб.]. Начальное значение устанавливается автоматически и равняется минимальной стоимости заказа, которая зависит от выбранного тарифа и других настроек заказа. Начинает динамически увеличиваться после того, как стоимость заказа будет превышать минимальную стоимость. }\\ [2mm]

            \hline \eltax{driver_element_driving_button_disp_after_in_place_button}{} & Кнопка “В пути” & \sr{Нажатие означает, что клиент сел в машину и водитель начал путь к месту назначения.}\\ [2mm]

            \hline \eltax{driver_element_end_button_disp_after_in_place_button}{} & Кнопка “Завершение” & \sr{Нажатие означает, что водитель достиг места назначения. Значения счетчика ELTAX-\ref{driver_element_order_costs_disp_after_in_place_button}, а также элементов ELTAX-\ref{driver_element_this_order_time_disp_after_in_place_button} и ELTAX-\ref{driver_element_this_order_dist_disp_after_in_place_button} прекращают увеличиваться, их значения сохраняются.}\\ [2mm]

            \hline

          \end{longtable}

      \paragraph{Вкладка после нажатия кнопки “Завершение”} \label{driver_app_taximeter_tab_after_end_button_disp} 
        \begin{itemize}

          \item{
            TITLE: Элементы вкладки после нажатия кнопки “Завершение”\\
            \sr{Во вкладке после нажатия кнопки “Завершение” в добавок либо в замен старым должны появляться новые элементы, которые перечислены и описаны в таблице \ref{driver_app_taximeter_tab_after_end_button_elements}.}\\
            PRIOR: MODERATE\\}

          \item{
            TITLE: Расположение элементов\\
            \sr{Окно условно делится на три части, расположенных одна под другой.}\\
            \sr{В верхней части в два ряда расположены элементы ELTAX-\ref{driver_element_driver_balance_disp}, ELTAX-\ref{driver_element_driver_pozyvnoy_disp}, ELTAX-\ref{driver_element_driver_auth_state_disp}, ELTAX-\ref{driver_element_driver_curr_coords_disp}, ELTAX-\ref{driver_element_this_order_time_disp_after_end_button}, ELTAX-\ref{driver_element_this_order_dist_disp_after_end_button}.}\\
            \sr{В средней части расположен элемент ELTAX-\ref{driver_element_order_costs_after_end_button_disp}.}\\
            \sr{В нижней части расположен элементы ELTAX-\ref{driver_element_additional_offers_button_disp}, ELTAX-\ref{driver_element_checkout_button_disp}, ELTAX-\ref{driver_element_change_button_disp}.}\\
            \sr{Слева расположена скрытая боковая панель ELTAX-\ref{driver_element_order_info_disp}.}\\
            PRIOR: MODERATE\\}
        \end{itemize}

        \setlength{\extrarowheight}{2mm}
          \begin{longtable}{|p{3cm}|p{3cm}|p{9cm}|}
              
          \caption {Новые элементы вкладки после нажатия кнопки “Завершение”} \label{driver_app_taximeter_tab_after_end_button_elements} \\

            \hline  \textbf{ID}  & \textbf{Название} & \textbf{Требования/Описание} \\ [2mm]
            \endfirsthead
            \hline  \textbf{ID}  & \textbf{Название} & \textbf{Требования/Описание} \\ [2mm]
            \endhead

            \hline \eltax{driver_element_this_order_time_disp_after_end_button}{} & Время, затраченное водителем на текущий заказ & \sr{Отображается сохраненное после нажатия кнопки “Завершение”(ELTAX-\ref{driver_element_end_button_disp_after_in_place_button}) значение счетчика ELTAX-\ref{driver_element_this_order_time_disp_after_in_place_button} .}\\ [2mm]

            \hline \eltax{driver_element_this_order_dist_disp_after_end_button}{} & Километраж текущего заказа & \sr{Отображается сохраненное после нажатия кнопки “Завершение”(ELTAX-\ref{driver_element_end_button_disp_after_in_place_button}) значение счетчика ELTAX-\ref{driver_element_this_order_dist_disp_after_in_place_button}.}\\ [2mm]

            \hline \eltax{driver_element_order_costs_after_end_button_disp}{} & Стоимость поездки & \sr{Отображается сохраненное после нажатия кнопки “Завершение”(ELTAX-\ref{driver_element_end_button_disp_after_in_place_button}) значение счетчика ELTAX-\ref{driver_element_order_costs_disp_after_in_place_button}.}\\ [2mm]

            \hline  \eltax{driver_element_additional_offers_button_disp}{} & Кнопка “Доп. услуги” & По нажатию кнопки “Доп. услуги” становится возможным выбор каких-либо дополнительных услуг, которые предоставил водитель во время поездки. \\ [2mm] 

            \hline \eltax{driver_element_checkout_button_disp}{} & Кнопка “Оплата” & \sr{По нажатию кнопки “Оплата”: \begin{itemize} \item Интерфейс изменяется на такой, как описано в разделе \ref{driver_app_taximeter_tab_first_opening}. \item Мобильное приложение делает запрос на сервер о завершении заказа, в котором передает итоговую стоимость поездки, коей является сохраненное значение счетчика ELTAX-\ref{driver_element_order_costs_disp_after_in_place_button} после нажатия кнопки “Завершение”(ELTAX-\ref{driver_element_end_button_disp_after_in_place_button}) или после изменения значения этого счетчика с помощью кнопки “Изменить”(ELTAX-\ref{driver_element_change_button_disp}). В ответе сервер присваивает водителю статус “Свободен”. \item Все вкладки приложения вновь становятся активными. \end{itemize}}\\ [2mm]

            \hline  \eltax{driver_element_change_button_disp}{} & Кнопка “Изменить” & По нажатию кнопки “Изменить” становится возможным изменение счетчика ELTAX-\ref{driver_element_order_costs_after_end_button_disp} вручную, но только в большую сторону. \\ [2mm]

            \hline

          \end{longtable}

  \subsection{Заказы} \label{driver_app_orders_tab}

    DESC: При нажатии на вкладку “Заказы” отображается страница со свободными заказами, вложенной вкладкой переключения на зарезервированные заказы и картой.

      \begin{itemize}

        \item{
          TITLE: Элементы вкладки\\
          \sr{Во вкладке должны присутствовать элементы, которые перечислены и описаны в таблице \ref{driver_app_orders_tab_table}.}\\
          PRIOR: MODERATE\\}

        \item{
          TITLE: Расположение элементов\\
          \sr{Окно делится на две части условной вертикальной линией.}\\
          \sr{У левого края окна находятся элементы ELTAX-\ref{orders_tab_element_free_orders}, ELTAX-\ref{orders_tab_element_rezerv_orders}.}\\
          \sr{В левой части находится элемент ELTAX-\ref{orders_tab_element_order}.}\\
          \sr{В правой части находится элемент ELTAX-\ref{orders_tab_element_map}.}\\
          PRIOR: MODERATE\\}

        \item{
          TITLE: Общие требования к функционалу\\
          \sr{Мобильное приложение принимает от сервера информацию о заказах, доступных водителю.}\\
          \sr{При приёме водителем какого-либо заказа мобильное приложение отправляет сообщение на сервер о закреплении заказа за водителем. Затем сервер присылает ответ о том, что заказ закреплен/не закреплен за водителем.}\\
          PRIOR: MODERATE\\}
      \end{itemize}

      %%%Таблица описания вкладки “Заказы”
      \begin{table}[h]
        \begin{center}
        \caption {Элементы вкладки “Заказы”}
        \label{driver_app_orders_tab_table}
        \setlength{\extrarowheight}{2mm}
        \begin{tabular}{|p{3cm}|p{3cm}|p{9cm}|}
          \hline     \textbf{ID}  & \textbf{Название} & \textbf{Требования/Описание} \\ [2mm]

          \hline \eltax{orders_tab_element_free_orders}{} & Вложенная вкладка “Свободные” & \sr{По входу в эту вкладку отображаются все свободные заказы (ELTAX-\ref{orders_tab_element_order}), они отсортированы по срочности и времени подачи.}\\ [2mm]

          \hline \eltax{orders_tab_element_rezerv_orders}{} & Вложенная вкладка “Резерв” & \sr{По входу в эту вкладку отображаются все предварительные заказы (ELTAX-\ref{orders_tab_element_order}).}\\ [2mm]

          \hline \eltax{orders_tab_element_order}{} & Список заказов & \sr{Список состоит из карточек “Заказ” (описание в разделе \ref{orders_tab_element_order_description}).} \\ [2mm]
            
          \hline \eltax{orders_tab_element_map}{} & Карта & \sr{Яндекс карта. На ней специальными отметками отображается текущее местоположение водителя и места, где появляются заказы.}\\ [2mm]

          \hline
        \end{tabular}
        \end{center}
      \end{table}

      \subsubsection{Карточка “Заказ”}  \label{orders_tab_element_order_description}

          DESC: Заказы в списке отображаются в виде карточек.

          \begin{itemize}
              
              \item{
                TITLE: Элементы карточки “Заказ”\\
                \sr{В карточке должны присутствовать элементы, которые перечислены и описаны в таблице \ref{orders_tab_element_table_order_description}.}\\
                PRIOR: MODERATE\\}
                
              \item{  
                TITLE: Расположение элементов карточки\\
                \sr{Окно условно делится на четыре строки.}\\
                \sr{В первой строке находятся элементы ELTAX-\ref{orders_tab_element_order_class}, ELTAX-\ref{orders_tab_element_order_type}.}\\
                \sr{Во второй строке находятся элементы ELTAX-\ref{orders_tab_element_order_from}, ELTAX-\ref{orders_tab_element_order_time_come_in}.}\\
                \sr{В третьей строке находятся элементы ELTAX-\ref{orders_tab_element_order_where_now}, ELTAX-\ref{orders_tab_element_order_time_to_aim}.}\\
                \sr{В четвертой строке находятся элементы ELTAX-\ref{orders_tab_element_order_demands}, ELTAX-\ref{orders_tab_element_order_dist_to_aim}.}\\
                PRIOR: MODERATE\\}

              \item{
                TITLE: Общие требования к карточке “Заказ”\\
                \sr{При нажатии на любую карточку “Заказ” из списка появляется элемент ELTAX-\ref{orders_tab_element_order_modal_window}.  }\\
                \sr{Если водитель принимает заказ от Яндекс.Такси, то при ожидании подтверждения от сервера подтверждения, что водитель назначен на заказ, на карточку выбранного заказа накладывается прелоадер (gif-анимация ожидания).}\\
                PRIOR: MODERATE\\}
   
            \end{itemize} 

            %%%Таблица описания карточки “Заказ”
            \setlength{\extrarowheight}{2mm}
              \begin{longtable}{|p{3cm}|p{3cm}|p{9cm}|}
              \caption {Элементы формы “Заказ”} \label{orders_tab_element_table_order_description}\\

                \hline  \textbf{ID}  & \textbf{Название} & \textbf{Требования/Описание} \\ [2mm]
                \endfirsthead
                \hline  \textbf{ID}  & \textbf{Название} & \textbf{Требования/Описание} \\ [2mm]
                \endhead

                \hline  \eltax{orders_tab_element_order_class}{} & Класс заказа & \sr{Существует три класса заказов: \begin{itemize} \item Эконом \item Комфорт \item Бизнес \end{itemize}}\\ [2mm]
                \hline  \eltax{orders_tab_element_order_type}{} & Тип заказа & \sr{Существует два вида заказов: \begin{itemize} \item Срочный \item Предварительный \end{itemize}} \\ [2mm]

                \hline  \eltax{orders_tab_element_order_from}{} & Куда & \sr{Отображается адрес источника заказа.}\\ [2mm]

                \hline  \eltax{orders_tab_element_order_time_come_in}{} & Время поступления заказа & \sr{Отображается время, когда появился заказ.}\\ [2mm]

                \hline  \eltax{orders_tab_element_order_where_now}{} & Откуда & \sr{Отображается адрес текущего местоположения водителя.}\\ [2mm]

                \hline  \eltax{orders_tab_element_order_time_to_aim}{} & Время до источника заказа & \sr{Отображается время, необходимое водителю для того, чтобы добраться до источника заказа. Просчитывается Яндекс.Навигатором.}\\ [2mm]

                \hline  \eltax{orders_tab_element_order_demands}{} & Примечания (требования) к заказу & \sr{Отображаются дополнительные требования к выполнению заказа.}\\ [2mm]

                \hline  \eltax{orders_tab_element_order_dist_to_aim}{} & Расстояние до источника заказа & \sr{Отображается расстояние в формате [км,м] до источника заказа. Просчитывается Яндекс.Навигатором.}\\ [2mm]

                \hline  \eltax{orders_tab_element_order_modal_window}{} & Модальное окно приёма заказа & \sr{Содержит следующие элементы: \begin{itemize} \item Заголовок - [“Принять заказ?”] \item Тип заказа \item Класс заказа \item Адрес, откуда поступил заказ \item Адрес, место назанчения заказа \item На какое время назначен заказ \item Наценки \item Кнопка “Принять” - принять выбранный заказ \item Кнопка “Отменить” - скрыть выбранный заказ \end{itemize}}\\ [2mm]

                \hline
              \end{longtable}

  \subsection{Навигатор} \label{driver_app_navigator_tab}

    DESC: В мобильное приложение встроен Яндекс.Навигатор.

    \subsubsection{Функциональные требования}

      \begin{itemize}
        \item{
          TITLE: Запуск навигатора.\\
          \sr{При нажатии на вкладку "Навигатор", на устройстве запускается приложение "Яндекс.Навигатор". Если приложение не установлено, то пользователю предлагается загрузить приложение на устройство из Play Market.}\\
          PRIOR: MODERATE\\}

        \item{
          TITLE: Построение маршрута для заказа.\\
          \sr{При выполнении текущего заказа в навигатор передаются координаты в зависимости от стадии заказа. По этим координатам строится маршрут.}\\
          \sr{Если статус заказа "Едет к клиенту", то в навигатор передаются: [текущее местоположение водителя] + [начальная точка заказа].}\\
          \sr{Если статус заказа "В пути", то в навигатор передаются: [начальная точка заказа] + [конечная точка заказа].}\\
          PRIOR: MODERATE\\}
      \end{itemize}

  \subsection{Счёт} \label{driver_app_bill_tab}

    DESC: Во вкладке баланс водитель может посмотреть актуальное состояние фискального и арендного баланса, а так же пополнить их. 

      TITLE: Элементы вкладки\\
      \sr{Во вкладке должны присутствовать элементы, которые перечислены и описаны в таблице \ref{driver_app_balance_tab_elements}.}\\
      PRIOR: MODERATE\\

      TITLE: Расположение элементов\\
      \sr{Экран делится на две части, при помощи вертикальной линии, в соотношении 6,5 : 3,5.}\\
      \sr{В левой части расположен элемент ELTAX-\ref{driver_element_transaction_area}}\\
      \sr{В правой части расположены элементы ELTAX-\ref{driver_element_fiscal_balance}, ELTAX-\ref{driver_element_rent_balance}, ELTAX-\ref{driver_element_ui_update_balance} в том же порядке в котором они преречислены.}\\
      PRIOR: MODERATE\\

      \begin{table}
        \begin{center}
        \caption{Элементы вкладки "Счет"}
        \label{driver_app_balance_tab_elements}
        \setlength{\extrarowheight}{2mm}
        \begin{tabular}{|p{3cm}|p{4cm}|p{8cm}|}
           \hline   \textbf{ID}&  \textbf{Название}&\textbf{Требования/Описание} \\ [2mm]


           \hline \eltax{driver_element_fiscal_balance}{} & Фискальный баланс & \sr{Элемент состоит из поля - [Фискальный баланс в рублях]}\\ [2mm]

           \hline \eltax{driver_element_rent_balance}{} & Арендный баланс & \sr{Элемент состоит из двух полей: [Баланс  в рублях], [Баланс конвертированный в дни]}\\ [2mm]

           \hline \eltax{driver_element_ui_update_balance}{} & Интерфейс выставления счета  & --- НАХОДИТСЯ В СТАДИИ РАЗРАБОТКИ ---\\ [2mm]

           \hline \eltax{driver_element_transaction_area}{} & Область транзакций  & --- НАХОДИТСЯ В СТАДИИ РАЗРАБОТКИ ---\\ [2mm]    

           \hline
        \end{tabular}
        \end{center}
      \end{table}   

  \subsection{Робот} \label{driver_app_robot_tab}

    DESC: При переходе на вкладку “Робот” отображается список настроек и список режимов и опций для каждого из них.

    \subsubsection{Общие требования} \mbox{}\\

      \begin{itemize}

        \item{
          TITLE: Отправка настроек и опций на сервер.\\
          \sr{При изменении опций или параметров, мобильное приложение автоматически отправляет обновленные настройки на сервер.}\\
          PRIOR: MODERATE\\
          }

        \item{
          TITLE: Сохранение настроек.\\
          \sr{Для каждого пользователя настройки сохраняются в базе СТ.}\\
          NOTE: При перезапуске или при разрыве соединения сервер возвращает приложению последние записанные в базу СТ настройки для конкретного пользователя.\\
          PRIOR: MODERATE\\
          }

      \end{itemize}

    \subsubsection{Интерфейс вкладки} \mbox{}\\

      \begin{itemize}

        \item{
          TITLE: Деление экрана.\\
          \sr{Экран делится на две части, при помощи вертикальной линии, в соотношении 1 : 1.}\\          
          PRIOR: MODERATE\\}

        \item {
          TITLE: Расположение общих настроек.\\
          \sr{В левой части расположен список общих настроек робота. (Описание - \ref{driver_app_robot_tab_general_settings})}\\
          PRIOR: MODERATE\\}

        \item {
          TITLE: Расположение настроек робота.\\
          \sr{В правой части расположен список режимов работы робота. (Описание - \ref{driver_app_robot_tab_robot_settings})}\\
          PRIOR: MODERATE\\}

      \end{itemize}

    \subsubsection{Общие настройки} \label{driver_app_robot_tab_general_settings} \mbox{}\\

      \begin{itemize}

        \item{
          \sr{В списке должны присутствовать настройки, перечисленные в таблице \ref{driver_app_robot_tab_table_general_settings}.}\\
          PRIOR: MODERATE\\}

      \end{itemize}

      %%%Таблица общих настроек робота.
      \begin{table}[h] 
        \begin{center}
        \caption {Общие настройки робота}
        \label{driver_app_robot_tab_table_general_settings}
        \setlength{\extrarowheight}{2mm}
        \begin{tabular}{|p{4cm}|p{3cm}|p{8cm}|}

          \hline     \textbf{Название}&\textbf{Способ изменения настройки}&\textbf{Описание} \\ [2mm]

          \hline   Заказы классом ниже & Переключатель ON/OFF & Водитель с машиной определенного класса может принимать заказы ниже по классу.\\ [2mm]

          \hline   Заказы от Яндекса & Переключатель ON/OFF & Возможность принимать заказы через канал Яндекс.Такси. \\ [2mm]
            
          \hline   Цепочка & Переключатель ON/OFF & Возможность закрепления за водителем заказов “в цепочку”. \\ [2mm]

          \hline   Дистанция поиска & Поле для ввода & Возможность задания водителем километража, в пределах которого сервер подбирает заказ. \\ [2mm]

          \hline   Время подачи & Поле для ввода & Возможность задания водителем максимального времени, за которое он хочет добраться до начальной точки заказа. \\ [2mm]

          \hline
        \end{tabular}
        \end{center}
      \end{table}

    \subsubsection{Роботы} \label{driver_app_robot_tab_robot_settings} \mbox{}\\

      \begin{itemize}

        \item{
          TITLE: Режимы работы роботов. \\
          \sr{В списке должны присутствовать режимы, перечисленные в таблице \ref{driver_app_robot_tab_modes}.}\\
          PRIOR: MODERATE\\}
      \end{itemize}

      %%%Таблица режимов работы робота.
      \begin{table}
          \begin{center}
          \caption {Режимы работы робота}
          \label{driver_app_robot_tab_modes}
          \setlength{\extrarowheight}{2mm}
          \begin{tabular}{|p{3cm}|p{3cm}|p{6cm}|p{3cm}|}

            \hline     \textbf{Название режима работы}&\textbf{Способ выбора режима работы}&\textbf{Список настроек}&\textbf{Описание настроек} \\ [2mm]

            \hline   Городской & Переключатель ON/OFF & Можно изменить следующие настройки: \begin{itemize} \item Округ. \end{itemize} & См. в таблице \ref{table:driver_app_robot_tab_table_town_mode} \\ [2mm]

            \hline   Портовый & Переключатель ON/OFF & Можно изменить следующие настройки: \begin{itemize} \item Порты.;  \item Встреча;  \item Проводы. \end{itemize} & См. в таблице \ref{table:driver_app_robot_tab_table_port_mode}  \\ [2mm]

            \hline
          \end{tabular}
          \end{center}
      \end{table}

      %%%Таблица описания настроек в городском режиме работы.
      \begin{table}
          \begin{center}
          \caption {Описание настроек в городском режиме работы}
          \label{table:driver_app_robot_tab_table_town_mode}
          \setlength{\extrarowheight}{2mm}
          \begin{tabular}{|p{4cm}|p{3cm}|p{8cm}|}

            \hline     \textbf{Название}&\textbf{Способ изменения}&\textbf{Описание} \\ [2mm]

            \hline   Округ & Выпадающий список & Водитель помечает чекбоксом округи в списке, таким образом исключая заказы с конечной точкой в выбранных округах. Доступен выбор из следующих округов: \begin{itemize} \item ЦАО \item САО \item ЮАО \item СЗАО \item ЮЗАО \item ВАО \item СВАО \item ЮВАО \item ЗАО \end{itemize} \\ [2mm]

            \hline
          \end{tabular}
          \end{center}
      \end{table}        

      %%%Таблица описания настроек в портовом режиме работы.
      \begin{table}
          \begin{center}
          \caption {Описание настроек в портовом режиме работы}
          \label{table:driver_app_robot_tab_table_port_mode}
          \setlength{\extrarowheight}{2mm}
          \begin{tabular}{|p{4cm}|p{3cm}|p{8cm}|}

            \hline     \textbf{Название}&\textbf{Способ изменения}&\textbf{Описание} \\ [2mm]

            \hline   Порты & Выпадающий список & Водитель помечает чекбоксом названия аэропортов. Доступен выбор из следующих аэропортов: \begin{itemize} \item Домодедово \item Шереметьево \item Внуково \end{itemize} \\ [2mm]

            \hline   Встреча & Checkbox & Опция, указывающая на то, что местом назначения для заказа этого типа будет один или несколько аэропортов, выбранных водителем. \\ [2mm]

            \hline   Проводы & Checkbox & Опция, указывающая на то, что местом прибытия для заказа этого типа является один или несколько аэропортов, выбранных водителем. \\ [2mm]

            \hline
          \end{tabular}
          \end{center}
      \end{table}

  \subsection{Настройки} \label{driver_app_settings_tab}

    DESC: Во вкладке “Настройки” пользователь может настроить мобильное приложение.

    \begin{itemize}

      \item{
      TITLE: Элементы вкладки\\
      \sr{По нажатию на вкладку должен появлятся выпадающий список, элементы которого перечислены и описаны в таблице \ref{options_tab_elements}.\\ NOTE: Некоторые элементы доступны только в определенном режиме работы мобильного приложения. В каком режиме работы доступен каждый элемент описано в столбце “Отображение в режиме”.}\\
      PRIOR: MODERATE\\}

      \item{
      TITLE: Отображение элементов списка\\
      \sr{Элемент списка состоит из двух частей: 
        \begin{itemize} 
          \item{Иконка. Расположена слева. Для каждого элемента она соответствующая, описана в столбце “Иконка” таблицы \ref{options_tab_elements}.}
          \item{Справа от иконки расположено название элемента.}
        \end{itemize}}
      PRIOR: MODERATE\\}

    \end{itemize}

    \setlength{\extrarowheight}{2mm}
        \begin{longtable}{|p{3cm}|p{3cm}|p{2cm}|p{7cm}|}
            
        \caption {Элементы выпадающего списка вкладки “Настройки”} \label{options_tab_elements} \\
          \hline  \textbf{Название} & \textbf{Отбражение в режиме} & \textbf{Иконка} & \textbf{Требования/Описание} \\ [2mm]
          \endfirsthead
          \hline  \textbf{Название} & \textbf{Отбражение в режиме} & \textbf{Иконка} & \textbf{Требования/Описание} \\ [2mm]
          \endhead

          \hline Checkbox “Занят” & Пользователь авторизован & N/A & \sr{При установке checkbox-а, мобильное приложение отправляет серверу запрос на изменение статуса. Пока checkbox установлен на водителя накладываются ограничения в соответствии с требованиями владельца.}\\ [2mm]

          \hline Вкладка “Уведомления” & Пользователь авторизован/Автономный & Колокол & \sr{Описание в разделе \ref{options_tab_notifications_driver_app}.}\\ [2mm]

          \hline Вкладка “Тарифы” & Пользователь авторизован & Кошелек и доллар & 
          \begin{itemize} 
            \item{
            \sr{При нажатии на вкладку “Тарифы” выдвигается информативное модальное окно, в котором находится актуальная информация о тарифах.}} 
            \item{
              \sr{Информация о тарифе представляет собой карточку. В каждой карточке должно быть указано: 
                \begin{itemize} 
                  \item{Название тарифа} 
                  \item{Минимальная стоимость поездки} 
                  \item{Время бесплатного ожидания и стоимость минуты простоя по истечению этого времени.} 
                  \item{
                    Внизу карточки находится кнопка “подробнее...”. По нажатию на кнопку должна появляться следующая информация о тарифе: 
                    \begin{itemize} 
                      \item{Тарифы по Москве} 
                      \item{Тарифы за пределами МКАД} 
                      \item{Расценки на поездки в аэропорты} 
                      \item{Поездки из одного аэропорта в другой} 
                    \end{itemize}
                    } 
                \end{itemize}
              }
            } 
          \end{itemize}\\ [2mm]

          \hline Вкладка “Вопрос/Ответ” & Пользователь авторизован/Автономный & Знак вопроса и знак восклицания & \sr{При нажатии на вкладку “Вопрос/Ответ” выдвигается информативное модальное окно, в котором находится список часто задаваемых вопросов и ответы на них.}\\ [2mm]

          \hline Вкладка “Контакты” & Пользователь авторизован & Анфас человека и телефонная трубка & \sr{При нажатии на вкладку “Контакты” выдвигается информативное модальное окно, в котором находится информация о контактах службы владельца.}\\ [2mm]
    
          \hline Вкладка “Тех. поддержка” & Пользователь авторизован & Молоток с имитацией удара & \sr{В этой вкладке пользователь имеет возможность сообщить об ошибке или написать пожелание по доработке разработчикам приложения. При нажатии на вкладку “Тех. поддержка” выдвигается модальное окно, элементы которого описаны в таблице \ref{options_tab_tech_help_elements}.}\\ [2mm]

          \hline Вкладка “Основные настройки” & Пользователь авторизован/Автономный & Две шестеренки & \sr{Описание в разделе \ref{options_tab_global_options}.}\\ [2mm]
     
          \hline Вкладка “Выбрать сервер” & Пользователь авторизован/Автономный & Две шестеренки & \sr{При нажатии на вкладку “Выбрать сервер” на экране появляется модальное окно с элементами, которые описаны в таблице \ref{options_tab_select_server_elements}.}\\ [2mm]

          \hline Кнопка “Войти” & Автономный & N/A & \sr{Описание в разделе \ref{options_tab_authorization}.}\\ [2mm]

          \hline Кнопка “Выход” & Пользователь авторизован/Автономный & N/A & \sr{По нажатию кнопки на экране появляется модальное окно с запросом подтверждения закрытия приложения.}\\ [2mm]
  
          \hline

        \end{longtable}

      %%%Техподдержка
      \begin{table}
        \begin{center}
        \caption{Элементы вкладки “Тех. поддержка”}
        \label{options_tab_tech_help_elements}
        \setlength{\extrarowheight}{2mm}
        \begin{tabular}{|p{3cm}|p{3cm}|p{9cm}|}
           \hline   \textbf{ID}&  \textbf{Название}&\textbf{Требования/Описание} \\ [2mm]

           \hline \eltax{tech_help_message}{} & Написать сообщение & \sr{Поле для ввода. Сюда пользователь может ввести содержание своего пожелания по работе приложения.}\\ [2mm]

           \hline \eltax{tech_help_send_message}{} & Отправить & \sr{Кнопка. При нажатии на кнопку мобильное приложение отправляет соответствующий запрос, содержащий в себе сообщение, написанное пользователем в поле для ввода ELTAX-\ref{tech_help_message}, на сервер.}\\ [2mm]

           \hline \eltax{tech_help_send_message}{} & Закрыть & \sr{Кнопка. При нажатии закрывается модальное окно.}\\ [2mm]

           \hline
        \end{tabular}
        \end{center}
      \end{table}

      %%%Выбрать сервер
      \begin{table}
        \begin{center}
        \caption{Элементы вкладки “Выбрать сервер”}
        \label{options_tab_select_server_elements}
        \setlength{\extrarowheight}{2mm}
        \begin{tabular}{|p{3cm}|p{3cm}|p{9cm}|}
           \hline   \textbf{ID}&  \textbf{Название}&\textbf{Требования/Описание} \\ [2mm]

           \hline \eltax{select_server_url}{} & Url-адрес & \sr{Поле для ввода. Сюда пользователь вводит url-адрес сервера, к которому хочет подключиться.}\\ [2mm]

           \hline \eltax{select_server_set_button}{} & Установить & \sr{Кнопка. Подтверждает ввод нового url-адреса, приложение совершает попытку подключения к новому серверу.}\\ [2mm]

           \hline
        \end{tabular}
        \end{center}
      \end{table}  

    \subsubsection{Уведомления} \label{options_tab_notifications}

      \paragraph{Функциональные требования} \mbox{}\\
        \begin{itemize}

          \item{TITLE: Новое уведомление. \\
                \sr{Новое уведомление помечается как непрочитанное, если пользователь не находится во вкладке "Уведомления" DriverApp. Инкрементируется счетчик непрочитанных уведомлений. Если на момент поступления нового уведомления пользователь находится во вкладке "Уведомления" DriverApp, новое уведомление помечается как прочитанное и счетчик непрочитанных уведомлений не инкрементируется.} \\
                PRIOR: MODERATE \\}

          \item{TITLE: Прочитанное уведомление. \\
                \sr{Уведомления считаются прочитанными, если была открыта вкладка "Уведомления" в DriverApp. Счетчик непрочитанных уведомлений при этом обнуляется.} \\
                PRIOR: MODERATE \\}

          \item{TITLE: Счетчик непрочитанных уведомлений. \\
                \sr{Означает количество непрочитанных уведомлений на данный момент. Инкрементируется, если появилось новое непрочитанное уведомление. Обнуляется при просмотре непрочитанных уведомлений через вкладку "Уведомления" в DriverApp.} \\
                PRIOR: MODERATE \\}
          
          \item{TITLE: Иконка. \\
                \sr{Выбирается в зависимости от инициатора уведомления из "Таблицы инициаторов уведомления".} \\
                PRIOR: MODERATE \\}

        \end{itemize}

      \paragraph{Уведомления в панели уведомлений Android(ПУА)} \mbox{}\\

        \subparagraph{Функциональные требования} \mbox{}\\

          \begin{itemize}

            \item{TITLE: Появление нового уведомления. \\
                  \sr{При появлении нового уведомления от какого-либо инициатора уведомления, оно выводится в панели уведомлений Android(ПУА). Если в данный момент появляется сразу несколько новых уведомлений, счетчик непрочитанных уведомлений инкрементируется на их количество.} \\
                  PRIOR: MODERATE \\}

            \item{TITLE: Открытие уведомления. \\
                  \sr{По нажатию на уведомление в ПУА, открывается DriverApp на вкладке "Уведомления". При этом, счетчик непрочитанных уведомлений обнуляется.} \\
                  PRIOR: MODERATE \\}

            \item{TITLE: Разрешенные действия с уведомлениями. \\
                  \sr{Разрешается нажать на уведомление в ПУА. В этом случае откроется DriverApp на вкладке "Уведомления", счетчик непрочитанных уведомлений обнулится.} \\ 
                  \sr{Разрешается убрать уведомление из ПУА путем скрола уведомления в сторону. В этом случае не произойдет никаких действий, счетчик непрочитанных уведомлений не обнулится.} \\
                  PRIOR: MODERATE \\}

            \item{TITLE: Интерфейс уведомления. \\
                  \sr{В панели уведомлений Android уведомления отображаются согласно \href{https://www.evernote.com/shard/s389/sh/8d4e738c-61f0-4500-98bb-5882d89c9906/9ab0dc75caa4a5144ac1e5042e808b4f}{прототипу}.} \\
                  PRIOR: MODERATE \\}

            \item{TITLE: Строка "Новых уведомлений"(верхняя строка). \\
                  \sr{Отображется: [Новых уведомлений: ] + [общее количество непрочитанных на данный момент уведомлений].} \\
                  PRIOR: MODERATE \\}

            \item{TITLE: Строка "Последних уведомлений"(нижняя строка). \\                          
                  \sr{Отображется: [последнее непрочитанное уведомление] + [,] + [предпоследнее непрочитанное уведомление] + [...] в конце, если непрочитанных сообщений больше двух.} \\
                  PRIOR: MODERATE \\}

            \item{TITLE: Дата. \\
                  \sr{Отображается время появления уведомления в формате "Часы:Минуты".} \\
                  PRIOR: MODERATE \\}

          \end{itemize}

      \paragraph{Уведомления в DriverApp во вкладке Уведомления} \mbox{}\\ \label{options_tab_notifications_driver_app}

        \subparagraph{Функциональные требования} \mbox{}\\

          \begin{itemize}

            \item{TITLE: Появление нового уведомления. \\
                  \sr{Непрочитанные уведомления появляются сверху списка во вкладке "Уведомления".} \\
                  PRIOR: MODERATE \\}

            \item{TITLE: Открытие вкладки "Уведомления". \\
                  \sr{При открытии вкладки "Уведомления", обнуляется счетчик непрочитанных уведомлений.} \\
                  PRIOR: MODERATE \\}

            \item{TITLE: Разрешенные действия с уведомлениями. \\
                  \sr{Уведомления во вкладке "Уведомления" разрешается только просмотреть.} \\
                  PRIOR: MODERATE \\}

            \item{TITLE: Интерфейс уведомления. \\
                  \sr{Во вкладке "Уведомления" уведомления отображаются согласно \href{https://www.evernote.com/shard/s389/sh/f7d9319d-bb3a-47de-b857-5a37e440419c/774bb30105ccb402d93522ae9e333b5e}{прототипу}.} \\
                  PRIOR: MODERATE \\}

            \item{TITLE: Строка "Уведомления"(верхняя строка). \\
                  \sr{Отображается содержание уведомления.} \\
                  PRIOR: MODERATE \\}

            \item{TITLE: Строка "Дата и время"(нижняя строка). \\
                  \sr{Отображается дата и время появления уведомления в формате "ДД.ММ.ГГГГ Часы:Минуты".} \\
                  PRIOR: MODERATE \\}

          \end{itemize}

      % Таблица уведомлений
      \label{taxometr_notifications_table}
      \setlength{\extrarowheight}{2mm}
          \begin{longtable}{|p{3cm}|p{4cm}|p{5cm}|p{3cm}|}
              \caption {Таблица уведомлений}\\

              \hline     \textbf{ID}&\textbf{Название}&\textbf{Содержимое уведомления} & \textbf{Инициатор}\\ [2mm]
              \endfirsthead
              \hline     \textbf{ID}&\textbf{Название}&\textbf{Содержимое уведомления} & \textbf{Инициатор}\\ [2mm]
              \endhead
              
              \hline  \nttax{notif_driver_of_remove_driver_from_the_order}{} & Снятие водителя с заказа. & Вы сняты с заказа + [№Заказа]. & SRVACT-\ref{act_remove_driver_driver_notification} \\ [2mm]

              \hline \nttax{notif_of_new_order_in_reserve}{} & Добавление нового заказа в резерв. & Заказ + [№Заказа] + поступил в резерв. & \\ [2mm]       %TODO: Дописать SRVACT-\ref{act_add_new_order_in_reserve}

              \hline
          \end{longtable}

    \subsubsection{Основные настройки} \label{options_tab_global_options}

      \paragraph{Тарифы}

        \begin{itemize}
          
          \item{
          TITLE: Отображение в режимах работы приложения\\
          \sr{Отображается как при автономном режиме работы, так и в режиме работы с авторизованным пользователем.}\\
          PRIOR: MODERATE\\}

          \item{
          TITLE: Элементы вкладки\\
          \sr{По нажатию на вкладку Тарифы на экране должно открываться новое окно, в котором будут присутствовать элементы, перечисленные и описанные в таблице \ref{options_tab_global_options_tarifs}.}\\
          PRIOR: MODERATE\\}
        \end{itemize}

        \begin{table}
        \begin{center}
        \caption{Элементы вкладки “Тарифы”}
        \label{options_tab_global_options_tarifs}
        \setlength{\extrarowheight}{2mm}
        \begin{tabular}{|p{3cm}|p{3cm}|p{9cm}|}
           \hline   \textbf{ID}&  \textbf{Название}&\textbf{Требования/Описание} \\ [2mm]

           \hline \eltax{global_options_back_button}{} & Назад & \sr{Кнопка. Располагается в левом верхнем углу. По нажатию приложение возвращается в предыдущее окно.}\\ [2mm]

           \hline \eltax{global_options_costs_per_min}{} & Стоимость минуты поездки & \sr{Поле для ввода. Пользователь вводит числовое значение в это поле. Введенное число означает стоимость минуты поездки в рублях.}\\ [2mm]

           \hline \eltax{global_options_set_button}{} & Установить & \sr{Кнопка. По нажатию на кнопку на сервер отправляется сообщение со значением поля ELTAX-\ref{global_options_costs_per_min}. Приложение возвращается в предыдущее окно.}\\ [2mm]

           \hline
        \end{tabular}
        \end{center}
      \end{table}

      \paragraph{История заказов}                                                                       %%% Пусть пока здесь побудет

        \begin{itemize}
          \item{
          TITLE: Отображение в режимах работы приложения\\
          \sr{Отображается только в режиме работы с авторизованным пользователем.}\\
          PRIOR: MODERATE\\}
        \end{itemize}

      \paragraph{Фото}\mbox{}\\                                                                                           %%% Не понятно, зачем?

        Пользователь имеет возможность сделать фото.

      \paragraph{Уведомления}\mbox{}\\

        DESC: Пользователь имеет возможность настроить от каких источников он хочет получать уведомления.

        \begin{itemize}

          \item{
          TITLE: Отображение в режимах работы приложения\\
          \sr{Отображается только в режиме работы с авторизованным пользователем.}\\
          PRIOR: MODERATE\\}

          \item{
            TITLE: Элементы вкладки\\
            \sr{Вкладка должна содержать:
              \begin{itemize}
                \item{Cписок из всех возможных источников уведомлений. Каждый источник уведомления состоит из следующих элементов: 
                  \begin{itemize} 
                    \item{Название} 
                    \item{Свитчер ON/OFF - включить/выключить получение уведомлений от соответствующего источника} 
                  \end{itemize}
                  Все источники уведомлений делятся на 3 категории. Они описаны в таблице \ref{options_tab_global_options_notif}.
                }
                \item{Кнопка “Назад”. По нажатию на нее происходит возврат на предыдущее окно.}
              \end{itemize}  
            }
            PRIOR: MODERATE\\
          }
        \end{itemize}

        \begin{table}
        \begin{center}
        \caption{Источники уведомлений}
        \label{options_tab_global_options_notif}
        \setlength{\extrarowheight}{2mm}
        \begin{tabular}{|p{5cm}|p{10cm}|}
           \hline   \textbf{Название}&\textbf{Требования/Описание} \\ [2mm]

           \hline Для водителя & Категория должна содержать следующие источники уведомлений: \begin{itemize} \item{Новый заказ} \item{Новое сообщение} \item{Уведомления во время работы таксометра} \end{itemize}\\ [2mm]

           \hline Таксометр & Категория должна содержать следующие источники уведомлений: \begin{itemize} \item{Поступил заказ в цепочку} \item{Заказ отменён} \item{Заказ из резерва отменён} \item{Смена тарифной зоны} \end{itemize}\\ [2mm]

           \hline Таксометр & Категория должна содержать следующие источники уведомлений: \begin{itemize} \item{Нет GPS} \item{Нет интернета} \end{itemize}\\ [2mm]

           \hline
        \end{tabular}
        \end{center}
      \end{table}