\section{Водительское приложение}

  \subsection{Общие требования к мобильному приложению}

    \begin{itemize}

      \item{
        TITLE: Интернет подключение\\
        \sr{Для работы приложения необходимо интернет-подключение.}\\
        PRIOR: MODERATE\\}

      \item{
        TITLE: Отправка GPS данных на сервер\\
        \sr{Мобильное приложение оповещает сервер о местоположении водителя отсылая GPS координаты с периодичностью в STAT-\ref{periodicity_transfer_coordinates_taximeter_to_server} секунд.}\\
        PRIOR: MODERATE\\}

    \end{itemize}

  \subsection{Интерфейс приложения}
   
    \subsubsection{Общие требования к интерфейсу}
      \begin{itemize}

        \item{
          TITLE: Ориентация приложения на экране устройства\\
          \sr{Приложение имеет альбомную ориентацию.}\\
          PRIOR: MODERATE\\}

        \item{
          TITLE: Элементы интерфейса\\
          \sr{Элементы интерфейса перечислены и описаны в таблице \ref{app_interface}.}\\
          PRIOR: MODERATE\\}

      \end{itemize}

    \begin{table}
    \begin{center}
    \caption {Интерфейс приложения}
    \label{app_interface}
    \setlength{\extrarowheight}{2mm}
      \begin{tabular}{|p{5cm}|p{10cm}|}
        \hline     \textbf{Название элемента}&\textbf{Описание элемента} \\ [2mm]

        \hline   Лента вкладок & Находится вверху экрана. Содержит следующие вкладки: \begin{itemize} \item Таксометр. (Описание - \ref{driver_app_taximeter_tab}) \item Заказы. (Описание - \ref{driver_app_orders_tab}) \item Навигатор. (Описание - \ref{driver_app_navigator_tab}) \item Счёт. (Описание - \ref{driver_app_bill_tab}) \item Робот. (Описание - \ref{driver_app_robot_tab}) \item Настройки. (Описание - \ref{driver_app_settings_tab}) \end{itemize}\\ [2mm]

        \hline   Основная область & Находится под 'Лентой вкладок' и занимает оставшуюся часть экрана. В этой области отображается интерфейс выбранной вкладки.\\ [2mm]

        \hline
      \end{tabular}
    \end{center}
    \end{table}

  \subsection{Пользователи} \label{driver_app_users_tab}

      \subsubsection{Роли, состояния приложения}

      		DESC: У приложения есть два состояния, они перечислены и описаны в таблице \ref{app_state}. 

             \begin{table}
             \begin{center}
             \caption {Состояния приложений}
             \label{app_state}
             \setlength{\extrarowheight}{2mm}
             \begin{tabular}{|p{5cm}|p{10cm}|}
                 \hline     \textbf{Состояние}&\textbf{Описание, уровень доступа} \\ [2mm]

                 \hline   Авторизованный пользователь & В этом состоянии пользователю доступны все элементы интерфейса и функциональные возможности приложения.\\ [2mm]
                 \hline   Автономный режим & В этом состоянии пользователю доступны следующие элементы интерфейса: \begin{itemize} \item Таксометр \item Навигатор \item Настройки \end{itemize}\\ [2mm]
                  \hline
             \end{tabular}
             \end{center}
             \end{table}

  	  \subsubsection{Авторизация} \label{driver_app_users_tab_authorization}

  		  DESC: Водитель имеет возможность авторизоваться в специальном окне, которое появляется после выбора пункта “Войти” во вкладке “Настройки”.%Водитель имеет возможность выбрать из выпадающего списка название своей компании, ввести свой позывной (логин) и пароль в специальном окне. Мобильное приложение отправляет эти данные на сервер для проверки, и если они валидны, то открывается соединение с сервером и все элементы интерфейса становятся доступны.

        \paragraph{Общие требования}
          \begin{itemize}
            \item{
              TITLE: Активность элементов интерфейса для неавторизованного пользователя.\\
              \sr{Если водитель не авторизован, все элементы интерфейса, кроме вкладок “Настройки”, “Навигатор” и “Таксометр” становятся недоступными.}\\
              PRIOR: MODERATE\\}

            \item{
              TITLE: Критерии определения успешной авторизации \\
              \sr{Мобильное приложение отправляет на сервер для проверки данные, введенные водителем в специальном окне (Описание - \ref{authorization_form_interface}), а также GPS координаты - местоположение водителя, и если все эти данные валидны, то открывается соединение с сервером и все элементы интерфейса становятся доступны, водитель считается авторизованным.}\\
              PRIOR: MODERATE\\}
          \end{itemize}

        \paragraph{Интерфейс окна авторизации} \label{authorization_form_interface}
          Интерфейс окна авторизации описан в таблице \ref{authorization_interface}.

          \begin{table}[h]
            \begin{center}
            \caption {Интерфейс окна авторизации}
            \label{authorization_interface}
            \setlength{\extrarowheight}{2mm}
              \begin{tabular}{|p{4cm}|p{3cm}|p{4cm}|p{4cm}|}
                \hline     \textbf{Название}&\textbf{Способ отображения}&\textbf{Расположение в окне}&\textbf{Функциональные требования} \\ [2mm]

                \hline   Компания & Выпадающий список & Первая сверху строка & Описан в \ref{authorization_functional_company}.\\ [2mm]

                \hline   Логин & Поле для ввода & Под “Компания” & Описан в \ref{authorization_functional_login}. \\ [2mm]

                \hline   Пароль & Поле для ввода со скрытыми символами & Под “Логин” & Описан в \ref{authorization_functional_pwd}. \\ [2mm]

                \hline   Запомнить & Checkbox & Под “Пароль” & Описан в \ref{authorization_functional_remember}. \\ [2mm]

                \hline   Войти & Кнопка & Под “Запомнить” & Описан в \ref{authorization_functional_enter}. \\ [2mm]

                \hline   Работать автономно & Кнопка & Под “Войти” & Описан в \ref{authorization_functional_autonom}. \\ [2mm]

                \hline
              \end{tabular}
            \end{center}
          \end{table}

        \paragraph{Функциональные требования} 
          
          \subparagraph{Компания} \label{authorization_functional_company}

            \begin{itemize}
              \item{
                TITLE: Выбор компании\\
                \sr{Водитель выбирает название своей компании из выпадающего списка, который выдает сервер.}\\
                PRIOR: MODERATE\\}
            \end{itemize}

          \subparagraph{Логин}  \label{authorization_functional_login}

            \begin{itemize}
              \item{
                TITLE: Общие требования к логину\\
                \sr{Должен быть на английском языке. Водитель вводит свой позывной с помощью клавиатуры Android в поле для ввода.}\\
                PRIOR: MODERATE\\}
            \end{itemize}

          \subparagraph{Пароль} \label{authorization_functional_pwd}

            \begin{itemize}
              \item{
                TITLE: Общие требования к паролю\\
                \sr{Водитель вводит пароль к своему логину с помощью клавиатуры Android в поле для ввода.}\\
                PRIOR: MODERATE\\}
            \end{itemize}            

          \subparagraph{Запомнить}  \label{authorization_functional_remember}

            \begin{itemize}
              \item{
                TITLE: Общие требования к checkbox-у\\
                \sr{При наличии флажка в checkbox-е, введенные водителем логин и пароль сохраняются при следующем входе в приложение. При отсутствии флажка, введенные данные не сохраняются.}\\
                PRIOR: MODERATE\\}
            \end{itemize}          

          \subparagraph{Войти}  \label{authorization_functional_enter}

            \begin{itemize}
              \item{
                TITLE: Действия при нажатии на кнопку\\
                \sr{При нажатии на кнопку “Войти” на сервер отправляются данные, введенные водителем в поля “Логин”(\ref{authorization_functional_login}) и “Пароль”(\ref{authorization_functional_pwd}), отправляется состояние флажка “Запомнить”(\ref{authorization_functional_remember}) и выбранный элемент из списка “Компания”(\ref{authorization_functional_company}). На сервере происходит проверка введенных данных, и если они валидны, происходит авторизация водителя в приложении. Если данные не проходят валидацию, приложение возвращает водителя обратно на окно авторизации, и водителю вновь необходимо ввести данные.}\\
                PRIOR: MODERATE\\}
            \end{itemize}

          \subparagraph{Работать автономно} \label{authorization_functional_autonom}

            \begin{itemize}
              \item{
                TITLE: Действия при нажатии на кнопку\\
                \sr{При нажатии на кнопку “Работать автономно”, авторизация не происходит и на сервер не отправляются никакие данные. Приложение возвращает водителя на вкладку “Таксометр” (Описание - \ref{driver_app_taximeter_tab}).}\\
                PRIOR: MODERATE\\}
            \end{itemize}

  \subsection{Таксометр} \label{driver_app_taximeter_tab}

    DESC: При переходе на вкладку “Таксометр” отображается экран таксометра.

    \subsubsection{Интерфейс вкладки}

      \paragraph{Начальный интерфейс вкладки} \mbox{}\\ \label{driver_app_taximeter_tab_start_interface}
        DESC: Интерфейс вкладки при первом её открытии. \\
        Описание начального интерфейса вкладки находится в таблице \ref{driver_app_taximeter_tab_start_interface_table}.\\

        %%%Начальный интерфейс вкладки
        \begin{table}[h]
          \begin{center}
          \caption {Начальный интерфейс вкладки “Таксометр”}
          \label{driver_app_taximeter_tab_start_interface_table}
          \setlength{\extrarowheight}{2mm}
            \begin{tabular}{|p{4cm}|p{3cm}|p{4cm}|p{4cm}|}
              \hline     \textbf{Название}&\textbf{Расположение на экране}&\textbf{Требования к интерфейсу}&\textbf{Функциональные требования} \\ [2mm]

              \hline   Основные сведения о водителе & Верхняя панель & Описаны в таблице \ref{driver_app_taximeter_tab_start_interface_table_driver_info}. & Описан в \ref{taximeter_functional_driver_info}.\\ [2mm]

              \hline   Состояние водителя & Средняя панель & Отображение состояния: “Свободен” / “Занят”. & Описаны в \ref{taximeter_functional_driver_state} \\ [2mm]

              \hline   Кнопка “Старт” & Нижняя панель & Кнопка с названием “Старт”. & Реакция на нажатие этой кнопки описана в \ref{taximeter_functional_start_button}\\ [2mm]

              \hline
            \end{tabular}
          \end{center}
        \end{table}

        %%%Основные сведения о водителе - интерфейс ++++++++++ функционал
        \setlength{\extrarowheight}{2mm}
          \begin{longtable}{|p{5cm}|p{5cm}|p{5cm}|}
            
            \caption {Основные сведения о водителе} \label{driver_app_taximeter_tab_start_interface_table_driver_info} \\

              \hline  \textbf{Название}&\textbf{Способ отображения (водитель авторизован)}&\textbf{Способ отображения (водитель не авторизован)} \\ [2mm]
              \endfirsthead
              \hline  \textbf{Название}&\textbf{Способ отображения(водитель авторизован)}&\textbf{Способ отображения(водитель не авторизован)} \\ [2mm]
              \endhead

              \hline   Баланс водителя & [Баланс: ] + арендный счёт водителя & [Баланс: ] + [-] \\ [2mm]

              \hline   Позывной водителя & [Позывной: ] + позывной водителя & Не отображается \\ [2mm]

              \hline   Состояние авторизации водителя & Индикатор зеленого цвета + [В сети] & Индикатор красного цвета + [Не в сети] \\ [2mm] %%%Добавить ссылку на вкладку "Настройки", "Войти".

              \hline   Текущее местоположение водителя & Возможны два варианта: \begin{itemize} \item GPS включен: Индикатор зеленого цвета + GPS координаты текущего местоположения водителя \item GPS выключен: Индикатор красного цвета + [Нет координат] \end{itemize} & Индикатор красного цвета + [Нет координат] \\ [2mm]
              
              \hline   Время, затраченное водителем на текущий заказ. & Обнуленный счетчик времени в формате [ЧЧ:ММ:СС] & Обнуленный счетчик времени в формате [ЧЧ:ММ:СС] \\ [2mm]
              
              \hline   Километраж текущего заказа & Обнуленный счетчик километража в формате [0,00 км] & Обнуленный счетчик километража в формате [0,00 км] \\ [2mm]
              
              \hline

          \end{longtable}

      \paragraph{Интерфейс вкладки после нажатия кнопки “Старт”} \mbox{}\\ \label{driver_app_taximeter_tab_interface_start_button}
        DESC: Интерфейс вкладки после нажатия кнопки “Старт”. 
        Описание интерфейса вкладки после нажатия на кнопку “Старт” находится в таблице \ref{driver_app_taximeter_tab_start_button_interface_table}.\\

        %%%Интерфейс вкладки после нажатия кнопки “Старт”
        \begin{table}[h]
          \begin{center}
          \caption {Интерфейс вкладки “Таксометр” после нажатия кнопки “Старт”}
          \label{driver_app_taximeter_tab_start_button_interface_table}
          \setlength{\extrarowheight}{2mm}
            \begin{tabular}{|p{4cm}|p{3cm}|p{4cm}|p{4cm}|}
              \hline     \textbf{Название}&\textbf{Расположение на экране}&\textbf{Требования к интерфейсу}&\textbf{Функциональные требования} \\ [2mm]

              \hline   Основные сведения о водителе & Верхняя панель & Аналогичны \ref{driver_app_taximeter_tab_start_interface_table_driver_info}. & Описаны в \ref{taximeter_functional_driver_info_after_start_button}. \\ [2mm]

              \hline   Стоимость поездки & Средняя панель & Описаны в таблице \ref{driver_app_taximeter_tab_start_button_interface_drive_costs}. & Описаны в \ref{taximeter_functional_drive_costs}. \\ [2mm]%Появляется вместо кнопки “Состояния водителя”. \\ [2mm]

              \hline   Кнопка “Завершить” & Нижняя панель & Кнопка с названием “Завершить”. & Реакция на нажатие этой кнопки описана в \ref{taximeter_functional_end_button}. \\ [2mm]%Появляется вместо кнопки “Старт”. Нажатие на эту кнопку меняет интерфейс вкладки на такой, как описано в \ref{driver_app_taximeter_tab_interface_end_button}. \\ [2mm]

              \hline   Информация о текущем заказе & Левая скрытая боковая панель & Описаны в таблице \ref{driver_app_taximeter_tab_start_button_interface_curr_order_info} & Описаны в \ref{taximeter_functional_curr_order_info}. \\ [2mm]%Отображается информация о текущем заказе, описанная в таблице \ref{driver_app_taximeter_tab_start_button_interface_table_order_info}. \\ [2mm]

              \hline
            \end{tabular}
          \end{center}
        \end{table}

        %%%Стоимость поездки - таблицу
        \label{driver_app_taximeter_tab_start_button_interface_drive_costs}

        %%%Информация о текущем заказе - таблицу
        \label{driver_app_taximeter_tab_start_button_interface_curr_order_info}
      
        счетчик рубликов, кнопка завершить, боковая панель информации о текущем заказе

      \paragraph{Интерфейс вкладки после нажатия кнопки “Завершение”} \mbox{}\\ \label{driver_app_taximeter_tab_interface_end_button}
        DESC: Интерфейс вкладки после нажатия кнопки “Завершение”. 
        Описание интерфейса вкладки после нажатия на кнопку “Завершение” находится в таблице \ref{driver_app_taximeter_tab_end_button_interface_table}.\\

        Кнопка оплата и изменить

    \subsubsection{Функциональные требования к элементам}
      
      \paragraph{Начальный экран вкладки} \label{taximeter_functional_start_screen} 
        
        \subparagraph{Основные сведения о водителе} \label{taximeter_functional_driver_info}
          \begin{itemize}

            \item{
              TITLE: Баланс водителя \\
              \sr{Происходит запрос на сервер за арендным счётом водителя}. \\
              PRIOR: MODERATE \\
            }

            \item{
              TITLE: Позывной водителя \\
              \sr{Позывным водителя является логин, введенный им при авторизации}. \\
              PRIOR: MODERATE \\
            }

            \item{
              TITLE: Состояние авторизации водителя \\
              \sr{Водитель считается авторизованным, если он ввел корректные логин и пароль в форму “Войти” во вкладке “Настройки” и с сервера пришел положительный ответ на запрос об авторизации с этими данными}. \\
              PRIOR: MODERATE \\
            }

            \item{
              TITLE: Текущее местоположение водителя \\
              \sr{Отображаются GPS координаты текущего местоположения водителя}. \\
              PRIOR: MODERATE \\
            }
            
            \item{
              TITLE: Время, затраченное водителем на текущий заказ \\
              \sr{Отображается счетчик времени, затраченного на текущий заказ. Динамически увеличивается ежесекундно после нажатия кнопки “Старт”(Описание - \ref{driver_app_taximeter_tab_interface_start_button}), пока водитель не нажмет на кнопку “Завершение”(Описание - \ref{driver_app_taximeter_tab_interface_end_button}), после чего значение будет хранится, пока водитель не нажмет на кнопку “Оплата”(Описание - \ref{driver_app_taximeter_tab_interface_checkout_button})}. \\
              PRIOR: MODERATE \\
            }

            \item{
              TITLE: Километраж текущего заказа \\
              \sr{Счетчик километража текущего заказа. Динамически увеличивается на соответствующее проеханное водителем расстояние после нажатия кнопки “Старт”(Описание - \ref{driver_app_taximeter_tab_interface_start_button}), пока водитель не нажмет на кнопку “Завершение”(Описание - \ref{driver_app_taximeter_tab_interface_end_button}), после чего значение будет хранится, пока водитель не нажмет на кнопку “Оплата”(Описание - \ref{driver_app_taximeter_tab_interface_checkout_button})}. \\
              PRIOR: MODERATE \\
            }          
          \end{itemize}

        \subparagraph{Состояние водителя} \label{taximeter_functional_driver_state}
          \begin{itemize}

            \item{
              TITLE: Состояние: “Свободен” \\
              \sr{Отображается, когда у водителя нету активных заказов}. \\
              PRIOR: MODERATE \\
            }

            \item{
              TITLE: Состояние: “Занят” \\
              \sr{Отображается, когда у водителя есть активный заказ}. \\
              PRIOR: MODERATE \\
            }

          \end{itemize}

      \paragraph{Нажатие кнопки “Старт”}  \label{taximeter_functional_start_button}

        \subparagraph{Интерфейс вкладки “Таксометр”}
          Интерфейс вкладки “Таксометр” меняется на такой, как описано в \ref{driver_app_taximeter_tab_interface_start_button}.\\

        \subparagraph{Основные сведения о водителе после нажатия кнопки “Старт”} \label{taximeter_functional_driver_info_after_start_button}

        \subparagraph{Стоимость поездки} \label{taximeter_functional_drive_costs}

        \subparagraph{Информация о текущем заказе} \label{taximeter_functional_curr_order_info}

      \paragraph{Нажатие кнопки “Завершение”} \label{taximeter_functional_end_button}

      \paragraph{Нажатие кнопки “Изменить”} \label{taximeter_functional_change_button}

      \paragraph{Нажатие кнопки “Оплата”} \label{taximeter_functional_checkout_button}

    3.2.3 Таксометр
    3.2.3.1 Работа с текущим заказом
    3.2.3.2 Просчет стоимости

  \subsection{Заказы} \label{driver_app_orders_tab}

    экран Заказа

  \subsection{Навигатор} \label{driver_app_navigator_tab}

    DESC: В мобильное приложение встроен Яндекс.Навигатор.

    \subsubsection{Функциональные требования}

      \begin{itemize}
        \item{
          TITLE: Запуск навигатора.\\
          \sr{При нажатии на вкладку "Навигатор", на устройстве запускается приложение "Яндекс.Навигатор". Если приложение не установлено, то пользователю предлагается загрузить приложение на устройство из Play Market}\\
          PRIOR: MODERATE\\}

        \item{
          TITLE: Построение маршрута для заказа.\\
          \sr{При выполнении текущего заказа в навигатор передаются координаты в зависимости от стадии заказа. По этим координатам строится маршрут.}\\
          \sr{Если статус заказа "Едет к клиенту", то в навигатор передаются: [текущее местоположение водителя] + [начальная точка заказа].}\\
          \sr{Если статус заказа "В пути", то в навигатор передаются: [начальная точка заказа] + [конечная точка заказа].}\\
          PRIOR: MODERATE\\}
      \end{itemize}

  \subsection{Счёт} \label{driver_app_bill_tab}

    DESC: Во вкладке баланс водитель может посмотреть актуальное состояние фискального и арендного баланса, а так же пополнить их. 

      TITLE: Элементы вкладки\\
      \sr{Элементы этого экрана описаны в таблице \ref{driver_app_balance_tab_elements}.}\\
      PRIOR: MODERATE\\

      TITLE: Расположение элементов\\
      \sr{Экран делится на две части, при помощи вертикальной линии, в соотношении 6,5 : 3,5.}\\
      \sr{В левой части расположен элемент ELTAX-\ref{driver_element_transaction_area}}\\
      \sr{В правой части расположены элементы ELTAX-\ref{driver_element_fiscal_balance}, ELTAX-\ref{driver_element_rent_balance}, ELTAX-\ref{driver_element_ui_update_balance} в том же порядке в котором они преречислены.}\\
      PRIOR: MODERATE\\

      \begin{table}
        \begin{center}
        \caption{Элементы вкладки "Счет"}
        \label{driver_app_balance_tab_elements}
        \setlength{\extrarowheight}{2mm}
        \begin{tabular}{|p{3cm}|p{4cm}|p{8cm}|}
           \hline   \textbf{ID}&  \textbf{Название}&\textbf{Требования/Описание} \\ [2mm]


           \hline \eltax{driver_element_fiscal_balance}{} & Фискальный баланс & \sr{Элемент состоит из поля - [Фискальный баланс в рублях]}\\ [2mm]

           \hline \eltax{driver_element_rent_balance}{} & Арендный баланс & \sr{Элемент состоит из двух полей: [Баланс  в рублях], [Баланс конвертированный в дни]}\\ [2mm]

           \hline \eltax{driver_element_ui_update_balance}{} & Интерфейс выставления счета  & --- НАХОДИТСЯ В СТАДИИ РАЗРАБОТКИ ---\\ [2mm]

           \hline \eltax{driver_element_transaction_area}{} & Область транзакций  & --- НАХОДИТСЯ В СТАДИИ РАЗРАБОТКИ ---\\ [2mm]    

           \hline
        \end{tabular}
        \end{center}
      \end{table}   

  \subsection{Робот} \label{driver_app_robot_tab}

    DESC: При переходе на вкладку “Робот” отображается список настроек и список режимов и опций для каждого из них.

    \subsubsection{Общие требования} \mbox{}\\

      \begin{itemize}

        \item{
          TITLE: Отправка настроек и опций на сервер.\\
          \sr{При изменении опций или параметров, мобильное приложение автоматически отправляет обновленные настройки на сервер.}\\
          PRIOR: MODERATE\\
          }

        \item{
          TITLE: Сохранение настроек.\\
          \sr{Для каждого пользователя настройки сохраняются в базе СТ.}\\
          NOTE: При перезапуске или при разрыве соединения сервер возвращает приложению последние записанные в базу СТ настройки для конкретного пользователя.\\
          PRIOR: MODERATE\\
          }

      \end{itemize}

    \subsubsection{Интерфейс вкладки} \mbox{}\\

      \begin{itemize}

        \item{
          TITLE: Деление экрана.\\
          \sr{Экран делится на две части, при помощи вертикальной линии, в соотношении 1 : 1.}\\          
          PRIOR: MODERATE\\}

        \item {
          TITLE: Расположение общих настроек.\\
          \sr{В левой части расположен список общих настроек робота. (Описание - \ref{driver_app_robot_tab_general_settings})}\\
          PRIOR: MODERATE\\}

        \item {
          TITLE: Расположение настроек робота.\\
          \sr{В правой части расположен список режимов работы робота. (Описание - \ref{driver_app_robot_tab_robot_settings})}\\
          PRIOR: MODERATE\\}

      \end{itemize}

    \subsubsection{Общие настройки} \label{driver_app_robot_tab_general_settings} \mbox{}\\

      \begin{itemize}

        \item{
          \sr{В списке должны присутствовать настройки, перечисленные в таблице \ref{driver_app_robot_tab_table_general_settings}.}\\
          PRIOR: MODERATE\\}

      \end{itemize}

      %%%Таблица общих настроек робота.
      \begin{table}[h] 
        \begin{center}
        \caption {Общие настройки робота}
        \label{driver_app_robot_tab_table_general_settings}
        \setlength{\extrarowheight}{2mm}
        \begin{tabular}{|p{4cm}|p{3cm}|p{8cm}|}

          \hline     \textbf{Название}&\textbf{Способ изменения настройки}&\textbf{Описание} \\ [2mm]

          \hline   Заказы классом ниже & Переключатель ON/OFF & Водитель с машиной определенного класса может принимать заказы ниже по классу.\\ [2mm]

          \hline   Заказы от Яндекса & Переключатель ON/OFF & Возможность принимать заказы через канал Яндекс.Такси. \\ [2mm]
            
          \hline   Цепочка & Переключатель ON/OFF & Возможность закрепления за водителем заказов “в цепочку”. \\ [2mm]

          \hline   Дистанция поиска & Поле для ввода & Возможность задания водителем километража, в пределах которого сервер подбирает заказ. \\ [2mm]

          \hline   Время подачи & Поле для ввода & Возможность задания водителем максимального времени, за которое он хочет добраться до начальной точки заказа. \\ [2mm]

          \hline
        \end{tabular}
        \end{center}
      \end{table}

    \subsubsection{Роботы} \label{driver_app_robot_tab_robot_settings} \mbox{}\\

      \begin{itemize}

        \item{
          TITLE: Режимы работы роботов. \\
          \sr{В списке должны присутствовать режимы, перечисленные в таблице \ref{driver_app_robot_tab_modes}.}\\
          PRIOR: MODERATE\\}
      \end{itemize}

      %%%Таблица режимов работы робота.
      \begin{table}
          \begin{center}
          \caption {Режимы работы робота}
          \label{driver_app_robot_tab_modes}
          \setlength{\extrarowheight}{2mm}
          \begin{tabular}{|p{3cm}|p{3cm}|p{6cm}|p{3cm}|}

            \hline     \textbf{Название режима работы}&\textbf{Способ выбора режима работы}&\textbf{Список настроек}&\textbf{Описание настроек} \\ [2mm]

            \hline   Городской & Переключатель ON/OFF & Можно изменить следующие настройки: \begin{itemize} \item Округ. \end{itemize} & См. в таблице \ref{table:driver_app_robot_tab_table_town_mode} \\ [2mm]

            \hline   Портовый & Переключатель ON/OFF & Можно изменить следующие настройки: \begin{itemize} \item Порты.;  \item Встреча;  \item Проводы. \end{itemize} & См. в таблице \ref{table:driver_app_robot_tab_table_port_mode}  \\ [2mm]

            \hline
          \end{tabular}
          \end{center}
      \end{table}

      %%%Таблица описания настроек в городском режиме работы.
      \begin{table}
          \begin{center}
          \caption {Описание настроек в городском режиме работы}
          \label{table:driver_app_robot_tab_table_town_mode}
          \setlength{\extrarowheight}{2mm}
          \begin{tabular}{|p{4cm}|p{3cm}|p{8cm}|}

            \hline     \textbf{Название}&\textbf{Способ изменения}&\textbf{Описание} \\ [2mm]

            \hline   Округ & Выпадающий список & Водитель помечает чекбоксом округи в списке, таким образом исключая заказы с конечной точкой в выбранных округах. Доступен выбор из следующих округов: \begin{itemize} \item ЦАО \item САО \item ЮАО \item СЗАО \item ЮЗАО \item ВАО \item СВАО \item ЮВАО \item ЗАО \end{itemize} \\ [2mm]

            \hline
          \end{tabular}
          \end{center}
      \end{table}        

      %%%Таблица описания настроек в портовом режиме работы.
      \begin{table}
          \begin{center}
          \caption {Описание настроек в портовом режиме работы}
          \label{table:driver_app_robot_tab_table_port_mode}
          \setlength{\extrarowheight}{2mm}
          \begin{tabular}{|p{4cm}|p{3cm}|p{8cm}|}

            \hline     \textbf{Название}&\textbf{Способ изменения}&\textbf{Описание} \\ [2mm]

            \hline   Порты & Выпадающий список & Водитель помечает чекбоксом названия аэропортов. Доступен выбор из следующих аэропортов: \begin{itemize} \item Домодедово \item Шереметьево \item Внуково \end{itemize} \\ [2mm]

            \hline   Встреча & Checkbox & Опция, указывающая на то, что местом назначения для заказа этого типа будет один или несколько аэропортов, выбранных водителем. \\ [2mm]

            \hline   Проводы & Checkbox & Опция, указывающая на то, что местом прибытия для заказа этого типа является один или несколько аэропортов, выбранных водителем. \\ [2mm]

            \hline
          \end{tabular}
          \end{center}
      \end{table}

  \subsection{Настройки} \label{driver_app_settings_tab}

    \subsubsection{Уведомления}

      \paragraph{Общие требования} \mbox{}\\

      \subparagraph{Функциональные требования} \mbox{}\\      
        \begin{itemize}

          \item{TITLE: Новое уведомление. \\
                \sr{Новое уведомление помечается как непрочитанное, если пользователь не находится во вкладке "Уведомления" DriverApp. Инкрементируется счетчик непрочитанных уведомлений. Если на момент поступления нового уведомления пользователь находится во вкладке "Уведомления" DriverApp, новое уведомление помечается как прочитанное и счетчик непрочитанных уведомлений не инкрементируется.} \\
                PRIOR: MODERATE \\}

          \item{TITLE: Прочитанное уведомление. \\
                \sr{Уведомления считаются прочитанными, если была открыта вкладка "Уведомления" в DriverApp. Счетчик непрочитанных уведомлений при этом обнуляется.} \\
                PRIOR: MODERATE \\}

          \item{TITLE: Счетчик непрочитанных уведомлений. \\
                \sr{Означает количество непрочитанных уведомлений на данный момент. Инкрементируется, если появилось новое непрочитанное уведомление. Обнуляется при просмотре непрочитанных уведомлений через вкладку "Уведомления" в DriverApp.} \\
                PRIOR: MODERATE \\}
          
          \item{TITLE: Иконка. \\
                \sr{Выбирается в зависимости от инициатора уведомления из "Таблицы инициаторов уведомления".} \\
                PRIOR: MODERATE \\}

        \end{itemize}

      \paragraph{Уведомления в панели уведомлений Android(ПУА)} \mbox{}\\

        \subparagraph{Функциональные требования} \mbox{}\\

          \begin{itemize}

            \item{TITLE: Появление нового уведомления. \\
                  \sr{При появлении нового уведомления от какого-либо инициатора уведомления, оно выводится в панели уведомлений Android(ПУА). Если в данный момент появляется сразу несколько новых уведомлений, счетчик непрочитанных уведомлений инкрементируется на их количество.} \\
                  PRIOR: MODERATE \\}

            \item{TITLE: Открытие уведомления. \\
                  \sr{По нажатию на уведомление в ПУА, открывается DriverApp на вкладке "Уведомления". При этом, счетчик непрочитанных уведомлений обнуляется.} \\
                  PRIOR: MODERATE \\}

            \item{TITLE: Разрешенные действия с уведомлениями. \\
                  \sr{Разрешается нажать на уведомление в ПУА. В этом случае откроется DriverApp на вкладке "Уведомления", счетчик непрочитанных уведомлений обнулится.} \\ 
                  \sr{Разрешается убрать уведомление из ПУА путем скрола уведомления в сторону. В этом случае не произойдет никаких действий, счетчик непрочитанных уведомлений не обнулится.} \\
                  PRIOR: MODERATE \\}

            \item{TITLE: Интерфейс уведомления. \\
                  \sr{В панели уведомлений Android уведомления отображаются согласно \href{https://www.evernote.com/shard/s389/sh/8d4e738c-61f0-4500-98bb-5882d89c9906/9ab0dc75caa4a5144ac1e5042e808b4f}{прототипу}.} \\
                  PRIOR: MODERATE \\}

            \item{TITLE: Строка "Новых уведомлений"(верхняя строка). \\
                  \sr{Отображется: [Новых уведомлений: ] + [общее количество непрочитанных на данный момент уведомлений].} \\
                  PRIOR: MODERATE \\}

            \item{TITLE: Строка "Последних уведомлений"(нижняя строка). \\                          
                  \sr{Отображется: [последнее непрочитанное уведомление] + [,] + [предпоследнее непрочитанное уведомление] + [...] в конце, если непрочитанных сообщений больше двух.} \\
                  PRIOR: MODERATE \\}

            \item{TITLE: Дата. \\
                  \sr{Отображается время появления уведомления в формате "Часы:Минуты".} \\
                  PRIOR: MODERATE \\}

          \end{itemize}

      \paragraph{Уведомления в DriverApp во вкладке Уведомления} \mbox{}\\

        \subparagraph{Функциональные требования} \mbox{}\\

          \begin{itemize}

            \item{TITLE: Появление нового уведомления. \\
                  \sr{Непрочитанные уведомления появляются сверху списка во вкладке "Уведомления".} \\
                  PRIOR: MODERATE \\}

            \item{TITLE: Открытие вкладки "Уведомления". \\
                  \sr{При открытии вкладки "Уведомления", обнуляется счетчик непрочитанных уведомлений.} \\
                  PRIOR: MODERATE \\}

            \item{TITLE: Разрешенные действия с уведомлениями. \\
                  \sr{Уведомления во вкладке "Уведомления" разрешается только просмотреть.} \\
                  PRIOR: MODERATE \\}

            \item{TITLE: Интерфейс уведомления. \\
                  \sr{Во вкладке "Уведомления" уведомления отображаются согласно \href{https://www.evernote.com/shard/s389/sh/f7d9319d-bb3a-47de-b857-5a37e440419c/774bb30105ccb402d93522ae9e333b5e}{прототипу}.} \\
                  PRIOR: MODERATE \\}

            \item{TITLE: Строка "Уведомления"(верхняя строка). \\
                  \sr{Отображается содержание уведомления.} \\
                  PRIOR: MODERATE \\}

            \item{TITLE: Строка "Дата и время"(нижняя строка). \\
                  \sr{Отображается дата и время появления уведомления в формате "ДД.ММ.ГГГГ Часы:Минуты".} \\
                  PRIOR: MODERATE \\}

          \end{itemize}

      % Таблица уведомлений
      \label{taxometr_notifications_table}
      \setlength{\extrarowheight}{2mm}
          \begin{longtable}{|p{3cm}|p{4cm}|p{5cm}|p{3cm}|}
              \caption {Таблица уведомлений}\\

              \hline     \textbf{ID}&\textbf{Название}&\textbf{Содержимое уведомления} & \textbf{Инициатор}\\ [2mm]
              \endfirsthead
              \hline     \textbf{ID}&\textbf{Название}&\textbf{Содержимое уведомления} & \textbf{Инициатор}\\ [2mm]
              \endhead
              
              \hline  \nttax{notif_driver_of_remove_driver_from_the_order}{} & Снятие водителя с заказа. & Вы сняты с заказа + [№Заказа]. & SRVACT-\ref{act_remove_driver_driver_notification} \\ [2mm]

              \hline \nttax{notif_of_new_order_in_reserve}{} & Добавление нового заказа в резерв. & Заказ + [№Заказа] + поступил в резерв. & \\ [2mm]       %TODO: Дописать SRVACT-\ref{act_add_new_order_in_reserve}

              \hline
          \end{longtable}