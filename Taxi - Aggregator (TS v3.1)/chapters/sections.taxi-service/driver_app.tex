\section{Водительское приложение}
	%\includegraphics[scale=0.25, keepaspectratio]{frog.jpg}
    %\includegraphics[width=10cm]{frog.jpg}
    %\includegraphics[draft]{frog.jpg}

  Бар

  \subsection{Общие требования к мобильному приложению}

    TITLE: Отправка GPS данных на сервер\\
    \sr{Мобильное приложение оповещает сервер о местоположении водителя отсылая GPS координаты с периодичностью в STAT-\ref{periodicity_transfer_coordinates_taximeter_to_server} секунд.}\\
    PRIOR: MODERATE\\

  \subsection{Пользователи}

      \subsubsection{Роли, состояния приложения}

      		DESC: У приложения есть два состояния, они перечислены и описаны в таблице \ref{app_state}. 

             \begin{table}
             \begin{center}
             \label{app_state}
             \caption {Состояния приложений}
             \setlength{\extrarowheight}{2mm}
             \begin{tabular}{|p{5cm}|p{10cm}|}
                 \hline     \textbf{Состояние}&\textbf{Описание, уровень доступа} \\ [2mm]

                 \hline   Авторизованный пользователь & В этом состоянии пользователю доступны все элементы интерфейса и функциональные возможности приложения.\\ [2mm]
                 \hline   Автономный режим & В этом состоянии пользователю доступны следующие функции: \begin{itemize} \item Таксометр \item Навигатор \item \item\end{itemize}\\ [2mm]
                  \hline
             \end{tabular}
             \end{center}
             \end{table}

  	  \subsubsection{Авторизация}

  		DESC: Водитель имеет возможность ввести имя компании, свой позывной (логин) и пароль в специальном окне. Мобильное приложение отправляет эти данные на сервер для проверки и если они валидны то открывается соединение с сервером и все элементы интерфейса становятся доступны.

  		Требования: 
  		Если водитель не авторизован, то все элементы интерфейса кроме настроек, навигатора и таксометра становятся не доступны.
  		Вместе с позывным и паролем, при авторизации отправляется также GPS координаты - местоположение водителя.

  \subsection{Таксометр}


    3.2.3 Таксометр
    3.2.3.1 Работа с текущим заказом
    3.2.3.2 Просчет стоимости

  \subsection{Заказы}

    экран Заказа

  \subsection{Навигатор}

    DESC: В мобильное приложение встроен Яндекс.Навигатор.

    % Требования

      TITLE: Запуск навигатора.\\
      \sr{При нажатии на вкладку "Навигатор", на устройстве запускается приложение "Яндекс.Навигатор". Если приложение не установлено, то пользователю предлагается загрузить приложение на устройство из Play Market}\\
      PRIOR: MODERATE\\

      TITLE: Построение маршрута для заказа.\\
      \sr{При выполнении текущего заказа в навигатор передаются координаты в зависимости от стадии заказа. По этим координатам строится маршрут.}\\
      \sr{Если статус заказа "Едет к клиенту", то в навигатор передаются: [текущее местоположение водителя] + [начальная точка заказа].}\\
      \sr{Если статус заказа "В пути", то в навигатор передаются: [начальная точка заказа] + [конечная точка заказа].}\\
      PRIOR: MODERATE\\

  \subsection{Робот}



    3.2.7 Настройки робота
    3.2.7.2 Таблицы настроек режимов роботов
    3.2.7.2.1 Общие настройки
    3.2.7.2.2 Настройки городского режима
    3.2.7.2.3 Настройки портового режима
    3.2.7.2.4 Настройки режима поиска по месту назначения
    3.2.7.2.5 Настройки режима поиска по месту прибытия

  \subsection{Баланс}

    DESC: Во вкладке баланс водитель может посмотреть актуальное состояние фискального и арендного баланса, а так же пополнить их. 

      TITLE: Элементы вкладки\\
      \sr{Элементы этого экрана описаны в таблице \ref{driver_app_balance_tab_elements}.}\\
      PRIOR: MODERATE\\

      TITLE: Расположение элементов\\
      \sr{Экран делится на две части, при помощи вертикальной линии, в соотношении 6,5 : 3,5.}\\
      \sr{В левой части расположен элемент ELTAX-\ref{driver_element_transaction_area}}\\
      \sr{В правой части расположены элементы ELTAX-\ref{driver_element_fiscal_balance}, ELTAX-\ref{driver_element_rent_balance}, ELTAX-\ref{driver_element_ui_update_balance} в том же порядке в котором они преречислены.}\\
      PRIOR: MODERATE\\

      \begin{table}
        \begin{center}
        \caption{Элементы вкладки "Счет"}
        \label{driver_app_balance_tab_elements}
        \setlength{\extrarowheight}{2mm}
        \begin{tabular}{|p{3cm}|p{4cm}|p{8cm}|}
           \hline   \textbf{ID}&  \textbf{Название}&\textbf{Требования/Описание} \\ [2mm]


           \hline \eltax{driver_element_fiscal_balance}{} & Фискальный баланс & \sr{Элемент состоит из поля - [Фискальный баланс в рублях]}\\ [2mm]

           \hline \eltax{driver_element_rent_balance}{} & Арендный баланс & \sr{Элемент состоит из двух полей: \begin{itemize} \item [Баланс  в рублях] \item [Баланс конвертированный в дни] \end{itemize}}\\ [2mm]

           \hline \eltax{driver_element_ui_update_balance}{} & Интерфейс выставления счета  & --- НАХОДИТСЯ В СТАДИИ РАЗРАБОТКИ ---\\ [2mm]

           \hline \eltax{driver_element_transaction_area}{} & Область транзакций  & --- НАХОДИТСЯ В СТАДИИ РАЗРАБОТКИ ---\\ [2mm]    

           \hline
        \end{tabular}
        \end{center}
      \end{table}   

  \subsection{Настройки}

    \subsubsection{Уведомления}

      % Таблица уведомлений
      \label{taxometr_notifications_table}
      \setlength{\extrarowheight}{2mm}
          \begin{longtable}{|p{3cm}|p{4cm}|p{5cm}|p{3cm}|}
              \caption {Таблица уведомлений}\\

              \hline     \textbf{ID}&\textbf{Название}&\textbf{Содержимое уведомления} & \textbf{Инициатор}\\ [2mm]
              \endfirsthead
              \hline     \textbf{ID}&\textbf{Название}&\textbf{Содержимое уведомления} & \textbf{Инициатор}\\ [2mm]
              \endhead
              
              \hline  \nttax{notif_driver_of_remove_driver_from_the_order}{} & Снятие водителя с заказа. & Вы сняты с заказа + [№Заказа]. & SRVACT-\ref{act_remove_driver_driver_notification} \\ [2mm]

              \hline
          \end{longtable}

    3.2.9 Настройки мобильного клиента
    3.2.8.1 Основные настройки
    3.2.8.1.1 Звуковые уведомления
    3.2.8.2 История заказов
    3.2.1.1 Кнопка выхода