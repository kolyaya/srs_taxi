\section{Водительское приложение}
	%\includegraphics[scale=0.25, keepaspectratio]{frog.jpg}
    %\includegraphics[width=10cm]{frog.jpg}
    %\includegraphics[draft]{frog.jpg}

  Бар

  \subsection{Общие требования к мобильному приложению}

    TITLE: Отправка GPS данных на сервер\\
    \sr{Мобильное приложение оповещает сервер о местоположении водителя отсылая GPS координаты с периодичностью в STAT-\ref{periodicity_transfer_coordinates_taximeter_to_server} секунд.}\\
    PRIOR: MODERATE\\

  \subsection{Пользователи}

      \subsubsection{Роли, состояния приложения}

      		DESC: У приложения есть два состояния, они перечислены и описаны в таблице \ref{app_state}. 

             \begin{table}
             \begin{center}
             \label{app_state}
             \caption {Состояния приложений}
             \setlength{\extrarowheight}{2mm}
             \begin{tabular}{|p{5cm}|p{10cm}|}
                 \hline     \textbf{Состояние}&\textbf{Описание, уровень доступа} \\ [2mm]

                 \hline   Авторизованный пользователь & В этом состоянии пользователю доступны все элементы интерфейса и функциональные возможности приложения.\\ [2mm]
                 \hline   Автономный режим & В этом состоянии пользователю доступны следующие функции: \begin{itemize} \item Таксометр \item Навигатор \item \item\end{itemize}\\ [2mm]
                  \hline
             \end{tabular}
             \end{center}
             \end{table}

  	  \subsubsection{Авторизация}

  		DESC: Водитель имеет возможность ввести имя компании, свой позывной (логин) и пароль в специальном окне. Мобильное приложение отправляет эти данные на сервер для проверки и если они валидны то открывается соединение с сервером и все элементы интерфейса становятся доступны.

  		Требования: 
  		Если водитель не авторизован, то все элементы интерфейса кроме настроек, навигатора и таксометра становятся не доступны.
  		Вместе с позывным и паролем, при авторизации отправляется также GPS координаты - местоположение водителя.

  \subsection{Таксометр}


    3.2.3 Таксометр
    3.2.3.1 Работа с текущим заказом
    3.2.3.2 Просчет стоимости

  \subsection{Заказы}

    экран Заказа

  \subsection{Навигатор}

    DESC: В мобильное приложение встроен Яндекс.Навигатор.

    \subsubsection{Функциональные требования}

      \begin{itemize}
        \item 
        {TITLE: Запуск навигатора.\\
        \sr{При нажатии на вкладку "Навигатор", на устройстве запускается приложение "Яндекс.Навигатор". Если приложение не установлено, то пользователю предлагается загрузить приложение на устройство из Play Market}\\
        PRIOR: MODERATE\\}

        \item{
        TITLE: Построение маршрута для заказа.\\
        \sr{При выполнении текущего заказа в навигатор передаются координаты в зависимости от стадии заказа. По этим координатам строится маршрут.}\\
        \sr{Если статус заказа "Едет к клиенту", то в навигатор передаются: [текущее местоположение водителя] + [начальная точка заказа].}\\
        \sr{Если статус заказа "В пути", то в навигатор передаются: [начальная точка заказа] + [конечная точка заказа].}\\
        PRIOR: MODERATE\\}
      \end{itemize}

  \subsection{Робот}

    DESC: При переходе на вкладку “Робот” отображается список настроек и список режимов и опций для каждого из них. Для того чтобы запустить одного или несколько режимов робота необходимо поставить галочки напротив выбранных режимов. Если водитель поставит галочку напротив одного из режимов, то опции этого режима станут активными. Опции выбранного режима автоматически применяются к запущенному роботу по их изменению. Если водитель больше не хочет автоматически получать заказы, он должен снять галочки со всех режимов роботов. Таким образом робот будет выключен.

      TITLE: Включение робота.\\
      \sr{Робот автоматически включается после выбора настроек и выбора режима работы, путем установки галочки либо изменения положения свитчера в положение "ON".}\\
      PRIOR: MODERATE\\

      TITLE: Выключение робота.\\
      \sr{Робот автоматически выключается после отмены выбора настроек и выбора режима работы, путем снятия галочки либо изменения положения свитчера в положение "OFF".}\\
      PRIOR: MODERATE\\

      TITLE: Отправка настроек и опций на сервер.\\
      \sr{По изменению любой настройки робота или любой из опций любого из режимов работы робота на сервер отправляются данные об изменении конкретной опции или настройки.}\\
      PRIOR: MODERATE\\

      TITLE: Общие настройки.\\
      \sr{В списке должны присутствовать режимы перечисленные в "Таблице общих настроек".}\\
      PRIOR: MODERATE\\

      TITLE: Режимы работы.\\
      \sr{В списке должны присутствовать режимы перечисленные в "Таблице режимов работы робота".}\\
      PRIOR: MODERATE\\

      TITLE: Ограничения настроек робота.\\
      \sr{Ограничения настроек робота хранятся и настраиваются на сервере.}\\
      PRIOR: MODERATE\\    

      TITLE: Возможность взять заказ при включенном роботе.\\
      \sr{При включенном роботе водитель может взять заказ.}\\
      PRIOR: MODERATE\\

      TITLE: Сохранение настроек при переходе со вкладки во вкладку.\\
      \sr{При переходе на вкладку “Робот” с включенным роботом на экране мобильного приложения отображаются текущие настройки робота выставленные водителем.}\\
      PRIOR: MODERATE\\

      TITLE: Интерфейс вкладки.\\
      \sr{Экран делится на две части, при помощи вертикальной линии, в соотношении 1 : 1.}\\
      \sr{В левой части расположен список общих настроек робота.}\\
      \sr{В правой части расположен список режимов работы робота.}\\
      PRIOR: MODERATE\\

    3.2.7 Настройки робота
    3.2.7.2 Таблицы настроек режимов роботов
    3.2.7.2.1 Общие настройки
    3.2.7.2.2 Настройки городского режима
    3.2.7.2.3 Настройки портового режима
    3.2.7.2.4 Настройки режима поиска по месту назначения
    3.2.7.2.5 Настройки режима поиска по месту прибытия

  \subsection{Баланс}

    DESC: Во вкладке баланс водитель может посмотреть актуальное состояние фискального и арендного баланса, а так же пополнить их. 

      TITLE: Элементы вкладки\\
      \sr{Элементы этого экрана описаны в таблице \ref{driver_app_balance_tab_elements}.}\\
      PRIOR: MODERATE\\

      TITLE: Расположение элементов\\
      \sr{Экран делится на две части, при помощи вертикальной линии, в соотношении 6,5 : 3,5.}\\
      \sr{В левой части расположен элемент ELTAX-\ref{driver_element_transaction_area}}\\
      \sr{В правой части расположены элементы ELTAX-\ref{driver_element_fiscal_balance}, ELTAX-\ref{driver_element_rent_balance}, ELTAX-\ref{driver_element_ui_update_balance} в том же порядке в котором они преречислены.}\\
      PRIOR: MODERATE\\

      \begin{table}
        \begin{center}
        \caption{Элементы вкладки "Счет"}
        \label{driver_app_balance_tab_elements}
        \setlength{\extrarowheight}{2mm}
        \begin{tabular}{|p{3cm}|p{4cm}|p{8cm}|}
           \hline   \textbf{ID}&  \textbf{Название}&\textbf{Требования/Описание} \\ [2mm]


           \hline \eltax{driver_element_fiscal_balance}{} & Фискальный баланс & \sr{Элемент состоит из поля - [Фискальный баланс в рублях]}\\ [2mm]

           \hline \eltax{driver_element_rent_balance}{} & Арендный баланс & \sr{Элемент состоит из двух полей: [Баланс  в рублях], [Баланс конвертированный в дни]}\\ [2mm]

           \hline \eltax{driver_element_ui_update_balance}{} & Интерфейс выставления счета  & --- НАХОДИТСЯ В СТАДИИ РАЗРАБОТКИ ---\\ [2mm]

           \hline \eltax{driver_element_transaction_area}{} & Область транзакций  & --- НАХОДИТСЯ В СТАДИИ РАЗРАБОТКИ ---\\ [2mm]    

           \hline
        \end{tabular}
        \end{center}
      \end{table}   

  \subsection{Настройки}

    \subsubsection{Уведомления}

      \paragraph{Общие требования} \mbox{}\\

      \subparagraph{Функциональные требования} \mbox{}\\      
        \begin{itemize}
          
          \item{TITLE: Появление нового уведомления. \\
                \sr{При появлении нового уведомления от какого-либо инициатора уведомления, оно выводится в панели уведомлений Android(ПУА). При этом, если непрочитанных уведомлений несколько, их количество заносится в счетчик непрочитанных сообщений и отображается в строке "Новых уведомлений" уведомления в ПУА.} \\
                PRIOR: MODERATE \\}

          \item{TITLE: Интерфейс уведомления. \\
                \sr{Уведомления должны отображаться на русском языке.} \\
                PRIOR: MODERATE \\}

          \item{TITLE: Иконка. \\
                \sr{Выбирается в зависимости от инициатора уведомления.} \\
                PRIOR: MODERATE \\}

        \end{itemize}

      \paragraph{Уведомления в панели уведомлений Android(ПУА)} \mbox{}\\

        \subparagraph{Функциональные требования} \mbox{}\\

          \begin{itemize}

            \item{TITLE: Открытие уведомления. \\
                  \sr{По нажатию на уведомление в ПУА, открывается DriverApp на вкладке "Уведомления". При этом, счетчик непрочитанных сообщений обнуляется.} \\
                  PRIOR: MODERATE \\}

            \item{TITLE: Разрешенные действия с уведомлениями. \\
                  \sr{Разрешается нажать на уведомление в ПУА. В этом случае откроется DriverApp на вкладке "Уведомления", счетчик непрочитанных сообщений обнулится. Также, разрешается убрать уведомление из ПУА путем скрола уведомления в сторону. В этом случае не произойдет никаких действий, счетчик непрочитанных уведомлений не обнулится.} \\
                  PRIOR: MODERATE \\}

            \item{TITLE: Интерфейс уведомления. \\
                  \sr{В панели уведомлений Android уведомления отображаются согласно \href{https://www.evernote.com/shard/s389/sh/8d4e738c-61f0-4500-98bb-5882d89c9906/9ab0dc75caa4a5144ac1e5042e808b4f}{прототипу}.} \\
                  PRIOR: MODERATE \\}

            \item{TITLE: Строка "Новых уведомлений"(верхняя строка). \\
                  \sr{Отображется: [Новых уведомлений: ] + [общее количество непрочитанных на данный момент уведомлений].} \\
                  PRIOR: MODERATE \\}

            \item{TITLE: Строка "Последних уведомлений"(нижняя строка). \\                          
                  \sr{Отображется: [последнее непрочитанное уведомление] + [,] + [предпоследнее непрочитанное уведомление] + [...] в конце, если непрочитанных сообщений больше двух.} \\
                  PRIOR: MODERATE \\}

            \item{TITLE: Дата. \\
                  \sr{Отображается время появления уведомления в формате "Часы:Минуты".} \\
                  PRIOR: MODERATE \\}

          \end{itemize}

      \paragraph{Уведомления в DriveApp во вкладке Уведомления} \mbox{}\\

        \subparagraph{Функциональные требования} \mbox{}\\

          \begin{itemize}

            \item{TITLE: Открытие вкладки "Уведомления". \\
                  \sr{При открытии вкладки "Уведомления", обнуляется счетчик непрочитанных уведомлений.} \\
                  PRIOR: MODERATE \\}

            \item{TITLE: Разрешенные действия с уведомлениями. \\
                  \sr{Уведомления во вкладке "Уведомленния" разрешается только просмотреть.} \\
                  PRIOR: MODERATE \\}

            \item{TITLE: Интерфейс уведомления. \\
                  \sr{Во вкладке "Уведомления" уведомления отображаются согласно \href{https://www.evernote.com/shard/s389/sh/f7d9319d-bb3a-47de-b857-5a37e440419c/774bb30105ccb402d93522ae9e333b5e}{прототипу}.} \\
                  PRIOR: MODERATE \\}

            \item{TITLE: Строка "Уведомления"(верхняя строка). \\
                  \sr{Отображается содержание уведомления.} \\
                  PRIOR: MODERATE \\}

            \item{TITLE: Строка "Дата и время"(нижняя строка). \\
                  \sr{Отображается дата и время появления уведомления в формате "ДД.ММ.ГГГГ Часы:Минуты".} \\
                  PRIOR: MODERATE \\}

          \end{itemize}

      % Таблица уведомлений
      \label{taxometr_notifications_table}
      \setlength{\extrarowheight}{2mm}
          \begin{longtable}{|p{3cm}|p{4cm}|p{5cm}|p{3cm}|}
              \caption {Таблица уведомлений}\\

              \hline     \textbf{ID}&\textbf{Название}&\textbf{Содержимое уведомления} & \textbf{Инициатор}\\ [2mm]
              \endfirsthead
              \hline     \textbf{ID}&\textbf{Название}&\textbf{Содержимое уведомления} & \textbf{Инициатор}\\ [2mm]
              \endhead
              
              \hline  \nttax{notif_driver_of_remove_driver_from_the_order}{} & Снятие водителя с заказа. & Вы сняты с заказа + [№Заказа]. & SRVACT-\ref{act_remove_driver_driver_notification} \\ [2mm]

              \hline \nttax{notif_of_new_order_in_reserve}{} & Добавление нового заказа в резерв. & Заказ + [№Заказа] + поступил в резерв. & \\ [2mm]       %TODO: Дописать SRVACT-\ref{act_add_new_order_in_reserve}

              \hline
          \end{longtable}

    3.2.9 Настройки мобильного клиента
    3.2.8.1 Основные настройки
    3.2.8.1.1 Звуковые уведомления
    3.2.8.2 История заказов
    3.2.1.1 Кнопка выхода