\subsection{Заказы}

  \subsubsection{Прием заказа} \mbox{} \\ \label{}

    \begin{itemize}

      \item{

        TITLE: Источники заказов.\\
        \sr{Служба Такси принимает и обрабатывает заказы из источников перечисленных в таблице \ref{ts_order_issue}.}\\
        PRIOR: CRITICAL\\

      }

    \end{itemize}
    
   
    \begin{table}[h]
      \begin{center}
      \caption {Источники заказов}
      \label{ts_order_issue}
      \setlength{\extrarowheight}{2mm}
      \begin{tabular}{|p{5cm}|p{10cm}|}
        \hline     \textbf{Название источника}&\textbf{Описание} \\ [2mm]

        \hline  Агрегатор & 
        
          \sr{Служба Такси принимает заказы от "Агрегатора"(Раздел - \ref{aggregator})  и обрабатывает их. (Раздел - \ref{selection_drivers_for_the_order})} 

          \sr{Служба Такси может принимать заказы от "Агрегатора" с уже закрепленным водителем СТ. В этом случае СТ уведомляет водителя и закрепляет за ним заказ.}

          \\ [2mm]

         \hline  Диспетчерская СТ & 

          \sr{Служба Такси принимает заказы от "Диспетчерской"(Раздел - \ref{dispatching}) и обрабатывает их. (Раздел - \ref{selection_drivers_for_the_order})} 

          \\ [2mm]

         \hline  Мобильное приложение (пассажирское) СТ  & -ЭТА ЧАСТЬ НАХОДИТСЯ В РАЗРАБОТКЕ- \\ [2mm]

         \hline  Веб-сайт СТ & -ЭТА ЧАСТЬ НАХОДИТСЯ В РАЗРАБОТКЕ- \\ [2mm]
         \hline
      \end{tabular}
      \end{center}
    \end{table}

  \subsubsection{Расписание} \mbox{} \\ \label{}

  \subsubsection{Сервисы заказа} \mbox{} \\ \label{}

    \paragraph{Входные/Выходные данные сервисов} \mbox{} \\ \label{}

      \begin{table} 
         \begin{center}
         \caption {Входные/Выходные данные сервисов}
         \label{}
         \setlength{\extrarowheight}{2mm}
         \begin{tabular}{|p{3cm}|p{3cm}|p{9cm}|}
             \hline \textbf{ID} & \textbf{Данные}&\textbf{Описание/Требования} \\ [2mm]

            \hline \crdt{crdt_order}{}    & Заказ & \\ [2mm]
            \hline \crdt{crdt_driver_id}{}    & ID водителя & \\ [2mm]
            \hline \crdt{crdt_drivers_list}{}    & Список водителей & \\ [2mm]
            \hline \crdt{crdt_radius}{}    & Радиус & \\ [2mm]
            \hline \crdt{}{}    & Ответ - [Не удалось закрепить] & \\ [2mm]
            \hline \crdt{}{}    & Ответ - [Заказ закреплен]  & \\ [2mm]
            \hline \crdt{crdt_drivers_dont_accept_the_order}{}    & Сообщение-ответ - "drivers don't accept the order" & \\ [2mm]

             \hline
         \end{tabular}
         \end{center}
      \end{table}

    \paragraph{Сервис фильтрации} \mbox{} \\ \label{driver_filters_taxi_service}

        \subparagraph{Входные данные} \mbox{} \\ \label{driver_filters_taxi_service_input_data}

          \begin{itemize}

            \item{

              TITLE: Входные данные.\\
              \sr{В качестве входных данных сервис принимает данные перечисленные в списке ниже.}\\
              PRIOR: MODERATE\\

            }

            \begin{itemize}
              \item Заказ (CRDT-\ref{crdt_order})
              \item Радиус (CRDT-\ref{crdt_radius})
            \end{itemize}

          \end{itemize}

        \subparagraph{Выходные данные} \mbox{} \\

          \begin{itemize}

            \item{

              TITLE: Выходные данные.\\
              \sr{В качестве выходных данных сервис возвращает список водителей (CRDT-\ref{crdt_drivers_list}).}\\
              PRIOR: MODERATE\\

            }

          \end{itemize}

        \subparagraph{Процесс фильтрации} \mbox{} \\

          \begin{itemize}

            \item {
              TITLE: Начало и процесс выполнения фильтрации.\\
              \sr{Фильтры расположены в порядке убывания значимости, соответственно выполняются они в том порядке в котором они описаны в таблице \ref{filters_table}.}\\
              PRIOR: MODERATE\\
            }


            \item {
              TITLE: Конец фильтрации.\\
              \sr{Процесс фильтрации заканчивается после выполнения последнего фильтра.}\\
              PRIOR: MODERATE\\
            }

            \item {
              TITLE: Входные/Выходные данные фильтров.\\
              \sr{На выходных данных каждого фильтра мы получаем список водителей.}\\
              \sr{На входных данных каждого фильтра мы получаем [Cписок водителей] + [Данные о заказе - \ref{driver_filters_taxi_service_input_data}].}\\
              \sr{[Cписок водителей] фильтр получает от фильтра выполненного ранее. FLT-\ref{flt_free_drivers} исключение и принимает на вход только [Данные о заказе - \ref{driver_filters_taxi_service_input_data}].}
              PRIOR: MODERATE\\
            }

          \end{itemize}
            

          
            
          
          \label{filters_table}
          \setlength{\extrarowheight}{2mm}
          \begin{longtable}{|p{2cm}|p{3cm}|p{10cm}|}


          \hline  \textbf{ID}  & \textbf{Название фильтра} & \textbf{Требования} \\ [2mm]
          \endfirsthead
          \hline  \textbf{ID}  & \textbf{Название фильтра} & \textbf{Требования} \\ [2mm]
          \endhead



          \hline  \flt{flt_free_drivers}{}  & Фильтр доступных водителей. & \sr{Сервис выбирает из базы данных Службы Такси водителей со статусом "В сети".} \\ [2mm]

          \hline  \flt{flt_diver_user_settings}{}  & Фильтр по пользовательским настройкам водителя. & 

            \sr{Сервер проверяет заказ на совместимость с пользовательскими настройками водителя. (Описание настроек в разделе \ref{server_driver_user_settings}) В этом случае сервер выбирает только тех водителей, чьи настройки совместимы с типом и параметрами заказа.}

            \\ [2mm]

          \hline  \flt{}{}  & Фильтр по классу заказа. & 

            \sr{Сервер проверяет водителей на соответствие класса их ТС с заказом. Выбирает водителей чьи транспортные средства соответствуют классу заказа.}

            \sr{Водитель может брать заказ классом ниже чем ТС водителя, если он установил соответствующие настройки. В этом случае водитель проходит проверку. (Описание настроек в разделе \ref{server_driver_user_settings})}

            \\ [2mm]

          \hline  \flt{}{}  & Фильтр по доп.опциям. & 

           \sr{Сервер проверяет ТС водителей на соответствие параметров с доп. опциями заказа. Выбирает только тех водителей чьи ТС соответствуют доп. опциям заказа.}

            \\ [2mm]

          \hline  \flt{}{}  & Фильтр по расстоянию. & 

            \sr{Сервер выбирает водителей чьи координаты находятся в пределах радиуса [radius] км. от начальной точки заказа.}

            \\ [2mm]

          \hline  \flt{}{}  & Фильтр по времени. & 

            \sr{Сервер выбирает водителей которые свободны во время подачи заказа.}

            \sr{Для каждого из этих водителей сервер с помощью Яндекс.Пробок вычисляет расчетное время подачи машины.}
              \begin{itemize}
                \item В случае если у водителя статус "Свободен", то мы вычисляем РВП от текущей координаты водителя до начальной точки заказа.
                \item Если водитель “Свободен” но у него есть предварительный заказ, то сервер выполняет FLT-\ref{flt_extra_filter}.
                \item В случае если у водителя есть активный заказа и включен робот с настройкой "цепочка", то мы вычисляем РВП от последней (“ближайшей” к времени подачи) записанной в расписании координаты до начальной точки заказа.
                \item Ко всем вышеперечисленным пунктам добавляется “Страховочное время”.
              \end{itemize}

            \sr{Водители, у которых расчетное время подачи машины превышает время, оставшееся до заявленного клиентом, исключаются из дальнейшего рассмотрения.}

            \\ [2mm]

          \hline  \flt{flt_extra_filter}{} (Extra) & Фильтр по РВП для предварительного заказа. (Опционально) &  

            \sr{Если водитель “Свободен” но у него есть предварительный заказ, то сервер выполняет действия описанные в списке ниже.}

              \begin{itemize}
                \item Вычисляем время выполнение поступившего заказа.
                \item Вычисляем время между заказами (предварительный и поступивший) - [a]
                  \begin{itemize}
                    \item Если значение отрицательное, то водитель не проходит фильтр.
                  \end{itemize}
                \item Затем вычисляем РВП от конечной точки поступившего заказа до начальной точки предварительного заказа - [b]
                \item Сравниваем [a] и [b]:
                  \begin{itemize}
                    \item Если [a] > [b], то водитель проходит фильтрацию.
                    \item Если [a] <= [b], то водитель не проходит фильтрацию.
                  \end{itemize}
              \end{itemize}

            \sr{Из дальнейшего рассмотрения исключаются водители которые не прошли фильтр FLT-\ref{flt_extra_filter}.}

            \\ [2mm]

          \hline

          \caption {Фильтры}
          \end{longtable}

    \paragraph{Сервис предложения заказа} \mbox{} \\ \label{}

      \subparagraph{Входные данные} \mbox{} \\ \label{driver_filters_taxi_service_input_data}

        \begin{itemize}

          \item{

            TITLE: Входные данные.\\
            \sr{В качестве входных данных сервис принимает данные перечисленные в списке ниже.}\\
            PRIOR: MODERATE\\

          }

          \begin{itemize}
            \item Заказ (CRDT-\ref{crdt_order})
            \item Список водителей (CRDT-\ref{crdt_drivers_list})
          \end{itemize}

        \end{itemize}

      \subparagraph{Выходные данные} \mbox{} \\

        \begin{itemize}

          \item{

            TITLE: Выходные данные.\\
            \sr{В качестве выходных данных сервис возвращает один вариант из данных перечисленных в списке ниже.}\\
            PRIOR: MODERATE\\

          }

          \begin{itemize}
            \item ID водителя (CRDT-\ref{crdt_driver_id})
            \item "drivers don't accept the order" (CRDT-\ref{crdt_drivers_dont_accept_the_order})
          \end{itemize}

        \end{itemize}

      \subparagraph{Процесс предложения заказа} \mbox{} \\

        \begin{itemize}

          \item {
            TITLE: Процесс предложения заказа.\\
            \sr{Процесс предложения заказа описан в ALG-\ref{alg_order_offering}.}\\
            \sr{Операции в процессе произодяться над списком водителей((CRDT-\ref{crdt_drivers_list})).}\\
            PRIOR: MODERATE\\
          }

        \end{itemize}

        \begin{alg}[Процесс предложения заказа водителям.] \label{alg_order_offering} \mbox{}

          \begin{enumerate}

            \item Сервис сортирует список по РВП. - SRVACT-\ref{srvact_sort_drivers_by_rvp}.

            \item Сервис предлагает заказ водителям. - SRVACT-\ref{srvact_offer_order_with_delay}.
            
            \item Если водитель присылает запрос на закрепление заказа или у водитель в списке помечен как "Робот", то сервис выполняет SRVACT-\ref{srvact_send_driver_id}. 

            \item Если водители по списку закончились и ни один из водителей не прислал запрос на закрепление заказа, то сервис выполняет SRVACT-\ref{srvact_send_drivers_dont_accept_the_order}, после таймаута - STAT-\ref{stat_timeout_after_order_offering}.

          \end{enumerate}

        \end{alg}

          \label{filters_table}
          \setlength{\extrarowheight}{2mm}
          \begin{longtable}{|p{2cm}|p{3cm}|p{10cm}|}


            \hline  \textbf{ID}  & \textbf{Действие сервиса} & \textbf{Требования} \\ [2mm]
            \endfirsthead
            \hline  \textbf{ID}  & \textbf{Действие сервиса} & \textbf{Требования} \\ [2mm]
            \endhead



            \hline  \srvact{srvact_sort_drivers_by_rvp}{}  & Сортировка водителей по РВП & \sr{} \\ [2mm]

            \hline  \srvact{srvact_offer_order_with_delay}{}  & Предложение с задержкой & \sr{} \\ [2mm]

            \hline  \srvact{srvact_send_driver_id}{}  & Отправление ID водителя & \sr{} \\ [2mm]

            \hline  \srvact{srvact_send_drivers_dont_accept_the_order}{}  & Отправление "drivers don't accept the order" & \sr{} \\ [2mm]



            \hline

            \caption {Фильтры}
          \end{longtable}

    \paragraph{Сервис закрепления заказа} \mbox{} \\ \label{}

  \subsubsection{Обработка заказа} \label{selection_drivers_for_the_order}

    DESC:  При поступлении заказа по одному из источников сервер формирует очередь из водителей и обрабатывает ее, в результате заказ будет взят на выполнение одним из них. 

    \paragraph{Обработка срочных заказов} \mbox{} \\ \label{}

      

    \paragraph{Обработка предварительных заказов} \mbox{} \\ \label{}

    \paragraph{Обработка портовых заказов} \mbox{} \\ \label{}
          
  \subsubsection{Обработка снятия водителя с заказа} \label{remove_driver_from_order}

    \paragraph{Функциональные требования} \mbox{}\\

      TITLE: Действия при поступлении сообщения.
     	\sr{При поступлении сообщения о снятии водителя с заказа сервер должен выполнять действия описанные в ALG - \ref{remove_driver_from_order_alg}.}\\
      PRIOR: CRITICAL\\

    \paragraph{Не функциональные требования} \mbox{}\\

      \sr{Действия в ALG - \ref{remove_driver_from_order_alg} описанны в таблице \ref{remove_driver_from_order_actions_table}.}\\
      PRIOR: CRITICAL\\

    % Алгоритм обработки снятия водителя с заказа

    	\begin{alg} [Алгоритм обработки снятия водителя с заказа] \label{remove_driver_from_order_alg} \mbox{}\\

        При поступлении запроса на открепление водителя от заказа, сервер выполняет следующий алгоритм:

        \begin{enumerate}
          \item Выполняет SRVACT-\ref{act_undocking_driver_from_order}.
          \item Выполняет SRVACT-\ref{act_remove_driver_driver_notification} и SRVACT-\ref{act_remove_driver_dispatcher_notification}.
          \item Если причиной снятия водителя с заказа была не отмена заказа, то заказ заново выходит в раздачу (\ref{selection_drivers_for_the_order}).
        \end{enumerate}
      \end{alg}

    % Таблица действий сервера при снятии водителя с заказа
    	\begin{table} [h]
         \begin{center}
         \caption {Действия сервера при снятии водителя с заказа.}
         \label{remove_driver_from_order_actions_table}
         \setlength{\extrarowheight}{2mm}
         \begin{tabular}{|p{3cm}|p{3cm}|p{9cm}|}
             \hline \textbf{ID} & \textbf{Название}&\textbf{Действие сервера} \\ [2mm]

             \hline \srvact{act_undocking_driver_from_order}{} & Открепление водителя. & Сервер открепляет водителя от заказа. \\ [2mm]
             \hline \srvact{act_remove_driver_driver_notification}{} & Уведомление водителя о снятии.  & Сервер посылает водителю сообщение о снятии его с заказа.\\ [2mm]
             \hline \srvact{act_remove_driver_dispatcher_notification}{} & Уведомление диспетчера о снятии. & Сервер посылает диспетчеру сообщение о снятии водителя с заказа. \\ [2mm]

             \hline
         \end{tabular}
         \end{center}
      \end{table}

  \subsubsection{Обработка отмены заказа}

      \paragraph{Функциональные требования} \mbox{}\\

        TITLE: Действия при поступлении сообщения.
      	\sr{При поступлении сообщения об отмене заказа сервер должен выполнять действия описанные в ALG - \ref{cancel_order_alg}.}\\
      	PRIOR: CRITICAL\\

      \paragraph{Не функциональные требования} \mbox{}\\

      	\sr{Действия в ALG - \ref{cancel_order_alg} описанны в таблице \ref{cancel_order_actions_table}.}\\
      	PRIOR: CRITICAL\\

    	\begin{alg}[Алгоритм обработки отмены заказа]\label{cancel_order_alg} \mbox{}\\

    		\begin{enumerate}
    			\item Если на заказ не назначен водитель, то сервер:
    			\begin{enumerate}
    				\item Выполняет SRVACT-\ref{act_order_cancel_status}.
    				\item Выполняет SRVACT-\ref{act_cancel_order_distribution}.
    			\end{enumerate}
    			\item Если на заказ назначен водитель, то сервер:
    			\begin{enumerate}
    				\item Выполняет SRVACT-\ref{act_order_cancel_status}.
    				\item Выполняет ALG - \ref{remove_driver_from_order_alg}.
    			\end{enumerate}
    		\end{enumerate}

    	\end{alg}

    	\begin{table} [h]
           \begin{center}
           \caption {Действия сервера при отмене заказа.}
           \label{cancel_order_actions_table}
           \setlength{\extrarowheight}{2mm}
           \begin{tabular}{|p{3cm}|p{3cm}|p{9cm}|}
               \hline \textbf{ID} & \textbf{Название}&\textbf{Действие сервера} \\ [2mm]

               \hline \srvact{act_order_cancel_status}{} & Присвоение статуса "Отмена" & Сервер присваивает заказу статус "Отменен". \\ [2mm]
               \hline \srvact{act_cancel_order_distribution}{} & Рассылка водителям (Отмена)  & Сервер делает рассылку сообщений водителям об отмене заказа.\\ [2mm]

               \hline
           \end{tabular}
           \end{center}
        \end{table}

  \subsubsection{Обработка изменения данных заказа} \mbox{}\\ \label{change_order_processor} 

    \paragraph{Функциональные требования} \mbox{}\\ 

      TITLE: Изменение полей заказа\\
      \sr{При изменении критичных полей заказа сервер выполняет ALG - \ref{edit_order_alg}.}\\ 
      PRIOR: MODERATE\\

    \paragraph{Не функциональные требования} \mbox{}\\

      \sr{Действия в ALG - \ref{edit_order_alg} описанны в таблице \ref{edit_order_actions_table}.}\\
      PRIOR: CRITICAL\\

      \sr{Критичными полями являются следующие поля заказа:
        \begin{itemize}
          \item Время
          \item Адрес подачи/прибытия
          \item Доп. услуги
          \item Способ оплаты
          \item Тариф
        \end{itemize}\mbox{}}\\ 
      PRIOR: CRITICAL\\   


      \begin{alg}[Алгоритм обработки изменения данных заказа]\label{edit_order_alg} \mbox{}\\
        \begin{longenum}
          \item Если на заказ не назначен водитель, то заказ заново выходит в раздачу (\ref{selection_drivers_for_the_order}).
          \item Если на заказ назначен водитель, то сервер:
          \begin{longenum}
            \item Выполняет SRVACT-\ref{act_one_driver_filter} для закрепленного за заказом водителя.
              \begin{longenum}
                \item Если водитель проходит фильтрацию, то сервер:
                  \begin{longenum}
                   \item Выполняет SRVACT-\ref{act_offer_driver_updated_order}.
                     \begin{longenum}
                      \item Если ответ положительный, то водитель продолжает выполнять заказ.
                      \item Если ответ отрицательный, то сервер выполняет ALG - \ref{remove_driver_from_order_alg}.
                      \item Если ответ не пришел в течении STAT-\ref{timeout_waiting_for_a_response_from_the_driver_request}, то сервер выполняет ALG - \ref{remove_driver_from_order_alg}.
                     \end{longenum}
                  \end{longenum}
                \item Если водитель не проходит фильтрацию, то сервер выполняет ALG - \ref{remove_driver_from_order_alg}. 
              \end{longenum}
          \end{longenum}
        \end{longenum}
      \end{alg}

      \begin{table} [h]
           \begin{center}
           \caption {Действия сервера при изменении данных заказа.}
           \label{edit_order_actions_table}
           \setlength{\extrarowheight}{2mm}
           \begin{tabular}{|p{3cm}|p{3cm}|p{9cm}|}
               \hline \textbf{ID} & \textbf{Название}&\textbf{Действие сервера} \\ [2mm]

               \hline \srvact{act_one_driver_filter}{} & Фильтрация для одного водителя. & Сервер выполняет фильтрацию (\ref{driver_filters_taxi_service}) для конкретного водителя. \\ [2mm]

               \hline \srvact{act_offer_driver_updated_order}{} & Предлагаем водителю обновленный заказ. & Сервер отправляет уведомление водителю, о том что заказ изменился. Вместе с запросом сервер передает водителю информацию об изменении.\\ [2mm] 

               \hline
           \end{tabular}
           \end{center}
        \end{table}