\subsubsection{Сервис уведомлений} \label{notification_service}

	\paragraph{Нефункциональные требования} \mbox{} \\ \label{}

      \begin{itemize}

        \item{

          TITLE: Ограничение по ответственности работы сервиса.\\
          \sr{Ответственность за целесообразность или соответствие уведомления с реальным событием несут инициирующие (вызывающие сервис уведомлений) уведомления сервисы.}\\
          PRIOR: MODERATE\\

        }

        \end{itemize}

    \paragraph{Возможные получатели уведомлений} \mbox{} \\ \label{}

    	\begin{table}[h]
			\begin{center}
			\label{table_potential_recipiest}
			\caption {Возможные получатели уведомлений}
			\setlength{\extrarowheight}{2mm}
			\begin{tabular}{|p{5cm}|p{10cm}|}

			\hline     \textbf{Получатель}&\textbf{Описание} \\ [2mm]

			\hline Водитель    & Описание в разделе \ref{essence_driver}\\ [2mm]
			\hline Пассажир    & Описание в разделе \ref{essence_passanger}\\ [2mm]
			\hline Диспетчер    & --В РАЗРАБОТКЕ--\\ [2mm]
			\hline

			\end{tabular}
			\end{center}
			\end{table}

	\paragraph{Входные данные} \mbox{} \\ \label{}

      \begin{itemize}

        \item{

          TITLE: Входные данные.\\
          \sr{В качестве входных данных сервис принимает данные перечисленные в списке ниже.}\\
          PRIOR: MODERATE\\

        }

        \begin{itemize}
          \item ID заказа (CRDT-\ref{crdt_order_id}) \&\& ID водителя (CRDT-\ref{crdt_driver_id})
          \item Тип уведомления (CRDT-\ref{crdt_notification_type}) \&\& Доп. параметры (Статус, сумма и прочее)
        \end{itemize}

      \end{itemize}

    \paragraph{Выходные данные} \mbox{} \\

      \begin{itemize}

        \item{

          TITLE: Выходные данные.\\
          \sr{В качестве выходных данных сервис возвращает вызвавшему его сервису положительный или отрицательный ответ, что значит успешное или неуспешное выполнение работы сервиса соответственно.}\\
          PRIOR: MODERATE\\

        }

      \end{itemize}

    \paragraph{Процесс сервиса} \mbox{} \\

        \begin{itemize}

             \item {
               TITLE: Процесс сервиса.\\
               \sr{Процесс сервиса описан в ALG-\ref{alg_notofocation_service_processing}.}\\
               PRIOR: MODERATE\\
             }

             \end{itemize}

        \begin{alg}[Процесс сервиса.] \label{alg_notofocation_service_processing} \mbox{}

             \begin{enumerate}

               \item Сервис определяет получателей уведомления. - SRVACT-\ref{srvact_defenition_of_recipiest}

               \item Сервис производит рассылку. - SRVACT-\ref{srvact_send_notifications}

             \end{enumerate}

             \end{alg}

        \label{}
        \setlength{\extrarowheight}{2mm}
        \begin{longtable}{|p{2cm}|p{3cm}|p{10cm}|}


          \hline  \textbf{ID}  & \textbf{Действие сервиса} & \textbf{Требования} \\ [2mm]
          \endfirsthead
          \hline  \textbf{ID}  & \textbf{Действие сервиса} & \textbf{Требования} \\ [2mm]
          \endhead



          \hline  \srvact{srvact_defenition_of_recipiest}{} & Определиние получателей. 
          	& 
          	\sr{Определение получателей выполняется в два этапа. Описание в списке ниже.}
			\begin{itemize}
				\item Исходя из типа уведомления, второго параметра во входных данных сервиса, сервис определяет возможных получателей.
				\item Исходя из первого параметра во входных данных сервиса, сервис формирует список получателей. (Пример - При смене статуса заказа сервис проверяет есть ли у заказа "Пассажир" (клиент), и если есть, то включает его в рассылку.)
			\end{itemize} 
			\\ [2mm]

          \hline  \srvact{srvact_send_notifications}{}  & Расылка уведомлений. 
          &
            \sr{Сервис рассылает уведомления избранным получателям.}


			\sr{Уведомления содержат сообщения индивидуальный для каждого типа получателя. Описание текстов сообщений находятся в таблице \ref{table_notificetion_types}.}
          \\ [2mm]


          \hline

          \caption {Действия сервиса}
        \end{longtable}