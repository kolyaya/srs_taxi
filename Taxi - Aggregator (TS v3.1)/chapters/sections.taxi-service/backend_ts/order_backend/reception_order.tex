\subsubsection{Прием заказа} \mbox{} \\ \label{service_reception_order}

    DESC: Сервис приема заказа обрабатывает полученные от различных источников запросы на создание заказа в Службе Такси. 

    \paragraph{Входные данные} \mbox{} \\ \label{}

        \begin{itemize}

          \item{

            TITLE: Входные данные.\\
            \sr{В качестве входных данных сервис принимает заказы из источников перечисленных в таблице \ref{ts_order_issue}.}\\
            PRIOR: MODERATE\\

          }

        \end{itemize}

    \paragraph{Процесс сервиса} \mbox{} \\

        \begin{itemize}

         \item {
           TITLE: Процесс сервиса.\\
           \sr{Процесс сервиса описан в ALG-\ref{alg_reception_order_service}.}\\
           PRIOR: MODERATE\\
         }

         \end{itemize}

        \begin{alg}[Процесс сервиса.] \label{alg_reception_order_service} \mbox{}

             \begin{enumerate}

               \item Сервис проверяет полученные от источников заказа данные на валидность. - SRVACT-\ref{srvact_checking_data_from_issue_on_validity}

                  \begin{enumerate}
               
                    \item Если данные валидны, то происходит переход к следующему шагу процедуры.
               
                    \item Если данные не валидны, то сервис возвращает ошибку источнику заказа. - SRVACT-\ref{srvact_error_data_response_to_issue_order_creation}
               
                  \end{enumerate}
               
               \item Сервис создает заказ в системе. - SRVACT-\ref{srvact_create_new_order}
              
               \item Сервис присваивает заказу статус "Новый". - SRVACT-\ref{srvact_change_order_status_on_new}

               \item Сервис уведомляет сервис-координатора заказов об успешном приеме заказа в систему. - SRVACT-\ref{srvact_notify_order_processing_service_about_success_order_recept}

             \end{enumerate}

             \end{alg}

        \label{}
        \setlength{\extrarowheight}{2mm}
        \begin{longtable}{|p{2cm}|p{3cm}|p{10cm}|}


          \hline  \textbf{ID}  & \textbf{Действие сервиса} & \textbf{Требования} \\ [2mm]
          \endfirsthead
          \hline  \textbf{ID}  & \textbf{Действие сервиса} & \textbf{Требования} \\ [2mm]
          \endhead



          \hline  \srvact{srvact_checking_data_from_issue_on_validity}{}  & Проверка данных от источника на валидность. 
                               & \sr{Проверка выполняется по условиям перечисленным в списке ниже.} 
                                    \begin{itemize}
                                      \item {В присланных данных должны присутствовать обязательныей для создания заказа поля. (Описание в таблице \ref{table_order_essence})}
                                      \item {Обязательные для создания заказа поля должны быть в валидных форматах. (Описание в таблице \ref{table_order_essence})}
                                    \end{itemize}
                               \\ [2mm]

          \hline  \srvact{srvact_error_data_response_to_issue_order_creation}{}  & Возврат ошибки источнику при невалидных данных. & \sr{Сервис возвращает ошибку источнику заказа.} \\ [2mm]

          \hline  \srvact{srvact_create_new_order}{}  & Создание заказа в системе Службы Такси. 
            & \sr{В рамках создания заказа сервис формирует поля на основе правил описанных в разделе \ref{order_essence}}. \\ [2mm]

          \hline  \srvact{srvact_change_order_status_on_new}{}  & Присвоение заказу статус "Новый". & \sr{Вызов микросервиса обработки статусов заказа (Раздел - \ref{service_order_status_processing}) с передачей параметров ([ID заказа] + [Статус "Новый"]).} \\ [2mm]

          \hline  \srvact{srvact_notify_order_processing_service_about_success_order_recept}{}  & Уведомление сервиса-координатора заказов. & \sr{Сервис оповещает сервис-координатор заказов о том что заказ принят в систему и готов для последующей обработки.} \\ [2mm]



          \hline

          \caption {Действия сервиса приема заказа.}
        \end{longtable}

   
    \begin{table}
      \begin{center}
      \caption {Источники заказов}
      \label{ts_order_issue}
      \setlength{\extrarowheight}{2mm}
      \begin{tabular}{|p{5cm}|p{10cm}|}
        \hline     \textbf{Название источника}&\textbf{Описание} \\ [2mm]

        \hline  Агрегатор & 
        
          \sr{Служба Такси принимает заказы от "Агрегатора"(Раздел - \ref{aggregator})  и обрабатывает их. (Раздел - \ref{selection_drivers_for_the_order})} 

          \sr{Служба Такси может принимать заказы от "Агрегатора" с уже закрепленным водителем СТ. В этом случае СТ уведомляет водителя и закрепляет за ним заказ.}

          \\ [2mm]

         \hline  Диспетчерская СТ & 

          \sr{Служба Такси принимает заказы от "Диспетчерской"(Раздел - \ref{dispatching}) и обрабатывает их. (Раздел - \ref{selection_drivers_for_the_order})} 

          \\ [2mm]

         \hline  Мобильное приложение (пассажирское) СТ  & -ЭТА ЧАСТЬ НАХОДИТСЯ В РАЗРАБОТКЕ- \\ [2mm]

         \hline  Веб-сайт СТ & -ЭТА ЧАСТЬ НАХОДИТСЯ В РАЗРАБОТКЕ- \\ [2mm]
         \hline
      \end{tabular}
      \end{center}
    \end{table}