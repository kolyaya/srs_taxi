\paragraph{Обработчик статусов заказов} \mbox{} \\ \label{service_order_status_processing}

	\subparagraph{Входные данные} \mbox{} \\ \label{}

      \begin{itemize}

        \item{

          TITLE: Входные данные.\\
          \sr{В качестве входных данных сервис принимает данные перечисленные в списке ниже.}\\
          PRIOR: MODERATE\\

        }

        \begin{itemize}
          \item ID заказа
          \item Статус изменения
        \end{itemize}

      \end{itemize}

    \subparagraph{Выходные данные} \mbox{} \\

      \begin{itemize}

        \item{

          TITLE: Выходные данные.\\
          \sr{В качестве выходных данных сервис возвращает один из ответов перечисленных в списке ниже.}\\
          PRIOR: MODERATE\\

        }

        \begin{itemize}
          \item Ответ - [Order status changed successfully]
          \item Ответ - [Error order status change]
        \end{itemize}

      \end{itemize}

    \subparagraph{Процесс сервиса} \mbox{} \\

        \begin{itemize}

             \item {
               TITLE: Процесс сервиса.\\
               \sr{Процесс сервиса описан в ALG-\ref{alg_order_status_processing}.}\\
               PRIOR: MODERATE\\
             }

             \end{itemize}

        \begin{alg}[Процесс сервиса обработки статуса заказов.] \label{alg_order_status_processing} \mbox{}

             \begin{enumerate}

               \item Сервис проверяет порядок изменения статуса. - SRVACT-\ref{srvact_check_the_order_status_changes}

	               	\begin{enumerate}

		               	\item Если проверка прошла успешно, то выполняем п.2
	              
	               		\item Если проверка выявила ошибку, то сервис отправляет сообщение - SRVACT-\ref{srvact_send_response_about_error_status_change}

	               	\end{enumerate}

               \item Сервис устанавливает заказу запрашиваемый статус. - SRVACT-\ref{srvact_set_new_status}

	               	\begin{enumerate}

		               	\item Если операция завершилась успешно, то сервис отправляет ответ об успешном изменении статуса. - SRVACT-\ref{srvact_send_response_about_success_status_change}
	              
	               		\item Если при выполнении операции возникла ошибка, то сервис отправляет сообщение - SRVACT-\ref{srvact_send_response_about_error_status_change}

	               	\end{enumerate}

             \end{enumerate}

             \end{alg}

        \label{}
        \setlength{\extrarowheight}{2mm}
        \begin{longtable}{|p{2cm}|p{3cm}|p{10cm}|}


          \hline  \textbf{ID}  & \textbf{Действие сервиса} & \textbf{Требования} \\ [2mm]
          \endfirsthead
          \hline  \textbf{ID}  & \textbf{Действие сервиса} & \textbf{Требования} \\ [2mm]
          \endhead



          \hline  \srvact{srvact_check_the_order_status_changes}{}  & Проверка порядка изменения статуса. & \sr{} \\ [2mm]

          \hline  \srvact{srvact_set_new_status}{}  & Установка нового статуса заказу. & \sr{} \\ [2mm]

          \hline  \srvact{srvact_send_response_about_success_status_change}{}  & Отправка ответа - [Order status changed successfully] & \sr{} \\ [2mm]

          \hline  \srvact{srvact_send_response_about_error_status_change}{}  & Отправка ответа - [Error order status change] & \sr{} \\ [2mm]

          \hline

          \caption {Действия сервиса обработки статусов заказа.}
        \end{longtable}

    \subparagraph{Статусы заказа} \mbox{} \\

    	TITLE: Статусы заказа\\
		\sr{Все возможные статусы заказа описаны в таблице \ref{table_order_status}.}\\
		PRIOR: MODERATE\\



    	\begin{table}
			\begin{center}
			\caption {Статусы заказа.}
			\label{table_order_status}
			\setlength{\extrarowheight}{2mm}
			\begin{tabular}{|p{3cm}|p{9cm}|p{3cm}|}

			\hline     \textbf{Название} & \textbf{Описание} & \textbf{Порядковый номер в последовательности статусов}\\ [2mm]

			\hline \textit{\textbf{Основные статусы:}}  &  & \\ [2mm]

			\hline Новый  

				& У заказа статус “Новый” если на заказ не назначен водитель. 
				& 1 \\ [2mm]

			\hline Назначен водитель  
				& На заказ назначен водитель.  
				& 2 \\ [2mm]

			\hline Едет к клиенту 
				& Водитель приступил к выполнению заказа и находиться в пути к клиенту. 
				& 3 \\ [2mm]

			\hline На месте  
				& Водитель прибыл на место назначения, ожидает клиента. 
				& 4\\ [2mm]

			\hline В пути  
				& Водитель везет клиента к месту назначения. 
				& 5\\ [2mm]

			\hline Завершен  
				& Заказ выполнен. 
				& 6\\ [2mm]

			\hline Отменен  &  & - \\ [2mm]

			\hline \textit{\textbf{Опциональные статусы:}}  &  & \\ [2mm]

			\hline Ожидание  & Водитель ожидает клиента. & В паре со статусом "На месте".\\ [2mm]
			\hline Опоздание  & Водитель опаздывает на заказ. & В паре со статусом "Едет к клиенту".\\ [2mm]
			\hline
			\end{tabular}
			\end{center}
		\end{table}
