\subsubsection{Сервис-координатор заказа} \label{selection_drivers_for_the_order}

    DESC:  При поступлении заказа по одному из источников сервер формирует очередь из водителей и обрабатывает ее, в результате заказ будет взят на выполнение одним из них. 

    \paragraph{Входные данные} \mbox{} \\ \label{}

	  \begin{itemize}

	    \item{

	      TITLE: Входные данные - Прием заказа\\
	      \sr{В качестве входных данных сервис принимает данные перечисленные в списке ниже.}\\
	      PRIOR: MODERATE\\

	    }

		    \begin{itemize}
		      \item Уведомление от сервиса приема заказа (Раздел-\ref{service_reception_order}) в котором содержиться [ID заказа] готового для обработки.
		    \end{itemize}

	    \item{

	      TITLE: Входные данные - Другие сервисы\\
	      \sr{В этом случае сервис принимает разного рода запросы (Пример: Отмена заказа), во всех обязательно должен присутствовать [ID заказа] на который направлен запрос.}\\
	      PRIOR: MODERATE\\

	    }

	  \end{itemize}

	\paragraph{Действия сервиса-координатора заказа} \mbox{} \\ \label{}

	  \label{}
	  \setlength{\extrarowheight}{2mm}
	  \begin{longtable}{|p{2cm}|p{3cm}|p{10cm}|}


	    \hline  \textbf{ID}  & \textbf{Действие сервиса} & \textbf{Требования} \\ [2mm]
	    \endfirsthead
	    \hline  \textbf{ID}  & \textbf{Действие сервиса} & \textbf{Требования} \\ [2mm]
	    \endhead


	    % Фильтрация
	    \hline  \srvact{srvact_call_filter_service}{}  
		    &  Фильтрация 
		    & \sr{Вызов микросервиса фильтрации. (Раздел - \ref{driver_filters_taxi_service}) В качестве аргументов передает:

         		\begin{itemize}
		            \item ID заказа
		            \item Радиус (Таблица \ref{table_of_searching_driver_radius})
		            \item Общая (по умолч.)/Индивидуальная(для одного водителя) - опциональный параметр.
	            \end{itemize} } 
	        \\ [2mm]

	    % Предложение заказа
	    \hline  \srvact{srvact_call_order_offering_service}{}  
	    	& Предложение заказа 
	    	& \sr{Вызов микросервиса предложения заказов. (Раздел - \ref{order_offering_service}) В качестве аргументов передает:

         		\begin{enumerate}
         			\item ID заказа
         			\item Список водителей
         			\item Общая (по умолч.)/Индивидуальная(для одного водителя) - опциональный параметр.
	         	\end{enumerate}} 
	    	\\ [2mm]

	    % Закрепление заказа
	    \hline  \srvact{srvact_call_fasten_order_service}{}  
	    	& Закрепление / Открепление заказа 
    		& \sr{Вызов микросервиса закрепления заказов. (Раздел - \ref{fasten_order_service}) В качестве аргументов передает: 

       			\begin{enumerate}
         			\item ID заказа
 					\item ID водителя
 					\item Открепить/Закрепить
         		\end{enumerate}} 
    		\\ [2mm]

	    % Присвоение нового статуса заказу.
	    \hline  \srvact{srvact_change_order_status_from_order_processing_service}  
	    	& Присвоение заказу нового статуса. 
	    	& \sr{Вызов микросервиса обработки статусов заказа (Раздел - \ref{service_order_status_processing}). В качестве аргументов передает: 

                \begin{enumerate}
                  \item ID заказа 
                  \item Статус
                \end{enumerate} 
				} 
			\\ [2mm]

		% Завершение заказа
	    \hline  \srvact{srvact_call_finish_order_service}{}  
	    	& Завершение заказа 
	    	& \sr{Вызов микросервиса завершения заказов. (Раздел - \ref{finish_order}) В качестве аргументов передает ID заказа.} 
	    	\\ [2mm]
        

        % Рассылка водителям сообщений о недоспуности заказа для выолнения.
        \hline \srvact{act_order_is_not_available_messege_distribution}{} 
        	& Рассылка водителям сообщений о недоспуности заказа для выолнения. 

	        & 
	        DESC: Это действие необходимо для того чтоб в водителском клиенте не отображался уже недоступный для выполнения заказ, при этом причина его недоступности (отмена или закрепление за другим водителем) не важна.

	        \sr{Сервер делает рассылку сообщений водителям о недоспуности заказа для выолнения.}
	        \\[2mm]

	    \hline

	    \caption {Действия сервиса-координатора заказа}
	  \end{longtable}


	\paragraph{Процессы обработки заказов} \mbox{} \\
    \subparagraph{Процесс обработки срочных заказов} \mbox{} \\ \label{}

  \begin{itemize}

       \item {
         TITLE: Процесс обработки срочного заказа.\\
         \sr{Процесс описан в ALG-\ref{alg_urgent_order_processing}.}\\
         PRIOR: MODERATE\\
       }

       \end{itemize}

  \begin{alg}[Процесс обработки срочного заказа.] \label{alg_urgent_order_processing} \mbox{}

      \begin{enumerate}

         	\item Процесс вызывает сервис фильтрации - SRVACT-\ref{srvact_call_filter_service}. В качестве аргументов передает:

         		\begin{enumerate}
         			\item ID заказа 
         			\item Радиус поиска 
         		\end{enumerate}

          \item Сервис фильтрации возвращает [Список водителей. Процесс вызывает сервис предложения заказов - SRVACT-\ref{srvact_call_order_offering_service}. В качестве аргументов передает:

         		\begin{enumerate}
         			\item ID заказа
         			\item Список водителей, полученный от сервиса фильтрации.)
         		\end{enumerate}
        
          \item Процесс дожидается ответа от сервиса предложения заказов, и на основании его выполняет соответствующие действия.
          
          	\begin{enumerate}
         			\item Если процесс получает положительный ответ + ID водителя, то процесс вызывает сервис закрепления заказов - SRVACT-\ref{srvact_call_fasten_order_service}, затем рассылает сообщения о недоступности заказа водителям которым был предложен заказ. -  SRVACT-\ref{act_order_is_not_available_messege_distribution} и переходит к следующему пункту процесса. В качестве аргументов сервису закрепления заказов передает: 

           			\begin{enumerate}
             			\item ID заказа 
     					    \item ID водителя, полеченный от сервиса предложения заказов.
             		\end{enumerate}

         			\item Если процесс получает ответ "drivers don't accept the order", то процесс ALG-\ref{alg_urgent_order_processing} начинается заново. В качестве аргументов передает: 

           			\begin{enumerate}
             			\item ID заказа
             			\item Инкрементированный (по таблице радиусов \ref{table_of_searching_driver_radius}) Радиус поиска
           			\end{enumerate}
           			
         		\end{enumerate}

          \item После закрепления заказа за водителем, процесс переходит в режим контроля выполнения заказа. В рамках этого режима процесс выполняет следующие действия.

            \begin{enumerate}
              \item Процесс обрабатывает входящие запросы от водителей на смену статусов с порядковыми номерами 3-5 (Таблица \ref{table_order_status}). При каждом поступлении такого запроса сервис присваивает заказу новый статус. - SRVACT-\ref{srvact_change_order_status_from_order_processing_service}. В качестве аргументов передает: 

                \begin{enumerate}
                  \item ID заказа 
                  \item Статус
                \end{enumerate} 

              \item После того как заказу был присвоен статус "В пути" (порядковый номером 5), процесс ожидает от водителя запрос на смену статуса на "Завершен". При поступлении такого запроса процесс переходит к следующему шагу.
            \end{enumerate}

          \item После поступления запроса на смену статуса на "Завершен", процесс завершает заказ по средствам вызова сервиса завершения заказов. - SRVACT-\ref{srvact_call_finish_order_service}. В качестве аргументов передает ID заказа.

          \item После выполнения последнего шага (получения положительного ответа от сервиса завершения заказа) процесс завершается.

      	\end{enumerate}

      \end{alg}
 
    \subparagraph{Процесс обработки предварительных заказов} \mbox{} \\ \label{}
    \paragraph{Обработка портовых заказов} \mbox{} \\ \label{}
    \subparagraph{События-исключения} \label{exception_events} \mbox{}\\

	DESC: В любом момент времени выполнения процесса обработки заказа в СТ может произойти событие-исключение, которое прервет основной процесс.

	\begin{itemize}
		\item {
			TITLE: Реакция на событие-исключение.\\
			\sr{При возникновении события основной процесс обработки заказа прерывается, и запускается процесс соответствующий событию-исключению.}\\
			PRIOR: MODERATE\\
		}
	\end{itemize}

    \subparagraph{Обработка отмены заказа}

  \begin{itemize}

    \item {
      TITLE: Процесс обработки отмены заказа.\\
      \sr{Процесс описан в ALG - \ref{cancel_order_alg}.}\\
      PRIOR: CRITICAL\\
    }

  \end{itemize}

  \begin{alg}[Алгоритм обработки отмены заказа]\label{cancel_order_alg} \mbox{}\\

    \begin{enumerate}

      \item Сервис присваивает заказу статус "Отменен" - SRVACT-\ref{act_order_cancel_status}.

      \item Если на заказ назначен водитель, то сервис снимает водителя с заказа. - Раздел \ref{remove_driver_from_order}.

    \end{enumerate}

  \end{alg} 
    \subsubsection{Обработка изменения данных заказа} \mbox{}\\ \label{change_order_processor} 

    \paragraph{Функциональные требования} \mbox{}\\ 

      TITLE: Изменение полей заказа\\
      \sr{При изменении критичных полей заказа сервер выполняет ALG - \ref{edit_order_alg}.}\\ 
      PRIOR: MODERATE\\

    \paragraph{Не функциональные требования} \mbox{}\\

      \sr{Действия в ALG - \ref{edit_order_alg} описанны в таблице \ref{edit_order_actions_table}.}\\
      PRIOR: CRITICAL\\

      \sr{Критичными полями являются следующие поля заказа:
        \begin{itemize}
          \item Время
          \item Адрес подачи/прибытия
          \item Доп. услуги
          \item Способ оплаты
          \item Тариф
        \end{itemize}\mbox{}}\\ 
      PRIOR: CRITICAL\\   


      \begin{alg}[Алгоритм обработки изменения данных заказа]\label{edit_order_alg} \mbox{}\\
        \begin{longenum}
          \item Если на заказ не назначен водитель, то заказ заново выходит в раздачу (\ref{selection_drivers_for_the_order}).
          \item Если на заказ назначен водитель, то сервер:
          \begin{longenum}
            \item Выполняет SRVACT-\ref{act_one_driver_filter} для закрепленного за заказом водителя.
              \begin{longenum}
                \item Если водитель проходит фильтрацию, то сервер:
                  \begin{longenum}
                   \item Выполняет SRVACT-\ref{act_offer_driver_updated_order}.
                     \begin{longenum}
                      \item Если ответ положительный, то водитель продолжает выполнять заказ.
                      \item Если ответ отрицательный, то сервер выполняет ALG - \ref{remove_driver_from_order_alg}.
                      \item Если ответ не пришел в течении STAT-\ref{timeout_waiting_for_a_response_from_the_driver_request}, то сервер выполняет ALG - \ref{remove_driver_from_order_alg}.
                     \end{longenum}
                  \end{longenum}
                \item Если водитель не проходит фильтрацию, то сервер выполняет ALG - \ref{remove_driver_from_order_alg}. 
              \end{longenum}
          \end{longenum}
        \end{longenum}
      \end{alg}

      \begin{table} [h]
           \begin{center}
           \caption {Действия сервера при изменении данных заказа.}
           \label{edit_order_actions_table}
           \setlength{\extrarowheight}{2mm}
           \begin{tabular}{|p{3cm}|p{3cm}|p{9cm}|}
               \hline \textbf{ID} & \textbf{Название}&\textbf{Действие сервера} \\ [2mm]

               \hline \srvact{act_one_driver_filter}{} & Фильтрация для одного водителя. & Сервер выполняет фильтрацию (\ref{driver_filters_taxi_service}) для конкретного водителя. \\ [2mm]

               \hline \srvact{act_offer_driver_updated_order}{} & Предлагаем водителю обновленный заказ. & Сервер отправляет уведомление водителю, о том что заказ изменился. Вместе с запросом сервер передает водителю информацию об изменении.\\ [2mm] 

               \hline
           \end{tabular}
           \end{center}
        \end{table}
    \subsubsection{Обработка снятия водителя с заказа} \label{remove_driver_from_order}

    \paragraph{Функциональные требования} \mbox{}\\

      TITLE: Действия при поступлении сообщения.
     	\sr{При поступлении сообщения о снятии водителя с заказа сервер должен выполнять действия описанные в ALG - \ref{remove_driver_from_order_alg}.}\\
      PRIOR: CRITICAL\\

    \paragraph{Не функциональные требования} \mbox{}\\

      \sr{Действия в ALG - \ref{remove_driver_from_order_alg} описанны в таблице \ref{remove_driver_from_order_actions_table}.}\\
      PRIOR: CRITICAL\\

    % Алгоритм обработки снятия водителя с заказа

    	\begin{alg} [Алгоритм обработки снятия водителя с заказа] \label{remove_driver_from_order_alg} \mbox{}\\

        При поступлении запроса на открепление водителя от заказа, сервер выполняет следующий алгоритм:

        \begin{enumerate}
          \item Выполняет SRVACT-\ref{act_undocking_driver_from_order}.
          \item Выполняет SRVACT-\ref{act_remove_driver_driver_notification} и SRVACT-\ref{act_remove_driver_dispatcher_notification}.
          \item Если причиной снятия водителя с заказа была не отмена заказа, то заказ заново выходит в раздачу (\ref{selection_drivers_for_the_order}).
        \end{enumerate}
      \end{alg}

    % Таблица действий сервера при снятии водителя с заказа
    	\begin{table} [h]
         \begin{center}
         \caption {Действия сервера при снятии водителя с заказа.}
         \label{remove_driver_from_order_actions_table}
         \setlength{\extrarowheight}{2mm}
         \begin{tabular}{|p{3cm}|p{3cm}|p{9cm}|}
             \hline \textbf{ID} & \textbf{Название}&\textbf{Действие сервера} \\ [2mm]

             \hline \srvact{act_undocking_driver_from_order}{} & Открепление водителя. & Сервер открепляет водителя от заказа. \\ [2mm]
             \hline \srvact{act_remove_driver_driver_notification}{} & Уведомление водителя о снятии.  & Сервер посылает водителю сообщение о снятии его с заказа.\\ [2mm]
             \hline \srvact{act_remove_driver_dispatcher_notification}{} & Уведомление диспетчера о снятии. & Сервер посылает диспетчеру сообщение о снятии водителя с заказа. \\ [2mm]

             \hline
         \end{tabular}
         \end{center}
      \end{table}


    

    