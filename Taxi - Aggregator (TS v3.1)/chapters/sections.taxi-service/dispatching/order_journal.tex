\subsection{Журнал заказов}

		DESC: Журнал содержит таблицу, которая отображает заказы Службы Такси.

		\subsubsection{Операционная панель} \mbox{}\\

				\sr{Присутствует кнопка “Новый заказ” при нажатии на которую происходит переход на экран создания нового заказа.}\\
				PRIOR: CRITICAL\\

			\paragraph{Фильтры}\mbox{}\\

				\sr{Область фильтров в операционной панели должна содержать все фильтры описанные в таблице \ref{order_filter_table}}\\
				PRIOR: CRITICAL\\

				\sr{По умолчанию журнал заказов отображает только активные заказы.}\\
				PRIOR: CRITICAL\\

				\sr{В области фильтров находится кнопка "Сброс", при нажатии на которую значения фильтров сбрасываются к настройкам по умолчанию.}\\
				PRIOR: MODERATE\\

				\begin{table}
	               \begin{center}
	               \caption {Таблица фильтров журнала заказов}
	               \label{order_filter_table}
	               \setlength{\extrarowheight}{2mm}
	               \begin{tabular}{|p{5cm}|p{10cm}|}

	                   \hline     \textbf{Фильтр}&\textbf{Требования} \\ [2mm]

	                   \hline Фильтр по времени (Интервальный) & \sr{С помощью фильрта по времени пользователь может задавать интервал дат. При выборе интервала в журнале должны отображаться только заказы время подачи которых попадает в заданный интервал.} PRIOR: MODERATE  \\ [2mm]
	                   \hline
	               \end{tabular}
	               \end{center}
	            \end{table}


		\subsubsection{Таблица заказов}\mbox{}\\

			\paragraph{Требования к таблице заказов}\mbox{}\\

				\sr{Таблица содержит колонки описанные в таблице \ref{columns_order_journal}.}\\
				PRIOR: CRITICAL\\

				\sr{При двойном нажатии на поле заказа всплывает модальное окно (\ref{order_modal_win}).}\\
				PRIOR: CRITICAL\\

			\setlength{\extrarowheight}{2mm}
			\begin{longtable}{|p{4cm}|p{9cm}|} 
				\caption{Поля в журнале заказов} \\

				\hline	\textbf{Поле}&\textbf{Описание} \\ [2mm]
				\endfirsthead
				\hline \textbf{Поле}&  \textbf{Описание}	
				\endhead

				\hline Номер заказа & Номер присваиваемый индивидуально каждому заказу при создании, в случае если заказ поступает от Яндекса или по другому каналу то номер генерируется и присваивается при поступлении заказа. \\[2mm]
				\hline Дата & Дата выполнения заказа в формате [Число].[Месяц].[Год].\\[2mm]
				\hline Остаток времени & Оставшееся время до контрольного, иными словами [контрольное время] - [теукущее время]. Формат - [хх мин].\\[2mm] 
				\hline Телефон & Номер телефона клиента.\\[2mm]
				\hline Статус & Статус заказа.\\[2mm]
				\hline Подача & Адрес подачи заказа. \\[2mm]
				\hline Прибытие & Адрес прибытия заказа.\\[2mm]
				\hline Время подачи & N/A \\[2mm]
				\hline Сумма & Приблизительная стоимость заказа.\\[2mm]
				\hline Позывной & Позывной водителя. \\[2mm]
				\hline Тариф & N/A \\[2mm]
				\hline Оплата & Способ оплаты заказа.\\[2mm]
				\hline Взят & Время закрепления заказа за водителем. \\[2mm]
				\hline На месте & Время установления водителем статуса “На месте”\\[2mm]

				\hline Как взят & Возможные варианты: \begin{itemize} 
															\item Робот далее через символ "тире"...:
																\begin{itemize}
																	\item Городской
																	\item Портовый
																	\item + [(Цепочка)], если заказа взят в цепочку.
																\end{itemize}
															\item Вручную
														  \end{itemize} \\[2mm]

				\hline Канал заказа & Возможные варианты: \begin{itemize} 
															\item Служба Такси
																\begin{itemize}
																	\item Диспетчерская
																	\item Мобильное приложение
																	\item Веб-сайт
																\end{itemize}
															\item Яндекс
															\item Юр. лицо
														  \end{itemize} \\[2mm]
				\hline
				\caption*{} \label{columns_order_journal}	
			\end{longtable}

		\subsubsection{Модальное окно заказа} \label{order_modal_win}

			\sr{Модальное окно делится на 3 части по вертикали (в соотношении 1:2:7): "Область вкладок"(\ref{tab_place_order_madal_win}), а содержимое каждой вкладки включает в себя "Операционная панель" и "Рабочее пространство".}\\
			PRIOR: MODERATE\\

			\paragraph{Область вкладок} \mbox{} \label{tab_place_order_madal_win}\\

				\begin{table} [h]
	               \begin{center}
	               \caption {Вкладки в модальном окне заказа}
	               \setlength{\extrarowheight}{2mm}
	               \begin{tabular}{|p{5cm}|p{10cm}|}
	                   \hline     \textbf{Название вкладки}&\textbf{Описание содержимого} \\ [2mm]

	                   \hline Заказ (по умолчанию) & Указаны параметры и настройки текущего заказа с возможностью редактировать их. Подробное описание вкладки описано в пункте \ref{current_order_modal_win}\\ [2mm]
	                   \hline
	               \end{tabular}
	               \end{center}
               \end{table}

            \paragraph{Заказ} \mbox{} \label{current_order_modal_win}\\
			 
            	\subparagraph{Операционная панель} \mbox{} \\

            		\sr{Область содержит все что описанно в таблице \ref{order_operation}} \\
            		PRIOR: CRITICAL \\

            		\begin{table} [h]
		               \begin{center}
		               \caption {Операции над заказом}
		               \label{order_operation}
		               \setlength{\extrarowheight}{2mm}
		               \begin{tabular}{|p{4cm}|p{3cm}|p{8cm}|}

		                   \hline     \textbf{Операция} & \textbf{Тип} & \textbf{Требования} \\ [2mm]


		                   \hline Снять водителя & Кнопка & \sr{В операционной панели расположена кнопка "Снять водителя", при нажатии на которую выполняется обработка снятия водителя с заказа.} \\ [2mm]

		                   \hline Отменить заказ & Кнопка & \sr{При нажатии на сервере выполняется обработка отмены заказа.} \\ [2mm]

		                   \hline Изменить статус & Выпадающий список + Кнопка & \begin{itemize} 
		                   																\item \sr{В выпадающем списке перечислены статусы заказа. По умолчанию стоит актуальный на текущий момент статус заказа.}PRIOR: MODERATE
		                   																\item \sr{При выборе статуса неактивная кнопка "Изменить", становится активной. При нажатии на кнопку на сервере выполняется обработка изменения статуса заказа.} PRIOR: MODERATE
		                  														\end{itemize}\\ [2mm]

		                   \hline
		               \end{tabular}
		               \end{center}
               		\end{table}	

            	\subparagraph{Рабочее пространство} \mbox{} \\

            		\begin{itemize}

            			\item{
		            		TITLE: Поля рабочего пространства.
		            		\sr{В рабочем пространстве перечислены все активные поля заказа. Описание находится в таблице \ref{order_form_table}.} \\
		            		PRIOR: CRITICAL \\
            			}

            			\item{
	            			TITLE: Редактирование полей.
		            		\sr{Пользователь может редактировать поля путем нажатия на поле ЛКМ.} \\
		            		PRIOR: CRITICAL \\
            			} 

            			\item{
		            		TITLE: Кнопка "Сохранить заказ".
		            		\sr{Под областью с полями расположена неактивная кнопка "Сохранить заказ", при редактировании одного из полей заказа кнопка становится активной. При нажатии на кнопку выполняется обновленный заказ предается на сервер СТ и на нем выполняется обработка изменения заказа(\ref{change_order_processor}).} \\
		            		PRIOR: MODERATE \\}

            		\end{itemize}
