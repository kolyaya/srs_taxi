\subsection{Справочники}
	
	DESC:  Справочник представляет собой таблицу содержащие те или иные данные выбранные из базы данных. Также предусмотрена возможность редактирования справочников которая включается при нажатии на кнопку “редактировать”. При этом всплывают кнопки “ОК” и “Отменить” при нажатии на первую все изменения заносятся в базу данных, при нажатии на вторую изменения откатываются.

	Требования:
		Редактировать справочник может только пользователь имеющий на это соответствующие привилегии.


	\subsubsection{Справочник водителей}

		DESC: На экране справочника водителей отображается таблица водителей которые работают в службе такси владельца. Экран делиться на две части: "Операционная панель" и "Таблица водителей". При нажатии на одного из водителей в таблице открывается окно в котором можно узнать подробную информацию о водителе.

		\paragraph{Операционная панель} \mbox{} \\

			\sr{В справочнике водителей оператор может зарегистрировать нового водителя нажав на кнопку “Добавить”. В появившемся окне оператор заполняет поля необходимые для регистрации нового водителя. Поля описаны в таблице \ref{table_profile_driver}.}
			PRIOR: CRITICAL \\

			\sr{Поиск водителей осуществляется при помощи строки полнотекстового поиска. По мере ввода символов в строку поиска таблица фильтруется на предмет вводных данных в режиме реального времени и оставляет только тех водителей ФИО которых соответствуют этим данным.} \\
			PRIOR: MODERATE \\


		\paragraph{Таблица водителей} \mbox{} \\

         	\sr{Таблица сортируется по колонкам, по средствам нажатия на первую строку ("Шапку") колонки.} \\
			PRIOR: MODERATE \\

			\sr{При двойном нажатии на одного из водителей открывается окно с информацией о водителе. Подробное описание этого окна находится в пункте \ref{profile_driver}}. \\
			PRIOR: CRITICAL

					\setlength{\extrarowheight}{2mm}
					\label{table_profile_driver}
                    \begin{longtable}{|p{5cm}|p{10cm}|}
                        \caption {Поля в профиле водителя}\\

                        \hline     \textbf{Поле}&\textbf{Описание} \\ [2mm]
                        \endfirsthead
                        \hline     \textbf{Поле}&\textbf{Описание} \\ [2mm]
                        \endhead

                        \hline  \textbf{\textit{Пользователь:}} & \\ [2mm]
	                        \hline  Позывной & Текстовое поле. \\ [2mm]
	                        \hline  Пароль   & Текстовое поле. \\ [2mm]
                        \hline  \textbf{\textit{Информация о водителе:}}  & \\ [2mm]
	                        \hline  ФИО   & Текстовое поле.\\ [2mm]
	                        \hline  Адрес   & Текстовое поле.\\ [2mm]
	                        \hline  Телефон   & Справа от поля оператор может нажать на кнопку “+” для создания дополнительного поля в случае если у водителя более одного активного номера. \\ [2mm]
	                        \hline  e-mail   & Текстовое поле.\\ [2mm]
                        \hline  \textbf{\textit{Условия работы:}}   & \\ [2mm]
	                        \hline  Тип водителя   & Выпадающий список в котором оператор выбирает “Привлеченник” или “Аренда”. \\ [2mm]
	                        \hline  Процентная ставка   & Выпадающий список \\ [2mm]
	                        \hline  Лимит   & Цифровое поле. Оператор устанавливает долговой лимит для водителя, при котором будетая панельирован доступ к резервированию заказов. По умолчанию значение долгового лимита равно 0 руб.\\ [2mm]
	                        \hline  Аренда   & Выпадающий список. Сумма списываемая с арендного баланса водителя ежедневно.	\\ [2mm]
                        \hline  \textbf{\textit{Водительское удостосерение:}}   & \\ [2mm]
	                        \hline  Серия   & N/A\\ [2mm]
	                        \hline  Номер   & N/A\\ [2mm]
	                        \hline  Дата выдачи   & N/A\\ [2mm]
	                        \hline  Действительный до & N/A\\ [2mm]
                        \hline \textbf{\textit{Транспортное средство}} & \\ [2mm]
                     	  	\hline ТС & В выпадающем списке водитель выбирает транспортное средство к которому привязывает водителя. \\ [2mm]
                        \hline
                    \end{longtable}

        \paragraph{Профиль водителя} \mbox{} \label{profile_driver} \\

        	DESC: Профиль водителя накладывается на элементы справочника. В операционной панели отображаются вкладки описанные в пунктах ниже, а вместо таблицы содержимое вкладки.\\

        	\sr{Вкладки расположены в том порядке, в котором они описаны в тех. спецификации.}\\
        	PRIOR: LOW\\

			\sr{Перед вкладками расположена кнопка "Назад", при нажатии на которую пользователь попдает обратно в справочник.}\\
			PRIOR: MODERATE\\

        	\subparagraph{Водитель} \mbox{} \\

        		DESC: Во вкладке "Водитель" расположны поля описанные в таблице \ref{table_profile_driver}, в которых записана информация о выбранном водителе.\\

        		\sr{Пользователь может редактировать информацию о водителе.}\\
        		PRIOR: CRITICAL\\

        		\sr{В нижней части экрана расположена кнопка "Сохранить изменения", кнопка становится активной если пользователь отредактировал одно из полей профиля водителей. При нажатии на кнопку все изменения сохраняются.}\\
        		PRIOR: CRITICAL\\

        	\subparagraph{ТС} \mbox{} \\

        	\subparagraph{Транзакции} \mbox{} \\

        	\subparagraph{Смены} \mbox{} \\

        		Также, перейдя по вкладке “Смена”, оператор может просмотреть периоды работы водителя. На экране присутствует фильтр по дате.

        	\subparagraph{Треки} \mbox{} \\

        		Оператор может посмотреть GPS-трек водителя.

        	\subparagraph{Заказы} \mbox{} \\

        		В окне есть возможность, перейдя по вкладке “Заказы”, посмотреть выполненные выбранным водителем заказы за выбранный промежуток, по умолчанию стоит 24 ч.

    \subsubsection{Справочник транспортных средств}

