\subsection{Административная панель}
		
		DESC:

		\sr{Доступ к административной панели может быть только у пользователя с ролью "Администратор"}
		PRIOR: CRITICAL

		\subsubsection{Управление пользователями}

				Экран управления пользователями делится на две части (в соотношении 2:8 по вертикали):
					\begin{itemize}
						\item Оперерационная панель, при помощи которой администратор проводит операции над пользователями.
						\item Таблица, содержащая в себе информацию о зарегистрированных пользователях. 
					\end{itemize}

				\paragraph{Операционная панель}\mbox{}\\
                	
                	Операционная панель содержит в себе следующие элементы: 
                		\begin{itemize}
							\item Добавление пользователя 
							\item Фильтры
							\item Операции 
						\end{itemize}


					\subparagraph{Добавление пользователя}\mbox{}\\ \label{dispatching_add_user}

						При нажатии на кнопку “Добавить пользователя” всплывает модальное окно для заполнения новой учетной записи. Поля необходимые для созданиня новой учетной записи перечислены в таблице \ref{user_profile_table_of_fields}.

						\begin{table}[!h]
							\begin{center}
							\caption {Поля в учетной записи пользователя}
							\label{user_profile_table_of_fields}
							\setlength{\extrarowheight}{2mm}
							\begin{tabular}{|p{5cm}|p{10cm}|}
							\hline     \textbf{}&\textbf{} \\ [2mm]

							\hline  ФИО   & Состоит из 3 полей. При вводе данных приложение проверяет валидность текста. \\ [2mm]
							\hline  Контактный телефон  &  Цифровое поле, при введении данных в которое происходит проверка на валидность вводимых значений.\\ [2mm]
							\hline  e-mail  & Текстовое поле. Вводимые данные проверяются на наличие символа “коммерческое at”.\\ [2mm]
							\hline  Логин & Текстовое поле. При вводе данных в приложении происходит проверка валидности и идентичность (проверка на совпадение) текста.\\ [2mm]
							\hline  Пароль   & Текстовое поле. Происходит проверка на выполнение условий ввода, заданных владельцем. \\ [2mm]
							\hline  Роль & Выпадающий список.\\ [2mm]
							\hline
							\end{tabular}
							\end{center}
						\end{table}

					\subparagraph{Фильтры}\mbox{}\\

						Общие требования:
							\begin{itemize}
								\item Фильтры учитывают фильтрацию друг друга. (Можно к примеру выполнить поиск по ФИО из Администраторов)
								\item Рядом с фильтрами расположена кнопка "Сброс". При нажатии на кнопку “Сброс” значения всех фильтров переходит в дефолтное (т.е. по умолчанию).
								\item При обновлении страницы все фильтры обнуляются. 
							\end{itemize}

		               \begin{table}[!h]  
			               \begin{center}
			               \caption {Фильтры по пользователям}
			               \setlength{\extrarowheight}{2mm}
			               \begin{tabular}{|p{5cm}|p{10cm}|}
			                   \hline     \textbf{Фильтр}&\textbf{Описание} \\ [2mm]

			                   \hline  Поиск по Логину/ФИО/e-mail & Строка поиска. Исходя из вводных данных в режиме реального времени фильтрует пользователей и отображает в таблице только тех кто соответствует вводным данным. \\ [2mm]
			                   \hline   Фильтр по роли  & Состоит из выпадающего списка, в котором перечислены роли. При выборе одного из пунктов фильтрует пользователей по ролям. \\ [2mm]
			                   \hline
			               \end{tabular}
			               \end{center}
		               \end{table} 

					\subparagraph{Операции}\mbox{}\\

						При выборе одного из пользователей (выбор происходит по нажатию), кнопки перечисленные в таблице \ref{user_managment_operations} становятся активными. 

		                \begin{table}[!h] 
			                \begin{center}
			                \caption {Операции над пользователями}
                            \label{user_managment_operations}
			                \setlength{\extrarowheight}{2mm}
			                \begin{tabular}{|p{5cm}|p{10cm}|} 
			                   \hline     \textbf{Операция (кнопка)}&\textbf{Действия при нажатии} \\ [2mm]

			                   \hline  Редактировать & Открывается экран формы регистрации (модальное окно), описанный в \ref{dispatching_add_user}, содержащий информацию о зарегистрированном пользователе. Администратор может внести необходимы изменения. Под формой расположены 2 кнопки “Сохранить” и “Отмена”. \\ [2mm]
			                   \hline  Изменить роль & Всплывает модальное окно, в котором предоставлен список ролей доступа с радиобоксами, по умолчанию установлена роль доступа, указанная при Регистрации. Под списком ролей расположены 2 клавиши “Сохранить” и “Отмена”. \\ [2mm]
			                   \hline Удалить & Всплывает модальное окно, в котором Администратор должен подтвердить действие по удалению Пользователя. В случае подтверждения, данные о Пользователе удаляются из Базы данных. \\ [2mm]
			                   \hline
			                \end{tabular}
			                \end{center}
		                \end{table}

				\paragraph{Таблица}\mbox{}\\
                	
                   Таблица состоит из колонок перечисленных в таблице \ref{users_table}.
                    
	               \begin{table}[!h]
	               \begin{center}
	               \caption {Таблица пользователей}
                   \label{users_table}
	               \setlength{\extrarowheight}{2mm}
	               \begin{tabular}{|p{5cm}|p{10cm}|}

	                   \hline  \textbf{Колонка таблицы}&\textbf{Описание} \\ [2mm]

	                   \hline  Логин & Логин пользователя. Так же это ссылка на профиль пользователя. \\ [2mm]
	                   \hline  ФИО & Текстовое поле, содержащее ФИО пользователя. \\ [2mm]
	                   \hline  E-Mail & Электронный ящик пользователя. \\ [2mm]
	                   \hline  Роль  & Текстовое поле, содержащее информацию о правах доступа пользователя. \\ [2mm]
	                   \hline
	               \end{tabular}
	               \end{center}
	               \end{table}

		\subsubsection{Редактор тарифов}

		\subsubsection{Редактор условий работы}

			DESC: Редактор условий работы предназначен для редактирования процентных ставок от заказов, и для редактирования условий аренды Транспортного средства у Службы Такси. \\
				
				\sr{Экран делиться на два блока: "Процентная ставка" и "Условия аренды"}\\
				PRIOR: MODERATE

			\paragraph{Процентная ставка} \mbox{} \\

				\sr{Администратор добавляет процентные ставки путем нажатия на кнопку "Добавить". При нажатии добовляется элемент в список процентных ставок.} \\
				PRIOR: MODERATE\\

				\sr{Список существующих процентных ставок отображается в виде [Процентная ставка] + [Элемент редактирования процентной ставки]}\\
				PRIOR: MODERATE\\

				\sr{Процентную ставку можно удалить путем нажатия на "Крестик" что расположен рядом с каждым элементом списка процентных ставок.}\\
				PRIOR: MODERATE\\

				\sr{Процентную ставку можно отредактировать путем нажатия на поле "Процентная ставка" и "руками" ввести новое значение}\\
				PRIOR: MODERATE\\

			\paragraph{Условия аренды} \mbox{} \\

				\sr{Список существующих условий аренды отображается в виде [Класс машины (Эконом, Комфорт, Бизнес)] + [Сумма за аренду ТС (В рублях)]}\\
				PRIOR: MODERATE\\

				\sr{Условие аренды можно отредактировать путем нажатия на поле "Сумма за аренду ТС (В рублях)" и "руками" ввести новое значение}\\
				PRIOR: MODERATE\\