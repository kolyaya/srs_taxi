\chapter{Общее описание}

		\section{Обзор продукта} 

		Система состоит из клиентского приложения, серверного приложения, диспетчерского веб-приложения, личного кабинета компании-партнера и базы данных. Мобильное приложение используется водителями для автоматической (с помощью механизма робота) или ручной работы с заказами. Серверное приложение принимает и обрабатывает заказы с трех каналов(Яндекс.Такси, диспетчерская, партнеры), обрабатывает статусы и координаты местоположения водителей, распределяет заказы между водителями используя алгоритмы распределения заказов, а также обработанные настройки водителя и параметры заказа, а также записывает обработанную информацию о водителях, клиентах, заказах и т.д. в базу данных. Личный кабинет предназначен для взаимодействия с партнерами владельца. Через диспетчерскую оператор может оформить заказ, мониторить местоположение, статус, деятельность водителей, а также состояние заказов. Так же оператор оповещается о произошедших происшествиях. Администратор службы такси с помощью диспетчерской может, по-мимо прочего редактировать тарифы, редактировать базу данных водителей, транспортных средств. Создавать группы пользователей, настраивать привилегии и создавать/удалять учетные записи пользователей. 

		\section{Ограничения и допущения}

		\begin{itemize}
		\item Мобильное приложение должно иметь соединение с Internet.
		\item Устройство на котором будет работать мобильное приложение должно иметь GPS
		\item Кодовая база серверного приложения должна быть написано на одном языке программирования.
		\item Все сторонние библиотеки, языки программирования, прочие инструменты разработки должны быть свободным программным обеспечением.
		\item Все водители должны находиться в городе Москва или в пределах Московской области
		\item Сервер должен работать под управлением ОС GNU/Linux 
		\item Сервер и мобильный клиент должны передавать данные по протоколу WebSocket
		\item Сервер и веб клиенты должны взаимодействовать по протоколу HTTP
		\item Мобильный клиент должен работать на платформе Android версии 2.3 и выше.
		\item Просчет расстояния и времени прибытия считается через Яндекс.Пробки. 
		\item Должна быть возможность вносить изменение в приложения не останавливая сервер.
		\item Время развертывания приложения должно быть менее двух минут.
		\end{itemize}

		
