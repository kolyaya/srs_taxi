

\chapter{Вступление}

		Эта часть дает описание всего что есть в документе, его цели (\ref{goals}). Также предоставлен список терминов и аббревиатур. 

		\section{Цель}\label{goals}

			Назначением этого документа - предоставить детальное описание функциональности автоматизированной системы. В нем детально описана каждая функция системы, так же предоставлен макеты интерфейса диспетчерской и водителей.

		\section{Контекст}

			Описываемое приложение представляет собой систему автоматизированную систему управления работающую в режиме мягкого реального времени. Система предназначена для автоматизации работы таксопарка, сотрудничества с партнерами службы такси, взаимодействия с сервисом Яндекс.Такси, оптимизации оперативности обслуживания клиентаентов таксопарка.

		\section{Термины, условные обозначения}

			\setlength{\extrarowheight}{2mm}
			\begin{longtable}{|p{4cm}|p{9cm}|}
				\hline  \textbf{Термин}&  \textbf{Определение} \\[2mm]
				\endfirsthead
				\hline \textbf{Термин}&  \textbf{Определение}	
				\endhead

				
				\hline  Владелец &  Служба такси являющаяся собственником данного программного комплекса.\\ [2mm]
				\hline  Клиент &  Физическое лицо осуществляющее заказ такси тем или иным способом.\\ [2mm]
				\hline  GPS &  Глобальная система позиционирования\\ [2mm]
				\hline  Робот &  Программный механизм автоматизирующий водительскую работу с заказами.\\ [2mm]
				\hline  QpS &  Метрика количества запросов  в секунду\\ [2mm]
				\hline  Служба-Партнер &  Служба такси которая сотрудничает с владельцем, предоставляя свои машины в случае если владелец не может обслужить принятый заказ. \\ [2mm]
			    \hline  Агрегатор-парнер &  Служба такси участвующая в раздаче заказов на равных условиях со службой-владельцем. \\ [2mm]
				\hline  Канал&  Путь по которому приходит заказ.\\ [2mm]
			    \hline  Роль&  Полномочия пользователя клиентской программы.\\ [2mm]
				\hline  Пользователь& Сотрудник службы Такси - оператор или администратор. \\ [2mm]
			    \hline  Параметры заказа&  Требования клиента к такси которое будет его обслуживать. Например есть или нет детское кресло или курящий или некурящий салон.\\ [2mm]
				\hline  Веб-сервис&  Логический веб интерфейс сервера предоставляющий по запросу те или иные данные или функциональность.\\ [2mm]
			    \hline  Цепочка&  Механизм подбора заказа для водителя с включенным роботом подбирающий заказ который будет взят водителем после завершения текущего, начальная точка которого близка к конечной точки текущего.\\ [2mm]
				\hline  Место назначения/Начальная точка&  Адрес куда клиент вызвал такси\\ [2mm]
			    \hline  Место прибытия/Конечная точка&  Адрес куда клиент собирается ехать\\ [2mm]
				\hline  Километраж&  Расстояние по дорогам с учетом пробок в пределах которого водитель желает получать заказы, настраиваемый водителем, при этом он не может быть меньше или больше, соответственно минимального или максимального значения задаваемого владельцем.\\ [2mm]
			    \hline  Контрольное время &  Время подачи машины на заказ.\\ [2mm]
				\hline  Радиус&  Эквивалентен километражу. Используется для вычисления диапазонов.\\ [2mm]
			    \hline  Максимальное время срочного заказа&  Текущее время плюс время установленное владельцем. \\ [2mm]
			    \hline	Срочный заказ &	Заказ время подачи которого не превышает максимальное время срочного заказа.\\ [2mm]
			    \hline	Предварительный заказ&	Заказ время подачи которого превышает максимальное время срочного заказа.\\ [2mm]
			    \hline	Безналичные заказы&	Заказы поступающие в Диспетчерскую от юридического лица. Заказ выполняется только владельцем. \\ [2mm]
			    \hline	* &	Значок, который обозначает, что поле является обязательным для заполнения.\\ [2mm]
			    \hline	Долговой лимит&	Лимит, установленный доверенным лицом владельца индивидуально каждому водителю.\\ [2mm]
			    \hline	Обнуление &	Переход к дефолтному значению. \\ [2mm]
			    \hline Долговой лимит & Нижнее значение суммы на балансе водителя при которой он может принимать заказы, после превышения которой водитель переводиться в статус “Заблокирован”. \\ [2mm]
			    \hline Активный заказ & С любым статусом кроме "Завершен". \\ [2mm]

			    \hline Активные поля & Поля которые отображаются в пользовательском интерфейсе. \\ [2mm]
				\hline Пассивные поля & Поля которые не отображаются в пользовательском интерфейсе, но используются при формировании файла (Заказ и тд.). \\ [2mm]

				\hline 
			\end{longtable} 

			\section{Приоритеты}

				Здесь находится описание приоритетов требований программного комплекса.

				\begin{description}
					\item[CRITICAL] - Высокий приоритет. Это означает что без этого требования программный комплекс работать не будет. Так же это означает что разработчик должен выполнить задачу строго в соответствии с требованием.
					\item[MODERATE] - Средний приоритет. Это означает что разработчик должен выполнить задачу в соответствии с требованием, но может изменить принцип работы функциональности если результат окажется таким же. 
					\item[LOW] - Низкий приоритет. Это означает что требование не критично для работы проекта.
				\end{description}

			\section{Константные значения}

                \begin{table}[h]
	                \begin{center}
	                \caption {Константные значения}
	                \setlength{\extrarowheight}{2mm}
	                \begin{tabular}{|p{3cm}|p{6cm}|p{4cm}|}
	                   \hline     \textbf{ID} & \textbf{Описание} & \textbf{Значение}\\ [2mm]


	                   \hline \stat{min_time_of_filing}{} & Минимальное время подачи машины.  & 25 мин.\\ [2mm]

	                   \hline \stat{display_duration_of_the_notification}{} & Продолжительность отображения уведомления.  & 5 сек.\\ [2mm]

	                   \hline \stat{timeout_waiting_for_a_response_from_the_driver_request}{} & Таймаут ожидания ответа от запроса водителю.  & 30 сек.\\ [2mm]
	                   
	                   \hline

	                \end{tabular}
	                \end{center}
                \end{table}