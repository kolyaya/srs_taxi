\section{Таблица взаимоисключащих режимов}

	DESC: Режимы роботов взаимоисключающие.Режимы роботов также ограничены при выборке нескольких режимов - если водитель выбрал один из режимов в первом столбце (см. ниже “Таблица взаимоисключающих режимов”), то под опциями этого режима активируются чекбоксы с наименованиями активных режимов для данного, нажав на которые активируются опции выбранного режима. К примеру, водитель выбрал “Поиск по адресу” и настроил, далее нажал на чекбокс с режимом “Городской” и настроил этот режим, далее выбрал под опциями “Городского” - “Хочу домой” и настроил. Затем включил робота.

	\subsection{Таблица взаимоисключающих режимов}

		\begin{table}[htb]
	        \begin{center}
	        \caption{Таблица взаимоисключающих режимов (Активен / Неактивен)}
	        \label{appendices_termins}
	        \setlength{\extrarowheight}{2mm}
	        \begin{tabular}{|p{3cm}|p{2cm}|p{2cm}|p{2cm}|p{3cm}|}
	           \hline   \textbf{}&\textbf{Городской}&\textbf{Портовый}&\textbf{Поиск по адресу}&\textbf{Хочу домой(Поиск по месту прибытия)} \\ [2mm]


	           \hline Городской & - & Активен & Неактивен & Активен\\ [2mm]

	           \hline Портовый & Активен & - & Неактивен & Активен\\ [2mm]

	           \hline Городской & Активен & Активен & - & Активен\\ [2mm]

	           \hline Городской & Неактивен & Неактивен & Неактивен & -\\ [2mm]    

	           \hline
	        \end{tabular}
	        \end{center}
      	\end{table}